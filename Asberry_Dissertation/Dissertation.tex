%%%%%%%%%%%%%%%%%%%%%%%%%%%%%%%%%%%%%%%%%
% Masters/Doctoral Thesis 
% LaTeX Template
% Version 2.5 (27/8/17)
%
% This template was downloaded from:
% http://www.LaTeXTemplates.com
%
% Version 2.x major modifications by:
% Vel (vel@latextemplates.com)
%
% This template is based on a template by:
% Steve Gunn (http://users.ecs.soton.ac.uk/srg/softwaretools/document/templates/)
% Sunil Patel (http://www.sunilpatel.co.uk/thesis-template/)
%
% Template license:
% CC BY-NC-SA 3.0 (http://creativecommons.org/licenses/by-nc-sa/3.0/)
%
% Modifications made by Drake Asberry to make document compliant with 
% University of Arizona dissertation formatting Guide 2019-01-29
% https://grad.arizona.edu/gsas/dissertations-theses/dissertation-and-thesis-formatting-guides
%
%%%%%%%%%%%%%%%%%%%%%%%%%%%%%%%%%%%%%%%%%

%----------------------------------------------------------------------------------------
%	PACKAGES AND OTHER DOCUMENT CONFIGURATIONS
%----------------------------------------------------------------------------------------

\documentclass[
12pt, % The default document font size, options: 10pt, 11pt, 12pt
%oneside, % Two side (alternating margins) for binding by default, uncomment to switch to one side
english, % ngerman for German
doublespacing, % Single line spacing (singlespacing), alternatives: onehalfspacing or doublespacing
%draft, % Uncomment to enable draft mode (no pictures, no links, overfull hboxes indicated)
nolistspacing, % If the document is onehalfspacing or doublespacing, uncomment this to set spacing in lists to single
liststotoc, % Uncomment to add the list of figures/tables/etc to the table of contents
%toctotoc, % Uncomment to add the main table of contents to the table of contents
%parskip, % Uncomment to add space between paragraphs
%nohyperref, % Uncomment to not load the hyperref package
headsepline, % Uncomment to get a line under the header
chapterinoneline, % Uncomment to place the chapter title next to the number on one line
%consistentlayout, % Uncomment to change the layout of the declaration, abstract and acknowledgements pages to match the default layout
openany, % eliminates all extra blank pages from entire document
]{DoctoralThesis}\usepackage[]{graphicx}\usepackage[]{color}
% maxwidth is the original width if it is less than linewidth
% otherwise use linewidth (to make sure the graphics do not exceed the margin)
\makeatletter
\def\maxwidth{ %
  \ifdim\Gin@nat@width>\linewidth
    \linewidth
  \else
    \Gin@nat@width
  \fi
}
\makeatother

\definecolor{fgcolor}{rgb}{0.345, 0.345, 0.345}
\newcommand{\hlnum}[1]{\textcolor[rgb]{0.686,0.059,0.569}{#1}}%
\newcommand{\hlstr}[1]{\textcolor[rgb]{0.192,0.494,0.8}{#1}}%
\newcommand{\hlcom}[1]{\textcolor[rgb]{0.678,0.584,0.686}{\textit{#1}}}%
\newcommand{\hlopt}[1]{\textcolor[rgb]{0,0,0}{#1}}%
\newcommand{\hlstd}[1]{\textcolor[rgb]{0.345,0.345,0.345}{#1}}%
\newcommand{\hlkwa}[1]{\textcolor[rgb]{0.161,0.373,0.58}{\textbf{#1}}}%
\newcommand{\hlkwb}[1]{\textcolor[rgb]{0.69,0.353,0.396}{#1}}%
\newcommand{\hlkwc}[1]{\textcolor[rgb]{0.333,0.667,0.333}{#1}}%
\newcommand{\hlkwd}[1]{\textcolor[rgb]{0.737,0.353,0.396}{\textbf{#1}}}%
\let\hlipl\hlkwb

\usepackage{framed}
\makeatletter
\newenvironment{kframe}{%
 \def\at@end@of@kframe{}%
 \ifinner\ifhmode%
  \def\at@end@of@kframe{\end{minipage}}%
  \begin{minipage}{\columnwidth}%
 \fi\fi%
 \def\FrameCommand##1{\hskip\@totalleftmargin \hskip-\fboxsep
 \colorbox{shadecolor}{##1}\hskip-\fboxsep
     % There is no \\@totalrightmargin, so:
     \hskip-\linewidth \hskip-\@totalleftmargin \hskip\columnwidth}%
 \MakeFramed {\advance\hsize-\width
   \@totalleftmargin\z@ \linewidth\hsize
   \@setminipage}}%
 {\par\unskip\endMakeFramed%
 \at@end@of@kframe}
\makeatother

\definecolor{shadecolor}{rgb}{.97, .97, .97}
\definecolor{messagecolor}{rgb}{0, 0, 0}
\definecolor{warningcolor}{rgb}{1, 0, 1}
\definecolor{errorcolor}{rgb}{1, 0, 0}
\newenvironment{knitrout}{}{} % an empty environment to be redefined in TeX

\usepackage{alltt} % The class file specifying the document structure

\usepackage[utf8]{inputenc} % Required for inputting international characters

\usepackage[T1]{fontenc} % Output font encoding for international characters

\usepackage{mathpazo} % Use the Palatino font by default

\usepackage[backend=bibtex,style=authoryear,natbib=true]{biblatex} % Use the bibtex backend with the authoryear citation style (which resembles APA)

\addbibresource{references.bib} % The filename of the bibliography

\usepackage[autostyle=true]{csquotes} % Required to generate language-dependent quotes in the bibliography

\usepackage{todonotes} % Make renderable notes in pdf
%----------------------------------------------------------------------------------------
%	MARGIN SETTINGS
%----------------------------------------------------------------------------------------

\geometry{
	paper = letterpaper, % Change to a4paper for international letter
	inner = 1in, % Inner margin
	outer=1in, % Outer margin
	bindingoffset=.5in, % Binding offset
	top=1in, % Top margin
	bottom=1in, % Bottom margin
	%showframe, % Uncomment to show how the type block is set on the page
}

%----------------------------------------------------------------------------------------
%	THESIS INFORMATION
%----------------------------------------------------------------------------------------
% Replace content within green curly brackets to reflect your own work and names

\thesistitle{Syllabification, Visual Segmentation and Visual Word Recognition in Spanish–English Bilinguals} % Your thesis title, this is used in the title, committee approval and abstract pages, print it elsewhere with \ttitle

\chair{Dr. Miquel Simonet} % Your dissertation chair's name, this is used in the title, abstract and committee approval pages, print it elsewhere with \chairname

\cochair{Co-Chair Name} % Your dissertation co-chair's name, this is not currently used anywhere in the template, but would be used in the title and committee approval page, print it elsewhere with \cochairname

\examiner{Examiner Name} % Your examiner's name, this is not currently used anywhere in the template, print it elsewhere with \examname

\degree{Doctor of Philosophy} % Your degree name, this is used in the title page and abstract, print it elsewhere with \degreename

\author{Drake Asberry} % Your name, this is used in the title page and abstract, print it elsewhere with \authorname

\addresses{Your Address} % Your address, this is not currently used anywhere in the template, print it elsewhere with \addressname

\subject{Hispanic Linguistics} % Your subject area, this is not currently used anywhere in the template, print it elsewhere with \subjectname

\keywords{} % Keywords for your thesis, this is not currently used anywhere in the template, print it elsewhere with \keywordnames

\university{University of Arizona} % Your university's name and URL, this is used in the title page and abstract, print it elsewhere with \univname

\department{Graduate Interdisciplinary Program in Second Language Acquisition and Teaching} % Your department's name and URL, this is used in the title page and abstract, print it elsewhere with \deptname

\group{\href{http://researchgroup.university.com}{Research Group Name}} % Your research group's name and URL, this is not currently used anywhere in the template, print it elsewhere with \groupname

\facultyA{Dr. Michael Hammond} % Your first faculty member's name and URL(can be added between green curly brackets before the member's name. Do not use this for your COMMITTEE CHAIR This is used in the committee approval page, print it elsewhere with \facnameA

\facultyB{Dr. Adam Ussishkin} % Your first faculty member's name and URL(can be added between green curly brackets before the member's name. This is used in the committee approval page, print it elsewhere with \facnameB

\facultyC{Dr. Maryia Fedzechkina} % Your first faculty member's name and URL(can be added between green curly brackets before the member's name. This may optionally be added to the committee approval page when number of committee members requires it, print it elsewhere with \facnameC

\facultyD{5th Committee Member Name} % Your first faculty member's name and URL(can be added between green curly brackets before the member's name. This may optionally be added to the committee approval page when number of committee members requires it, print it elsewhere with \facnameD

\defense{April 15, 2020} % This should have the long date format of your scheduled defense, print it elsewhere with \defensedate

\AtBeginDocument{
\hypersetup{pdftitle=\ttitle} % Set the PDF's title to your title
\hypersetup{pdfauthor=\authorname} % Set the PDF's author to your name
\hypersetup{pdfkeywords=\keywordnames} % Set the PDF's keywords to your keywords
\hypersetup{hidelinks} % Set all hyperlinks standard color text
}

%\usepackage{times} % Uncomment to use Times New Roman font
\IfFileExists{upquote.sty}{\usepackage{upquote}}{}
\begin{document}

%\frontmatter % Uncomment to use roman page numbering style (i, ii, iii, iv...) for the pre-content pages

\pagestyle{plain} % Default to the plain heading style until the thesis style is called for the body content


%-------------------------------------------------------------------------------------------------------------------------------
%	Pre-Thesis Content
%-------------------------------------------------------------------------------------------------------------------------------

%----------------------------------------------------------------------------------------
%	TITLE PAGE
%----------------------------------------------------------------------------------------
%
% All requirements of the graduate college as of 2019-01-30
% The title page must be the first page of your document (All pages must be numbered and match 
% the numbers listed in Table of Contents. However, a page number is not required to be printed
% on the actual title page).
% The title page must meet the following requirements:
% Title is set in ALL CAPS
% Student name matches official name in UAccess
% Rule line appears
% Official Department Name is Used
% Degree is indicated correctly
% Copyright year matches year of graduation on page

\begin{titlepage}
\begin{singlespacing} % needed for documents set to 1.5 or 2.0 spacing, Comment out otherwise
\begin{center}

\vfill

\MakeUppercase{\ttitle}\\ %Thesis Title in ALL CAPS
\vspace{0.4in}
by\\ \vspace{0.4in}
{\authorname}\\ % Places author name as specified in preamble
\vspace{0.6in}
\HRule \\[0.1cm] % Horizontal line
Copyright \textcopyright\space\authorname\space{\the\year}\\ % Copyright Date

\vspace{0.4in}

A Dissertation Submitted to the Faculty of the\\ % University required text
\vspace{0.4in}
\MakeUppercase{\deptname} \\  % Department name in Small Caps
\vspace{0.4in}
In Partial Fulfillment of the Requirements \\ \medskip % University required text
For the Degree of \\  % University required text
\vspace{0.4in}
\MakeUppercase{\degreename} \\ % Thesis type
\vspace{0.4in} 
In the Graduate College \\  % University required text
\vspace{0.4in}
\MakeUppercase{The \univname} \\ % University name in Small Caps
\vspace{0.6in}
%\normalsize
{\the\year}\\[4cm] % date
%\includegraphics{Logo} % University/department logo - uncomment to place it

\vfill
\end{center}
\end{singlespacing}% needed for documents set to 1.5 or 2.0 spacing, Comment out otherwise
\end{titlepage}

%\cleardoublepage %Uncomment to add blank page after Title page.


\setcounter{page}{2} % Starts pagination at 2 on the Committee Approval Form with no page number displayed on Title page.

%----------------------------------------------------------------------------------------
%	COMMITTEE APPROVAL PAGE
%----------------------------------------------------------------------------------------
%
% All requirements of the graduate college as of 2019-01-30
% The committee approval page must be the second page of your document
% The committee approval page must meet the following requirements:
% Title on approval page matches title on page 1 (Title Page)
% Dissertation chair (or co-chair) is indicated
% All members and chair (or co-chairs) have signed the approval page
% Date of defense is listed

%\addchaptertocentry{Committee Approval Page} % Add the committee approval page to the table of contents
\begin{singlespacing} % needed for documents set to 1.5 or 2.0 spacing, Comment out otherwise
\begin{center}
%\large

THE \MakeUppercase{\univname} \\
GRADUATE COLLEGE
\end{center}

\vspace*{0.3in}

\noindent As members of the Dissertation Committee, we certify that we have read the dissertation prepared by \authorname \space entitled "\ttitle "\space and recommend that it be accepted as fulfilling the dissertation requirement for the Degree of \degreename.

\vspace*{0.3in}

\noindent\underline{\makebox[4.0in][r]{}} \hspace{0.4in} Date: \defensedate \\
{\bfseries\chairname}\\
\emph{(Chair)}
\vspace*{0.3in}

\noindent\underline{\makebox[4.0in][r]{}} \hspace{0.4in} Date: \defensedate \\
{\bfseries\facnameA}\\
\emph{(Member)}
\vspace*{0.3in}

\noindent\underline{\makebox[4.0in][r]{}} \hspace{0.4in} Date: \defensedate \\
{\bfseries\facnameB}\\
\emph{(Member)}
\vspace*{0.3in}

\noindent\underline{\makebox[4.0in][r]{}} \hspace{0.4in} Date: \defensedate \\
{\bfseries\facnameC}\\
\emph{(Member)}
\vspace*{0.5in}

% If 4th committee member is needed, copy the preceding 4 lines, change to facnameD in copied lines
% You will then need to adjust vertical spacing to keep committee approval page to 1 page length

\noindent Final approval and acceptance of this dissertation is contingent upon the candidate's submission of the final copies of the dissertation to the Graduate College.

\vspace*{0.2in}

\noindent I hereby certify that I have read this dissertation prepared under my direction and recommend that it be accepted as fulfilling the dissertation requirement.
\vspace*{0.5in}

\noindent\underline{\makebox[4.0in][r]{}} \hspace{0.4in} Date: \defensedate \\
Dissertation Director: \chairname \\
%{\bfseries \emph{Instructor \\ Hispanic Linguistics}} % Update hard-coded to job title and department
\vfill
\end{singlespacing}% needed for documents set to 1.5 or 2.0 spacing, Comment out otherwise



%----------------------------------------------------------------------------------------
%	STATEMENT BY AUTHOR
%----------------------------------------------------------------------------------------
%
% No longer required for the graduate college as of 2019-01-30
% Uncomment all lines in this section  with "%%" at the beginning if your document requires it

%%\begin{statement}
%%\begin{singlespacing} % needed for documents set to 1.5 or 2.0 spacing, Comment out otherwise
%%\addchaptertocentry{\authorshipname} % Add the declaration to the table of contents

%The following block of text was the required text of the Graduate College (2019-02-01)
%%This dissertation has been submitted in partial fulfillment of the requirements for an advanced degree at the \univname\space and is deposited in the University Library to be made available to borrowers under rules of the Library. \\ \smallskip 

%%Brief quotations from this dissertation are allowable without special permission, provided that an accurate acknowledgement of the source is made. Requests for permission for extended quotation from or reproduction of this manuscript in whole or in part may be granted by the copyright holder.

%%\vspace*{0.3in}
%%\begin{center} 
%%SIGNED: \authorname
%%\end{center}
%%\end{singlespacing}% needed for documents set to 1.5 or 2.0 spacing, Comment out otherwise
%%\end{statement}


%----------------------------------------------------------------------------------------
%	ACKNOWLEDGEMENTS
%----------------------------------------------------------------------------------------
%
% Acknowledgements are not a necessary item. Comment out if not being used 
\include{FrontBackMatter/Acknowledgements} % This calls the Acknowledgements.tex 


%----------------------------------------------------------------------------------------
%	DEDICATION
%----------------------------------------------------------------------------------------
%
% Dedications are not a necessary item. Comment out to remove from document
\dedicatory{For my family and friends. \\\bigskip Dedicated to my Granny and Pa who I lost during my time in Arizona. They have always supported me and rejoiced in my accomplishments. I only wish that they were still here today to celebrate the completion of this dissertation.} 


%----------------------------------------------------------------------------------------
%	Quotation
%----------------------------------------------------------------------------------------
%
% This page is not really necessary, but if you feel the need to include some quote here is your
% chance. 
%
%\include{FrontBackMatter/Quotation} % Uncomment to use. This calls the Quotation.tex


%----------------------------------------------------------------------------------------
%	LIST OF CONTENTS/FIGURES/TABLES PAGES
%----------------------------------------------------------------------------------------
%
% Table of Contents (TOC) must include:
% a: all major sections with the document in a consistent manner
% b: section headings in document must match their listings (exact words) in TOC
%
\tableofcontents % Prints the main table of contents
%
% Lists of figures and tables must include accurate page numbers
\listoffigures % Prints the list of figures

\listoftables % Prints the list of tables


%----------------------------------------------------------------------------------------
%	ABBREVIATIONS
%----------------------------------------------------------------------------------------
%
% Abbreviations are not a necessary item. Comment out the line below to remove from document
%
%\include{FrontBackMatter/Abbvreviations} % This calls the Abbreviations.tex Uncomment to include


%----------------------------------------------------------------------------------------
%	PHYSICAL CONSTANTS/OTHER DEFINITIONS
%----------------------------------------------------------------------------------------
%
% Constants are not a necessary item. Comment out the line below to remove from document
%
%\include{FrontBackMatter/Constants} % This calls the Constants.tex Uncomment to include


%----------------------------------------------------------------------------------------
%	SYMBOLS
%----------------------------------------------------------------------------------------
%
% Symbols are not a necessary item. 
%
%\include{FrontBackMatter/Symbols} % This calls the Symbols.tex Uncomment to include


%----------------------------------------------------------------------------------------
%	Abstract
%----------------------------------------------------------------------------------------
% This is required to appear before the first chapter of the dissertation

\include{FrontBackMatter/Abstract} % This calls the Abstract.tex


%----------------------------------------------------------------------------------------
%	THESIS CONTENT - CHAPTERS
%----------------------------------------------------------------------------------------

%\mainmatter % Begin numeric (1,2,3...) page numbering Uncomment if using roman numerals in front matter


\pagestyle{thesis} % Return the page headers back to the "thesis" style

% Include the chapters of the thesis as separate files from the Chapters folder
% Uncomment the lines as you write the chapters
%\renewcommand{\chaptermarkformat}{\thechapter}


% Chapter 1 with LaTeX code only

%----------------------------------------------------------------------------------------

%----------------------------------------------------------------------------------------


%----------------------------------------------------------------------------------------
% Load statistics in memory from separate file
%<<content1, child='stat.rnw'>>=
@
%----------------------------------------------------------------------------------------


\chapter{Introduction} % Main chapter title

\label{Chapter1} % Change X to a consecutive number; for referencing this chapter elsewhere, use \ref{Chapter2}

%----------------------------------------------------------------------------------------
%	SECTION 0
%----------------------------------------------------------------------------------------
% Not needed for introduction
%\section{Abstract}


%----------------------------------------------------------------------------------------
%	SECTION 1
%----------------------------------------------------------------------------------------

\section{Introduction}

This three article dissertation was written in order to complete the degree requirements of the Second Language Acquisition and Teaching program (SLAT) at the University of Arizona. The participant population for this dissertation are three distinct groups early and late Spanish–English bilingual speakers. The early bilingual group consists of speakers who began learning before the age of %INSERT AGE%. 
One of the late bilingual groups consists of native English speakers who were L2 learners of Spanish while the other consists of native Spanish speakers who were L2 learners of English. The dissertation investigates the ability of these three speaker populations and their ability to use the linguistic unit, a syllable, in their language processing strategies for Spanish. The three articles that compose this dissertation---chapters 2, 3 and 4---each correspond to one independent article and utiliize the same three bilingual speaker populations discussed above.

%-----------------------------------
%	SECTION 2
%-----------------------------------
\section{Background}
%\emph{this may be too direct of a start for the introduction, maybe something a little lighter or broad scheme (funnel approach).}
A syllable is a pronounceable linguistic unit of a given language that generally contains a highly sonorant sound as it nucleus—typically a vowel in most languages. Early research sought to discover the \emph{minimal perceptual unit} and the syllable was a logical and testable linguistic unit. In spoken language processing, the syllable made its highlight when it was found that participants could detect syllables faster than they could detect individual phonemes of which the syllable was comprised \citep*{Savin1970-oy}. This finding spurred interest in researchers focus on the syllable as well as other linguistic units such as word, phrases and sentences. The findings of these additional studies revealed that the processes involved in parsing spoken language were complex and a minimal perceptual unit was unlikely to be found. %\textbf{Should probably give more details about these studies here}.%
Specifically, one linguistic unit—phoneme, syllable, word, etc.—could not be the sole mechanism in which listeners of language break the speech stream into smaller or processable chunks \citep{Foss1973-ll,Healy1976-js,McNeill1973-bo}. 

Mehler and Hayes \parencite*{Mehler1981-wp} captured the need for a change in the direction of research regarding the syllable, “Traditionally, psycholinguistics research has invested the bulk of its efforts into uncovering the units used in speech processing. Although it is currently fashionable to claim that such work is pointless since it has no very clear outcome, many of the more meaningful advances in the field have come from projects whose framework included the problem of processing units.” They went on to delineate two different levels in which the research around the syllable could move forward: (1) The syllable as a phonological unit of the language which can efficiently explain the grammar of language and (2) The syllable as a unit which aides speech perception and language comprehension. As a result in the early 1980s, several researchers began refocusing their own investigations in accordance to this second vein of syllable research. %\textbf{Should cite and describe additional studies here}.% 
Even as researchers conducted more pointed research on the syllable's role in language processing strategies, the results continued to suggest research questions needed further subdivision. Ultimately, two distinct subprocesses of language processing in which the syllable may play a role---segmentation and lexical access---were proposed. 
% discuss the differences between segmentation and lexical access here.


%-----------------------------------
%	SECTION 3
%-----------------------------------

\section{Present Study}

This dissertation utilizes several different methodologies as a means to investigate research questions that fall under Mehler and Hayes \parencite*{Mehler1981-wp} second level of research. There are three overarching questions that underlie this dissertation project as a whole:
\begin{enumerate}
%revisit these research questions later
\item Whether or not the common linguistic unit—the syllable—is available to Spanish–English bilinguals when processing the Spanish language?
\item Whether or not the syllable is a strategy used by Spanish–English bilinguals in language segmentation and/or lexical access?
\item Does the age of acquisition, early versus late bilinguals, affect the ability and efficiency in which Spanish–English bilinguals can make use of the syllable?
\end{enumerate}

The three articles in this dissertation provide additional information about syllable structure and the representation of the syllable in Spanish to the knowledge base of the field. Within the segmentation strain of research, investigating differing types of bilingualism of the speaker who have the same two languages at their disposal provides new information to the fields of bilingualism and second language acquisition. At the time of writing this dissertation, the majority of research studies have been conducted with monolingual speakers. When studies have investigated the role of the syllable in language processing by bilinguals, they have generally compared the bilinguals against monolingual speakers of the two respective languages. However, previous studies suggest that bilinguals are not simply the summation of two independent monolingual speakers. For example, %Cutler citation% 
found that French--English bilinguals listening to French responded in the same manner as French monolinguals, but when listening to English, these speakers did not respond similarly to English monolinguals. Therefore, this dissertation does not utilize monolingual Spanish or monolingual English speaker controls, but instead utilizes three distinct bilingual populations of Spanish and English which are compared against each other. %\textbf{This type of setup allows for the control group to be the Spanish native L2 English speaker group and two test groups be early and late learners of Spanish.}% 
Many previous studies on lexical access have generally been conducted using the visual word recognition paradigm to study speech processing in conjunction with the syllable. Since the visual word recognition tasks have been conducted with both monolingual and bilingual populations, this dissertation seeks to add to the knowledge base of the syllable’s role through the replication of previous findings while testing a different population of bilingual speakers under the visual word recognition paradigm.

The format for the remaining chapters of this dissertation project will be as follows in order to address the main overarching questions: chapter 2 explores the representation of the syllable in the minds of Spanish–English bilinguals with a two option forced-choice syllabic intuition task. Chapter 3 utilizes a visual word segmentation task to compare the efficiency of the processes employed by the bilingual speakers. Chapter 4 includes a visual priming experiment with a lexical decision task that explores the syllable’s role in lexical access by Spanish–English bilinguals. Chapter 5 then concludes the dissertation project by drawing overall conclusions and how all three individual studies were necessary to draw the conclusions that were borne out through the various testing methodologies used to explore the role of the syllable in Spanish language processing in the three articles of the dissertation.


%----------------------------------------------------------------------------------------
%	SECTION 2
%----------------------------------------------------------------------------------------

%\section{Methods}

%Not needed

%-----------------------------------
%	SUBSECTION 2-1
%-----------------------------------
%\subsection{Participants}

%Probably still important here to some degree

%Give accurate descriptions of participant groups, how they were classified, number of participants, etc. 


%----------------------------------------------------------------------------------------
%	SECTION 3
%----------------------------------------------------------------------------------------

%\section{Results}

%Not needed


%----------------------------------------------------------------------------------------
%	SECTION 4
%----------------------------------------------------------------------------------------

%\section{Discussion}
%Not needed


%----------------------------------------------------------------------------------------
%	SECTION 5
%----------------------------------------------------------------------------------------

\section{References}

%Insert references for this chapter here.










% Chapter 2 with LaTeX code only

%----------------------------------------------------------------------------------------

%----------------------------------------------------------------------------------------


%----------------------------------------------------------------------------------------
% Load statistics in memory from separate file
%<<content1, child='stat.rnw'>>=
@
%----------------------------------------------------------------------------------------

\chapter{Syllabification} % Main chapter title

\label{Chapter2} % Change X to a consecutive number; for referencing this chapter elsewhere, use \ref{Chapter2}

%----------------------------------------------------------------------------------------
%	SECTION 0
%----------------------------------------------------------------------------------------

\section{Abstract}

Give Syllabification abstract here

Keywords: (list all words necessary)

%----------------------------------------------------------------------------------------
%	SECTION 1
%----------------------------------------------------------------------------------------

\section{Introduction}

%Talk about syllabificationdifferences in syllabification between spanish and english

The syllable intuition experiment is an important step in the process because how the three groups of participants syllabify the words will have a direct effect on how fast or useful the syllable is in their segmentation strategy. It is expected that syllabic intuitions will vary based on being a native speaker of Spanish, an early learner of Spanish or a late learner of Spanish. These differences are likely to stem from age of acquisition, language dominance or type of schooling. This experiment will begin to build a data source of Spanish–English bilinguals syllabification of Spanish words that will help to determine whether or not age of acquisition or language dominance are sources are different syllabification patterns. 


%\textbf{DELETE ME Write my introduction to the syllabification article here (Will write the introduction LAST) The section of your article most likely to be read, not skimmed or skipped. The first paragraph or two is the overview of the article: describe the problem, question or theory motivating the research.} 

%-----------------------------------
%	SUBSECTION 1-1
%-----------------------------------
\subsection{Background}

%In the second part, describe relevant theories, review past research and give more details on the current research question. Do not forget about signposting, which headings and subheadings can naturally create for the reader.

%-----------------------------------
%	SUBSECTION 1-2
%-----------------------------------

\subsection{Present Study}
%Treiman


\begin{figure}[th]
\centering
\includegraphics{Figures/three_way_test}
\decoRule
\caption[A test of figures]{testing 1, 2, 3}
\label{fig:Test}
\end{figure}


%\textbf{DELETE ME The third section titled, “The Present Experiment” or “The Present Research”, follows and contains experimental descriptions and how they address the questions being asked. For most articles, keep your introduction under 10 pages}

%----------------------------------------------------------------------------------------
%	SECTION 2
%----------------------------------------------------------------------------------------

\section{Methods}

%Write this section FIRST is the section that describes how the research was conducted. A good one shows how well thought out the experiment design is and allows other researchers to easily replicate it. This section also follows a formula:

%-----------------------------------
%	SUBSECTION 2-1
%-----------------------------------
\subsection{Participants}

The participants were split into three distinct populations of Spanish–English bilinguals. The Spanish–dominant group consisted of native speakers of Spanish that lived in Sonora, Mexico and learned English after %INSERT AGE%. 
The English–dominant group were native English speakers that lived in Tucson, Arizona and learned Spanish after%INSERT AGE%. 
The early bilingual group consisted of participants who lived in Tucson, Arizona, but were exposed to both English and Spanish before the %INSERT AGE%.
\emph{The participants will be placed into their appropriate group based off of the combined results of Bilingual Language Profile (BLP), the LexTALE and the LexTALE-Esp vocabulary tests.} 

Based off of responses in the BLP, participants who report proficiency in a language other than English and Spanish will be removed from the data analysis. According to the findings of Lemhöfer \& Broersma \parencite*{Lemhofer2012-hz}, the LexTALE vocabulary tests can distinguish between lower intermediate (up to 59 percent), upper intermediate (60–80 percent) and advanced (above 80 percent) levels of proficiency based on average percent correct responses. Participants that score lower than %INSERT CUTOFF PERCENTAGE HERE% 
were excluded due to having too low of a proficiency score in one of the two languages. The Spanish–dominant group had a %INSERT CUTOFF PERCENTAGE HERE% 
success rate in Spanish and it was higher than their English success rate. The English-dominant group scored %INSERT CUTOFF PERCENTAGE HERE%
or higher in English and it was higher than their Spanish scores. The early bilingual group had a correctness score of %INSERT CUTOFF PERCENTAGE HERE% 
or more in both English and Spanish. %INSERT NUMBER OF PARTICIPANTS% 
participants were excluded from the analysis due to falling below the minimum standards used to describe each group. %INSERT NUMBER OF PARTICIPANTS%
participants were recruited from each bilingual population and are reported in the analysis section. 
\emph{During data collection, more participants than reported here were collected since it was expected that several participants would be removed for one or more of the reasons listed above. When more than %INSERT NUMBER OF PARTICIPANTS% 
participants remained eligible after removing participants who did not fit the population criteria, a random sampling of %INSERT NUMBER OF PARTICIPANTS% 
participants were selected from eligible pool of participants.}

%\textbf{DELETE ME Give accurate descriptions of participant groups, how they were classified, number of participants, etc.}

%-----------------------------------
%	SUBSECTION 2-2
%-----------------------------------
%participants lexTALE BLP 
% Computers, box in instrumentation
\subsection{Instrumentation}
The LexTALE and the LexTALE-Esp are tasks used to correlate vocabulary knowledge and language proficiency in English and Spanish respectively \citep{Izura2014-yw,Lemhofer2012-hz}. The LexTALE-Esp has also been shown to discriminate well between highly proficient Spanish speaking participants with different language dominances \citep{Ferre2017-jq}. The LexTALE and LexTALE-Esp tests will be used in order to group participants into the appropriate bilingual population—Spanish–dominant bilinguals of L2 English, English–dominant bilinguals of L2 Spanish or early bilinguals who were exposed to both English and Spanish before %INSERT AGE%. 
LexTALE is publicly available online and has been designed to run in PRAAT, Matlab and Presentations. For data collection purposes in the current study, participants completed both the LexTALE and LexTALE-Esp using PsychoPy. 

A second instrument used in this study was the Bilingual Language Profile (BLP), which is a survey based assessment tool for determining language dominance \citep{Birdsong2012-wd}. It assesses language history, use, proficiency and attitudes of participants in less than 10 minutes. This assessment tool has been used in numerous language studies with a focus on bilingualism and is available for free under the creative commons license. This tool allowed for the collection of information about participants demographics---name, age, sex, place of residence and educational background---which took place at experiment setup in the current study. It also allowed for participants to indicate their language history, use, proficiency and attitudes and was the last task complete during the current study. Since the BLP survey was completed after all experimental trials, the participants were able to choose whether they received Spanish or English instructions for the survey. The BLP is publicly available in a paper-based format or electronic format through the use of Google Forms. In an effort to make the experiment seamless as possible, the participants in the current study took the BLP within the PsychoPy platform as well. 

%Button box
The button box contained five different color buttons---left to right (white, green, blue, yellow and red)---which were referenced in all instruction 

%\textbf{DELETE ME Any special instrumentation could be included here (I am not sure that mine deserves a devoted section to instrumentation.}


%-----------------------------------
%	SUBSECTION 2-3
%-----------------------------------

\subsection{Design}
The syllabic intuitions were elicited from participants through a forced two-choice task and were given as much time as needed to respond. Participants were shown a real word in Spanish on the screen and then were presented with two options below that represented a CV or CVC structure. There were 24 critical word pairs that were used as the stimuli in a previous experiment. Each word pair shared the first three letters, but differed in initial syllable structure. Using the example word pair \emph{balada–baldosa} to illustrate, it can easily been seen that both words begin with the letters \emph{bal}. However, the initial syllable structure of \emph{balada} is \emph{ba}, a CV structure, while the initial syllable structure of \emph{baldosa} is \emph{bal}, a CVC structure, when following standard Spanish syllabification. All other word pairs followed the same pattern where one word had a CV word-initial syllable structure and the other word began with a CVC syllable---where C represents a consonant and V represents a vowel. The order of presentation for the two syllabic options, CV or CVC, remained the same for the entire experiment for the participant, but was counterbalanced across participants. In order to counterbalance for hand dominance and visual presentation in this task, half of the participants used the left response button to indicate a CVC response and the right response button to indicate a CV response while the other half of the participants did the opposite. Since the position of response buttons aligned with screen position, half of the participants visually saw the orthographic representation of the CVC syllable to the left of screen center and the CV syllable orthographic representation on the right while the other half of the participants saw the opposite visual displays for syllable orthographic representations.

%\textbf{DELETE ME Give an accurate idea of how the overall project was designed, what previous studies is it based off, theoretical principles, etc. DO NOT write about what participants do in this section.}


%-----------------------------------
%	SUBSECTION 2-4
%-----------------------------------

\subsection{Procedure}
Participants were seated in front of a laptop computer with a USB button box in order to complete the experiment in PsychoPy. At the beginning of the experiment, participants entered in demographic information as asked in the Basic Language Profile (BLP). Once they had entered the demographic information, all participants recruited from for this study completed the current experiment as an additional task of two additional ongoing experiments. Some participants completed this syllable intuition experiment following a visual word segmentation experiment while others completed it following a lexical decision experiment. Depending on the prior experiment completed, the overall procedural order of the time in the lab changed.

When the syllable intuition experiment was followed by the visual segmentation experiment, %INSERT NUMBER OF PARTICIPANTS%
participants began the experimental session with the Spanish vocabulary task---LexTALE-Esp. Immediately upon the completion of the LexTALE-Esp, participants were given instructions for the practice portion of the visual segmentation experiment. Following the practice portion, participants were given an instruction screen that indicated they had completed the practice portion, reminded about the controls needed for the experiment, allowed to ask any remaining question about the process. When the participant was ready to begin the actual experiment, they pressed the white button on the response box to begin. Once the participants had finished the entire visual segmentation experiment, they were presented with instructions for the practice trials of the syllable intuition experiment. Once they completed the practice trials, they were given a second instruction screen indicating they had completed the practice portion and were about to begin the real experiment. Once participants had completed the syllabic intuition experiment, they completed the LexTALE-Eng followed by the BLP. 

When the syllable intuition experiment was followed lexical decision based priming experiment, the order differed slightly. In this scenario, participants were immediately presented with instructions for practice trials for the lexical decision priming experiment. Once they had completed the practice trials, a new instruction screen appeared stating that they were getting ready to start the actual experimental trials. Then participants received instructions for the syllable intuition experiment of which was immediately followed the three remaining tasks---LexTALE-Esp, LexTALE-Eng and the BLP.

The participants were instructed that they would see Spanish words presented individually on the computer screen and that each word would be followed by two options for the first syllable of the presented word. There was no time constraint for the participant to select the option that they considered to be the first syllable of the presented word and the experiment would not progress until the participant had indicated a response. %In the second experiment (Lexical Access), this was initially ran with 4 second time limit. It affected participants 062, 065-067, 069-072, 074-076 so these 11 participants should be checked for missing values.% 
In total, there were 10 practice trials that followed the same criteria and procedure as the 48 experimental trials presented to the participants. %same 11 participants had timing issue in practice trials, check if possible but values may not have been stored.% 
For each trial, the word in lowercase letters appeared in the center of the screen for 1500 milliseconds before being replaced by the two syllable options---one a CV syllable and the other a CVC syllable---which remained on the screen, in all capital letters, until a response was entered. Following each response, the screen remained blank for 500 milliseconds before the next word appeared. %It may be good to create a visual representation of the experiment here to illustrate what each participant saw.%
For half of the participants, the CV syllable option was on the left side of the screen while the CVC syllable option was on the right. For the other half of the participants, the CVC syllable option was on the left side of the screen while the CV syllable option was on the right. Participants were given separate instruction screens for practice and experimental trials that differed only in letting the participant know whether they are about to enter a practice or experimental phase. %\emph{Is this too repetitive? Should this be described here or in the Design section, currently it is in both} In this task it is worth noting that timing was not being measured but following the training session, most participants were trying to indicate their answer before the 1.5 second display time of the word since the syllables were always on the same side of the screen.

%\textbf{DELETE ME WHEN FINISHED The first refer to mainly scope and size of research project while the procedure section is usually much more in-depth. Here, you connect your procedures to those in already published articles when possible and given detailed descriptions of classifications and scales used in procedure. Essentially ensure the reader does not suspect anything is being hidden and the researcher is honest. Do not repeat any unnecessary information in subsequent experiments of the paper.}

%----------------------------------------------------------------------------------------
%	SECTION 3
%----------------------------------------------------------------------------------------

\section{Results}

The practice trials for syllabification were not recorded and therefore are not represented in the reported data below. Participants syllabification patterns are first analyzed by group and then are compared across groups. 


%Discuss the deviation by group from the standard Spanish syllabification pattern. (Error rates)
%Discuss whether significant differences between groups are found. (anova between subjects)

%Are there patterns related to syllable structure CV or CVC.

%stats from Miquel: 
% 2 choice forced decision task (A or B) 
% logistic regression
% log Anova (Arcsin or logit)
Columns needed include:
\begin{enumerate}
\item{participant}
\item{correct syllable}
\item{correct answer}
\item{left key}
\item{right key}
\item{condition}
\item{response}
\item{correct response}
\item{response time}
\end{enumerate}

%\textbf{DELETE ME describe the analyses the researcher has done, but do not overload. Instead of creating a laundry list of statistics, create the story you want to tell using only the statistics that are related to addressing your problem. For each task, review the hypothesis, give the statistics and say what the result of the test means. Do not discuss the findings until you reach the discussion session. This is where tables and figures can help keep the paper looking clean and crisp instead of cluttered unorganized statistical test lists that are hard to follow. Figures show patterns while tables give details.}

%----------------------------------------------------------------------------------------
%	SECTION 4
%----------------------------------------------------------------------------------------

\section{Discussion}

%\textbf{split into several small sections following each individual experiment’s results sections where applicable. The General Discussion steps back and begins with an overview of the problem and then the findings. A general rule of thumb is to keep this section shorter than the introduction. Only give limitation directly related to the current study, not the general limitation of the research or the field as a whole and be sure to give a good reason for why these limitations are not as bad as they sound on the surface.}


%----------------------------------------------------------------------------------------
%	SECTION 5
%----------------------------------------------------------------------------------------

\section{References}

%Insert references for this chapter here









%% Chapter Template

\chapter{Introduction} % Main chapter title

\label{Chapter1} % Change X to a consecutive number; for referencing this chapter elsewhere, use \ref{Chapter2}

%----------------------------------------------------------------------------------------
%	SECTION 0
%----------------------------------------------------------------------------------------

\section{Abstract}

Give abstract of entire dissertation here

Keywords: (list all words necessary)

%----------------------------------------------------------------------------------------
%	SECTION 1
%----------------------------------------------------------------------------------------

\section{Introduction}

This is a proposal for the three article dissertation that will be written in order to complete the degree requirements of the Second Language Acquisition and Teaching program (SLAT) at the University of Arizona. The main focus of this dissertation is on early and late Spanish–English bilingual populations and their ability to use the syllable in order to process spoken Spanish. The three articles that make up this dissertation will all focus on the same three speaker populations and will be written as three independent articles.

%-----------------------------------
%	SUBSECTION 1-1
%-----------------------------------
\subsection{Background}

A syllable is a pronounceable linguistic unit of a given language that generally contains a highly sonorant sound as it nucleus—typically a vowel in most languages. The syllable has been the focus of many speech processing studies in the past, but many avenues of research have yet to be investigated. Originally, the syllable made its highlight as spoken language processing unit when it was found that participants could detect syllables faster than they could detect individual phonemes (Savin \& Bever, 1970). With the findings of several other studies, it became clear that speech processing was a complicated task that was not likely to be answered through finding a minimal perceptual unit. Specifically, one linguistic unit—phoneme, syllable, word, etc.— could not be the sole mechanism in which listeners of language break the speech stream into smaller or processable chunks (Foss \& Swinney, 1973; Healy \& Cutting, 1976; McNeill \& Lindig, 1973). Mehler and Hayes (1981) captured the need for a change in the direction of research regarding the syllable, “Traditionally, psycholinguistics research has invested the bulk of its efforts into uncovering the units used in speech processing. Although it is currently fashionable to claim that such work is pointless since it has no very clear outcome, many of the more meaningful advances in the field have come from projects whose framework included the problem of processing units.” They go on to delineate two different levels in which the research around the syllable could move forward: (1) The syllable as a phonological unit of the language which can efficiently explain the grammar of language and (2) The syllable as a unit which aides speech perception and comprehension. This dissertation will be comprised of three articles utilizing psycholinguistic methodologies that fall into the second level of investigation, which also builds off the researchers who began refocusing their investigations in this vein of syllable research in the early 1980s. Even with this new angle, the results of previous research began to reveal that even focusing on the syllable’s use in speech processing may be too large of a question. The research suggests that there are two distinct subprocesses of speech processing in which the syllable may play a role—speech segmentation and lexical access. The first article of this dissertation will focus on speech segmentation while the second and third articles will focus on lexical access.

%-----------------------------------
%	SUBSECTION 1-2
%-----------------------------------

\subsection{Present Study}
The three proposed articles for this dissertation will add information about syllable structure and the representation of the syllable in Spanish to the knowledge base of the field. Within the speech segmentation strain of research, several gaps still remain. The first being the bilingual degrees of the speaker populations being studied. Up to this point in time, the majority of the investigations have been conducted with monolingual speakers. Those that have investigated the role of the syllable in speech processing of bilinguals have generally compared the bilinguals against the monolingual speakers of the two languages. This dissertation will not utilize monolingual Spanish speaker controls, but instead will utilize three distinct bilingual populations of Spanish and English—late learners of L2 English, late learners of L2 Spanish and balanced bilinguals (early learners of both Spanish and English)—which will be compared against the other bilingual populations. This type of setup will allow for the control group to be the Spanish native L2 English speaker group and two test groups be early and late learners of Spanish. Many previous lexical access studies have generally been conducted using the visual word recognition paradigm to study speech processing in conjunction with the syllable while only Italian has really been investigated in this realm utilizing spoken word recognition techniques. Therefore, this dissertation will add to the knowledge base of the syllable’s role through replication studies using a different population of speakers in the visual word recognition while an expansion to other language groups will be the outcome for the spoken word recognition study. 
The overarching questions that underlie this dissertation project are: (1) Whether or not the common linguistic unit—the syllable—is available to Spanish–English bilinguals when processing spoken Spanish?, (2) Whether or not the syllable is a strategy used by Spanish–English bilinguals in speech segmentation and/or lexical access? and (3) Does the age of acquisition, early versus late bilinguals, affect the ability and efficiency in which Spanish–English bilinguals can make use of the syllable? The format for the remaining chapters of this dissertation project will be as follows in order to address the main overarching questions: chapter two will include two experiments that explore the representation of the syllable in the minds of Spanish–English bilinguals using a syllabic intuition two option forced-choice task and a segmentation task using an auditory monitoring technique. Chapter three will include a visual masked priming experiment with a lexical decision task that explores the syllable’s role in lexical access by Spanish–English bilinguals. It will again incorporate a syllabic intuition task to add more additional knowledge of syllabification differences across native speakers, early learners and late learners of Spanish. Chapter four will include a cross-modal fragment semantic priming experiment with a lexical decision task to investigate for further evidence of the use of the syllable to gain access to the mental lexicon of Spanish–English bilinguals. As in the previous two chapters, chapter 4 will also include a syllabic intuition task. Chapter five will then conclude the dissertation project by drawing overall conclusions and how all three individual studies were necessary to draw the conclusions that were borne out through the various methodologies used to explore the role of the syllable in Spanish speech processing in the three articles of the dissertation.


%----------------------------------------------------------------------------------------
%	SECTION 2
%----------------------------------------------------------------------------------------

\section{Methods}

Not needed

%-----------------------------------
%	SUBSECTION 2-1
%-----------------------------------
\subsection{Participants}


Probably still important here to some degree

Give accurate descriptions of participant groups, how they were classified, number of participants, etc. 


%----------------------------------------------------------------------------------------
%	SECTION 3
%----------------------------------------------------------------------------------------

\section{Results}

Is this needed?


%----------------------------------------------------------------------------------------
%	SECTION 4
%----------------------------------------------------------------------------------------

\section{Discussion}
Is this needed?


%----------------------------------------------------------------------------------------
%	SECTION 5
%----------------------------------------------------------------------------------------

\section{References}

Insert references for this chapter here.










%% Chapter Template

\chapter{Syllabification} % Main chapter title

\label{Chapter2} % Change X to a consecutive number; for referencing this chapter elsewhere, use \ref{Chapter2}

%----------------------------------------------------------------------------------------
%	SECTION 0
%----------------------------------------------------------------------------------------

\section{Abstract}

Give Syllabification abstract here

Keywords: (list all words necessary)

%----------------------------------------------------------------------------------------
%	SECTION 1
%----------------------------------------------------------------------------------------

\section{Introduction}

Talk about syllabification
differences in syllabification between spanish and english

The syllable intuition experiment is an important step in the process because how the three groups of participants syllabify the words will have a direct effect on how fast or useful the syllable is in their segmentation strategy. It is expected that syllabic intuitions will vary based on being a native speaker of Spanish, an early learner of Spanish or a late learner of Spanish. These differences are likely to stem from age of acquisition, language dominance or type of schooling. This experiment will begin to build a data source of Spanish–English bilinguals syllabification of Spanish words that will help to determine whether or not age of acquisition or language dominance are sources are different syllabification patterns. 

\textbf{DELETE ME Write my introduction to the syllabification article here (Will write the introduction LAST)
The section of your article most likely to be read, not skimmed or skipped. The first paragraph or two is the overview of the article: describe the problem, question or theory motivating the research.} 

%-----------------------------------
%	SUBSECTION 1-1
%-----------------------------------
\subsection{Background}

In the second part, describe relevant theories, review past research and give more details on the current research question. Do not forget about signposting, which headings and subheadings can naturally create for the reader.

%-----------------------------------
%	SUBSECTION 1-2
%-----------------------------------

\subsection{Present Study}


\textbf{DELETE ME The third section titled, “The Present Experiment” or “The Present Research”, follows and contains experimental descriptions and how they address the questions being asked.
For most articles, keep your introduction under 10 pages}

%----------------------------------------------------------------------------------------
%	SECTION 2
%----------------------------------------------------------------------------------------

\section{Methods}

Write this section FIRST
is the section that describes how the research was conducted. A good one shows how well thought out the experiment design is and allows other researchers to easily replicate it. This section also follows a formula:

%-----------------------------------
%	SUBSECTION 2-1
%-----------------------------------
\subsection{Participants}

The participants were split into three distinct populations of Spanish–English bilinguals. The Spanish–dominant group consisted of native speakers of Spanish that lived in Sonora, Mexico and learned English after \[INSERT AGE HERE\]. The English–dominant group were native English speakers that lived in Tucson, Arizona and learned Spanish after \[INSERT AGE HERE\]. The early bilingual group consisted of participants who lived in Tucson, Arizona, but were exposed to both English and Spanish before the \[INSERT AGE HERE\].
\emph{The participants will be placed into their appropriate group based off of the combined results of Bilingual Language Profile (BLP), the LexTALE and the LexTALE-Esp vocabulary tests.} 

Based off of responses in the BLP, participants who report proficiency in a language other than English and Spanish will be removed from the data analysis. According to the findings of Lemhöfer \& Broersma (2012), the LexTALE vocabulary tests can distinguish between lower intermediate (up to 59 percent), upper intermediate (60–80 percent) and advanced (above 80 percent) levels of proficiency based on average percent correct responses. Participants that score lower than \[INSERT CUTOFF PERCENTAGE HERE\] were excluded due to having too low of a proficiency score in one of the two languages. The Spanish–dominant group had a  \[INSERT PERCENTAGE HERE\] success rate in Spanish and it was higher than their English success rate. The English-dominant group scored \[INSERT PERCENTAGE HERE\] or higher in English and it was higher than their Spanish scores. The early bilingual group had a correctness score of \[INSERT PERCENTAGE HERE\] or more in both English and Spanish. \[INSERT \# OF PARTICIPANTS\] participants were excluded from the analysis due to falling below the minimum standards used to describe each group. \[INSERT \# OF PARTICIPANTS\]  participants were recruited from each bilingual population and are reported in the analysis section. 
\emph{During data collection, more participants than reported here were collected since it was expected that several participants would be removed for one or more of the reasons listed above. When more than \[INSERT \# OF PARTICIPANTS\] participants remained eligible after removing participants who did not fit the population criteria, a random sampling of \[INSERT \# OF PARTICIPANTS\] participants were selected from eligible pool of participants.}

\textbf{DELETE ME Give accurate descriptions of participant groups, how they were classified, number of participants, etc.}

%-----------------------------------
%	SUBSECTION 2-2
%-----------------------------------

\subsection{Design}
The same 24 critical word pairs used in experiment 1 will be used in experiment 2. \textbf{this is has seen a major overall since writing and now treated separately}

Following the syllable monitoring in experiment 1, the same participants will immediately complete the syllabic intuition experiment. It is important to note that no participant will see a repeat of any critical word previously seen in experiment 1 to avoid any effects of repetition priming. In order to accomplish this, the participants will see the other word of the critical word pairs that they saw in experiment 1. For the example word pair “balada–baldosa”, if the participant saw “balada” in experiment 1, they will see “baldosa” in the experiment 2 or vice versa. In order to counterbalance for hand dominance and visual presentation in this task, half of the participants will use the left response button to indicate a CVC response and the right response button to indicate a CV response while half of the participants will do the opposite. Half of the participants will visually see the CVC syllable orthographic representation on the left side of the critical item and the CV syllable orthographic representation on the right side of the critical item while the other half of the participants will see the opposite visual displays for syllable orthographic representations.

\textbf{DELETE ME Give an accurate idea of how the overall project was designed, what previous studies is it based off, theoretical principles, etc.
DO NOT write about what participants do in this section.}

%-----------------------------------
%	SUBSECTION 2-3
%-----------------------------------
\subsection{Instrumentation}
The LexTALE and the LexTALE-Esp are tasks used to correlate vocabulary knowledge and language proficiency in English and Spanish respectively (Izura, Cuetos, \& Brysbaert, 2014; Lemhöfer \& Broersma, 2012). The LexTALE-Esp has also been shown to discriminate well between highly proficient Spanish speaking participants with different language dominances (Ferré \& Brysbaert, 2017). The LexTALE and LexTALE-Esp tests will be used in order to group participants into the appropriate bilingual population—Spanish–dominant bilinguals of English (L2), English–dominant bilinguals of Spanish (L2) or Balanced bilinguals (exposed to both English and Spanish since early childhood). Again, the LexTALE is available online and has been designed to run in PRAAT, Matlab and Presentations, but the participants will take it using PsychoPy at the experiment testing location immediately following the experimental tasks. 
A second instrument used in this study is the Bilingual Language Profile (BLP), which is a survey based assessment tool for determining language dominance (Birdsong, Gertken, \& Amengual, 2012). It assesses language history, use, proficiency and attitudes of participants in less than 10 minutes. This assessment tool is commonly used in language studies with a focus on bilingualism and is available for free under the creative commons license. This tool allows researchers to collect information about participants demographics—name, age, sex, place of residence and educational background. It also allows for the participant to talk about their language history, use, proficiency and attitudes. For the current dissertation project, this survey will be administered while the participants are in the experiment testing location through Google Forms as designed by the creators of the BLP. Since no modification will be made to the BLP survey, the scores will be automatically calculated as designed by the creators. In all cases, the BLP survey will be completed after the experimental trials to avoid any confounding factors of language activation, Spanish or English, the participants will be able to choose whether they would like to take the survey with Spanish or English instructions.


\textbf{DELETE ME Any special instrumentation could be included here (I am not sure that mine deserves a devoted section to instrumentation.}


%-----------------------------------
%	SUBSECTION 2-4
%-----------------------------------

\subsection{Procedure}

The participants were instructed that they would see Spanish words presented individually on the computer screen and that each word would be followed by two options for the first syllable of the presented word. There was no time constraint for the participant to select the option that they considered to be the first syllable of the presented word and the experiment would not progress until the participant had indicated a response. In total, there were 10 practice trials that followed the same criteria and procedure as the 48 experimental trials presented to the participants. For each trial, the word in lowercase letters appeared in the center of the screen for 1500 milliseconds before being replaced by two syllable options---one a CV syllable and the other a CVC syllable---which remained on the screen, in all capital letters, until a response was entered. Following each response, the screen remained blank for 500 milliseconds before the next word appeared. For half of the participants, the CV syllable option was on the left side of the screen while the CVC syllable option was on the right. For the other half of the participants, the CVC syllable option was on the left side of the screen while the CV syllable option was on the right.


\textbf{combine this with first paragraph}
Participants will be instructed that a word will appear in the center of the screen and to indicate their response as quickly and accurately as possible. Participants will press one button on the response box if the first syllable of the word is a CVC structure (Consonant-Vowel-Consonant) and will press the other button on the response box if the first syllable of the word is a CV structure (Consonant-Vowel). The orthographic representation of each syllable structure type will also be printed on the screen to the left and the right of the critical word to ensure participants are not confused by the CV/CVC terminology. Participants will be given 5 practice trials followed by 24 experimental trials that each had their own set of directions differing only in letting the participant know whether they are about to enter a practice or experimental phase.

\textbf{DELETE ME WHEN FINISHED
The first refer to mainly scope and size of research project while the procedure section is usually much more in-depth. Here, you connect your procedures to those in already published articles when possible and given detailed descriptions of classifications and scales used in procedure. Essentially ensure the reader does not suspect anything is being hidden and the researcher is honest. Do not repeat any unnecessary information in subsequent experiments of the paper.}


%----------------------------------------------------------------------------------------
%	SECTION 3
%----------------------------------------------------------------------------------------

\section{Results}

The practice trials will be removed from the data collected and so that only the critical trials remain. Participants syllabification patterns will be analyzed by group and then compared across groups. One valuable piece of data that will be analyzed here is the success rate at determining the correct initial syllable of the Spanish words presented on screen. However, the types of error patterns is also valuable because a preference for CVC or CV type syllabification may emerge. 

\textbf{DELETE ME describe the analyses the researcher has done, but do not overload. Instead of creating a laundry list of statistics, create the story you want to tell using only the statistics that are related to addressing your problem. 
For each task, review the hypothesis, give the statistics and say what the result of the test means. 
Do not discuss the findings until you reach the discussion session. 
This is where tables and figures can help keep the paper looking clean and crisp instead of cluttered unorganized statistical test lists that are hard to follow. 
Figures show patterns while tables give details.}


%----------------------------------------------------------------------------------------
%	SECTION 4
%----------------------------------------------------------------------------------------

\section{Discussion}

split into several small sections following each individual experiment’s results sections where applicable. 
The General Discussion steps back and begins with an overview of the problem and then the findings. A general rule of thumb is to keep this section shorter than the introduction. Only give limitation directly related to the current study, not the general limitation of the research or the field as a whole and be sure to give a good reason for why these limitations are not as bad as they sound on the surface.


%----------------------------------------------------------------------------------------
%	SECTION 5
%----------------------------------------------------------------------------------------

\section{References}

Insert references for this chapter here.









 
%% Chapter Template

\chapter{Visual Word Segementation} % Main chapter title

\label{Chapter3} % Change X to a consecutive number; for referencing this chapter elsewhere, use \ref{Chapter3}
%----------------------------------------------------------------------------------------
%	SECTION 0
%----------------------------------------------------------------------------------------

\section{Abstract}

Give Segmentation abstract here

Keywords: (list all words necessary)

%----------------------------------------------------------------------------------------
%	SECTION 1
%----------------------------------------------------------------------------------------

\section{Introduction}

One of the processes necessary to process spoken speech is the ability to break up the continuous stream of spoken language into manageable chunks of information—i.e. segmentation. In order to segment speech, attaching meaning to the segmented chunks of sounds is not a prerequisite as it is in the retrieval of lexical items. This means a speaker can segment speech by isolating or detecting certain language sounds from a spoken language in which they have no knowledge—a common practice in psycholinguistic research to ensure prelexical decisions are being made (A. Cutler, Mehler, Norris, \& Segui, 1986). Understanding that people can successfully segment speech of an unknown language, having naïve and native listeners of a language search spoken speech for a particular sequence of sounds tells researchers several things. First, it allows for comparisons of different languages and testing hypotheses about speech segmentation strategies depending on experimental design. Furthermore, it can give additional support the effects that may or may not be found are not due to experimental design or items.
It is important to note that this dissertation builds off of nearly three decades of research that also fell into the second level of investigation of the syllable discussed by Mehler and Hayes (1981)—a unit which aides speech perception and comprehension. These researchers abandoned the search for the minimal perceptual unit and turned their focus towards the syllable’s role in speech perception despite the fact it was not likely to be a minimal perceptual unit for speech processing. While the monitoring paradigm was not going to help researchers discover the minimal perceptual unit used to process spoken language, this methodology was able to be used for testing the syllable’s role in speech processing. The syllable monitoring paradigm presents a target syllable such as “pa, pal, ba, bal, ca or car”. Then a participant must then identify the target syllable that is embedded in speech, which is presented as a list of words, carrier items, rather than connected and continuous speech. Using French subjects in a French syllable monitoring experiment, participants were asked to find CV or CVC syllables in lists of bisyllabic words, which had initial syllable structures of CV or CVC where both the target and carrier items were presented auditorily (Jacques Mehler, Dommergues, Frauenfelder, \& Segui, 1981). For example, an experimental trial started with the target presentation “The target is “pa”, which was immediately followed by the list of words, ranging from two to five words, which may or may not contain a critical word as the last item of the list. Only one out of every ten sequences of words contained a critical carrier item. The critical trials contained one carrier item of a pair (e.g. PA.LA.CE–PAL.MIER or CAR.TON–CA.ROTTE) that appeared as the last item of the word sequence, where both words always shared the initial three phonemes, but they differed in syllable structure. The experimental conditions varied on the target syllable and initial syllable of the carrier item could match could match (find “pa” in PA.LACE, “pal” in PAL.MIER, “ca” in CA.ROTTE or “car” in CAR.TON) or mismatch (find “pal” in PA.LACE, “pa” in PAL.MIER, “ca” in CAR.TON or “car” in CA.ROTTE).  In a French syllable monitoring experiment, participants were asked to press a response key as soon as they have heard their target sound fragment in the speech they were hearing, French speakers found the target syllable faster when the initial syllable structure of the carrier item matched the target. The real French nouns used as critical items showed a faster detection time of the target pa when PALACE rather than PALMIER was the critical item. Likewise, the target pal was found faster when the critical item was  PALMIER rather than PALACE (Jacques Mehler et al., 1981). This “crossover effect” found by Mehler, Dommergues, Frauenfelder and Segui (1981)—faster detection times when target syllable structure matched initial syllable of carrier word—was not found by native English speakers who monitored real English words where one word in each pairing had an unclear syllable boundary—an ambisyllabic \[l\] (A. Cutler et al., 1986). In other words, English speakers did not show differences in the speed in which they identified the target syllable according to the syllable structure of the carrier item. Finding “pa” in PA.LACE was not easier than finding “pal” in PA.LACE nor was finding “pal” in PAL.PI.TATE easier than finding “pa” in PAL.PI.TATE. The follow up to this finding was to have native English speakers monitor the French noun used by Mehler et al. (1981) where syllable boundary ambiguity was not present in the carrier items. Given that these were naive listeners of French, the segmentation process is being examined here because these listeners will have no lexical entries for any of the French words they hear. The English listeners again did not show the crossover effect suggesting that syllable based segmentation is not a strategy used in speech segmentation by English monolinguals even when it would be beneficial to do so (A. Cutler et al., 1986). Importantly, this finding suggests that segmentation strategies are not universal, but are language specific. In order to give additional strength to their argument, the researchers used the English experimental items that did not produce the crossover effect with native English speakers, and ran the same experiment with native speakers of French who had not learned English. The French listeners still showed the crossover effect even when listening to an unknown language—in this case, English. 
Given the nature of the differences found between English and French monolingual speakers’ strategy for speech segmentation, the next logical step while remaining in the same of vein of research was to investigate the segmentation strategy used by bilinguals. In a series of experiments, both English and French language modes of French–English balanced bilingual speakers were tested (A. Cutler, Mehler, Norris, \& Segui, 1992). The French mode of the experiment used the same stimuli as Mehler et al. (1981) which had showed a strong syllabic effect with French monolinguals, but showed no such effect with the bilingual population. Post-hoc analysis split the bilinguals into two groups according to language dominance as reported by the participant in response to the following question, “Suppose you developed a serious disease, and your life could only be saved by a brain operation which would unfortunately have the side effect of removing one of your languages. Which language would you choose to keep?” This manner of determining language dominance was then used in all remaining experiments of the study. Under this post-hoc analysis where language dominance was considered for the experiments conducted in French, French-dominant listeners patterned like French monolinguals. Likewise, English dominant listeners patterned like English monolinguals from previous studies for the experiments conducted in French. In the French versions of the experiments, French-dominant participants exhibited the syllable-based segmentation strategy while the English-dominant participants did not show evidence of a syllable-based segmentation.  In the English versions of the experiments, French dominant speakers patterned unlike the French monolinguals while the English dominant speakers continued to pattern like the English monolinguals. Like previous studies with English monolinguals and the English-dominant bilinguals of the Cutler et al. (1986), the French-dominant bilingual were not able to find “pa” in PALACE easier than “pal” in PALACE nor were they able to find “pal” in PALPITATE faster than “pa” in PALPITATE. Where French was considered the dominant language of the speaker, it appeared that they had identified the ineffectiveness of the syllable based segmentation strategy and were able to inhibit its application while listening to English. When English was considered the dominant language of the speaker, it appeared that these speakers could not utilize a syllable based segmentation strategy despite the fact that one of their two languages could use a syllabic segmentation strategy.  This suggested that bilinguals—even in their most balanced form—are not two monolinguals within a bilingual mind.
English and French are quite different in their phonological structure: (1) French is syllable-timed while English is stressed-timed, (2) French has fixed stress while English has variable stress, (3) French has no vowel reduction while English has rampant vowel reduction and (4) French has clear syllable boundaries while English has ambiguous syllable boundaries. As a result of the phonological structures of French and English, in the previous experiments stressed syllables were always the carrier syllables for English speakers while unstressed syllables were always the carrier syllables for the French speakers. This stress factor was considered as a potential confound in that stressed syllables are known to carry more phonetic details and last longer than unstressed syllables. To overcome this difference, a different group of bilingual speakers were recruited—Catalan–Spanish bilinguals—because stress is variable in both languages and can therefore be controlled (Sebastián-Gallés, Dupoux, Seguí, \& Mehler, 1992). Unlike French and English that varied on multiple factors, Catalan and Spanish only differ in vowel reduction—Catalan has vowel reduction while Spanish does not. With stress being controlled across both languages, vowel reduction can be isolated as it is allowed in Catalan, like in English, but it is not in French or Spanish. They found that Catalan dominant speakers monitoring Catalan speech produced the crossover effect only when the initial syllable of the carrier item was unstressed. Spanish was also considered in the same study where the researchers found no crossover effect by Spanish dominant speakers monitoring in Spanish. Given the similarities between Spanish and French, both have clear and unambiguous syllable boundaries nor vowel reduction, it is surprising that they did not find a crossover effect with stressed or unstressed initial syllables of the carrier items as they did with Catalan speakers. Sebastián-Gallés et al. (1992) attempted to force a post-lexical decision by the Spanish dominant speakers with the incorporation of an additional semantic relatedness task. This succeeded in slowing the reaction time by an average of 250 milliseconds and found a syllabic effect in both initially stressed and unstressed Spanish words. The syllabic effect similar to the those found in Sebastián-Gallès et al. (1992) was later replicated in Italian by Tabossi et al. (2000).
In contrast to the findings of Sebastián-Gallés et al. (1992), Bradley, Sánchez-Casas and García-Albea (1993) found a crossover effect when Spanish speakers monitored Spanish carrier items. They also found no crossover effect for English speakers monitoring English carrier items. Similar to Cutler et al. (1986), Bradley et al. (1993) tested naive listeners with the same material. They found no crossover effect for English monolinguals monitoring Spanish carrier items, which was comparable to the French findings. However, unlike the French findings, the Spanish monolinguals monitoring in English also showed no syllabic segmentation effect. Bradley et al. (1993) then turned to Spanish–English bilinguals where they again found no crossover effects when monitoring Spanish carrier items. This suggests that these native speakers of Spanish and English L2 speakers have abandoned their native segmentation strategy even when listening to one of their native languages (Spanish). This result again differs from the Cutler et al. (1986) French–English bilinguals because the French kept the native syllable-based segmentation strategy when listening to French and abandoned it only when listening to English where it was no longer effective.


\textbf{DELETE ME Write my introduction to the syllabification article here (Will write the introduction LAST)
The section of your article most likely to be read, not skimmed or skipped. The first paragraph or two is the overview of the article: describe the problem, question or theory motivating the research.} 

%-----------------------------------
%	SUBSECTION 1-1
%-----------------------------------
\subsection{Background}

In the second part, describe relevant theories, review past research and give more details on the current research question. Do not forget about signposting, which headings and subheadings can naturally create for the reader.

%-----------------------------------
%	SUBSECTION 1-2
%-----------------------------------

\subsection{Present Study}
The first article of this dissertation will ground itself here in the previous literature by first attempting to replicate the Spanish findings of Bradley et al. (1993). with the Spanish-dominant bilingual group, which will serve as the control, and the English-dominant bilingual group, which will serve as the control floor. The third bilingual group in the experiment will provide new knowledge on the syllable’s role in speech processing.  The first group, henceforth the Spanish-dominant group, will consist of participants who grew up in Mexico speaking only Spanish and later learned their L2 English after the age of twelve. The participants in this group will be recruited from Guanajuato, Mexico and will have received their formal grade-level schooling in Spanish. The Spanish-dominant group will serve as the control in this experiment. The second group, the English-dominant group or late learners of Spanish, will consist of participants who grew up in Arizona speaking only English and later learned their L2 Spanish after the age of twelve. The participants for this group will be recruited from Tucson, Arizona and will have received their formal grade-level schooling in English. This group will serve as the first of two experimental groups. The last group, the Balanced-bilingual group or early learners of Spanish, will consist of participants that were exposed to both Spanish and English from infancy. These participants will have acquired both Spanish and English simultaneously, but unlike the Spanish-dominant group, they will have received their formal grade-level schooling in English. The Balanced-bilingual group participants will also be recruited from Tucson, Arizona and serve as the second experimental group. Utilizing these three groups, it will be possible isolate differences in segmentation strategies that are due to the age of acquisition. The experiment will be run completely in Spanish mode and the expectation is that the Spanish-dominant group will employ a syllabic based segmentation strategy while the English-dominant group will not. The balanced bilingual group could go in one of two ways. Since the experiment is being conducted in Spanish—a language that encourages the use of a syllabic segmentation strategy—the first possible outcome is that the early learners of Spanish will exhibit the same pattern as the Spanish-dominant group. If this is the outcome of the segmentation experiment, then it would provide stronger evidence for a syllable-based segmentation strategy for speakers of Spanish. The second option is that the early learners will pattern like the English-dominant counterparts and fail to employ the syllable-based segmentation despite the fact that the language input supports it. This would contradict the findings of monolingual and Spanish-dominant bilingual speakers of English, who appear to use this syllable-based strategy. If these were to be the results of the first experiment, it would give additional support to the findings of Bradley et al. (1993) and there would be several possible factors that would warrant further investigation. Since the population of balanced bilinguals will be recruited from Tucson, Arizona and have completed their schooling in English, it would be a worthwhile endeavor to find a comparable balanced bilingual population who received their schooling in Spanish. This would allow for a comparison of the effect of schooling and explicit teaching of syllables, which typically occurs when children are taught to read and its relation to speech segmentation strategies. 
Another avenue to investigate would be the syllabic intuitions of participants, which could be a factor given that English and Spanish differ in ways similar to English and French participants of previous research studies. It may be that the three bilingual populations do not agree on the syllabic structures of the speech they are segmenting as a result of language background profiles. It will be possible to look at syllabic intuitions from the data that will be collected in a syllabic intuition task conducted in the second experiment of the first article in the proposed dissertation project. 
The main purpose of this study is to determine whether a representation of the syllable is a represented linguistic unit in the minds of bilingual Spanish speakers. This chapter will address the following questions:
1.	Does a representation of the syllable exist in minds of Spanish–English bilinguals which is available to aide in pre- or post-lexical levels of segmenting spoken Spanish?
2.	Does the age of acquisition of Spanish in Spanish–English bilinguals determine whether or not syllabic intuitions of Spanish match the intuitions of native Spanish speakers?
This first question on the representation of the syllable has been one with unclear results in previous research. As a precaution to building this entire dissertation on less-than-stable previous findings, a quasi-pilot study was completed in the summer of 2018. One PsychoPy experiment that included two separate tasks—an identification task and a syllabification task—was designed and conducted completely in Spanish. Eight (8) native speakers of Spanish who attended school in a Spanish speaking country and learned English as adults were recruited from the University of Arizona main campus in Tucson, AZ. All participants were highly proficient in their English (L2) as they were currently enrolled in a masters or doctorate program or had just recently completed their graduate degree at the University of Arizona. This group showed a consistent syllabification pattern and showed a significant interaction between the visually presented targets and carrier items in the identification task. Participants responded faster when the target syllable structure matched the initial syllable structure of carrier item than when they did not coincide.
Finding this effect in the population of Spanish dominant bilingual speakers of English, which was representative of one of the three populations being investigated, gave the confidence needed to proceed with the dissertation project. Therefore, the second chapter of the dissertation will use PsychoPy to conduct two experiments—an identification task using a syllable monitoring paradigm and a syllabic intuition task using a two option forced-choice methodology. Three additional instruments, Bilingual Language Profile (BLP), the LexTALE English vocabulary test and the LexTALE-Esp Spanish vocabulary test, will also be used to collect demographic information and language proficiency data that will be used for placement of participants into the appropriate bilingual group or to exclude them altogether from the data analysis. The nature of these tasks is discussed in the instruments section of the experiment 1. However, it is important to note that both of these tasks will take place following the completion of experiments 1 and 2, syllable monitoring and syllabic intuition experiments, in an effort to avoid biasing participants on the nature of the investigations in which they are participating.


\textbf{DELETE ME The third section titled, “The Present Experiment” or “The Present Research”, follows and contains experimental descriptions and how they address the questions being asked.
For most articles, keep your introduction under 10 pages}

%----------------------------------------------------------------------------------------
%	SECTION 2
%----------------------------------------------------------------------------------------

\section{Methods}

Write this section FIRST
is the section that describes how the research was conducted. A good one shows how well thought out the experiment design is and allows other researchers to easily replicate it. This section also follows a formula:

%-----------------------------------
%	SUBSECTION 2-1
%-----------------------------------
\subsection{Participants}

The participants will make up three distinct populations of Spanish–English bilinguals, which will remain the same throughout all the experiments carried out in the following three chapters. The participants will be placed into their appropriate group based off of the combined results of Bilingual Language Profile (BLP), the LexTALE and the LexTALE-Esp vocabulary tests. Based off of responses in the BLP, participants who report proficiency in a language other than English and Spanish will be removed from the data analysis. According to the findings of Lemhöfer \& Broersma (2012), the LexTALE vocabulary tests can distinguish between lower intermediate (up to 59 percent), upper intermediate (60–80 percent) and advanced (above 80 percent) levels of proficiency based on average percent correct responses. Participants that score lower than 50 percent will be excluded due to having too low of a proficiency score in one of the two languages. The Spanish–dominant group will have a 80 percent success rate in Spanish and it will be higher than their English success rate. The English-dominant group will score a 80 percent or higher in English and it will be higher than their Spanish scores. The balanced bilingual group will have a correctness score of 80 percent or more in both English and Spanish. Due to the likely fact that several participants will necessarily be excluded from the analysis, 30 participants will be recruited for each population in each chapter. If more than twenty participants remain eligible after removing participants who do not fit the population criteria, a random sampling of 20 participants will be selected from eligible pool of participants.

\textbf{DELETE ME Give accurate descriptions of participant groups, how they were classified, number of participants, etc.}

%-----------------------------------
%	SUBSECTION 2-2
%-----------------------------------

\subsection{Design}

There will be 24 word pairs selected as critical items where the initial syllable structure varies between a CV and a CVC structure while the first three phonemes are shared between the two—i.e. ba.la.da–bal-do-sa, cu.le.bra–cul.pa.ble, mo.re.ra–mor.ci.llo, jo.ro.ba–jor.na.da, etc. In addition to the 24 critical word pairs, another 294 real Spanish words will be selected to use as fillers and are also balanced according to initial syllable structure—147 start with a CV syllable and 147 start with a CVC syllable. Both critical items and fillers are all trisyllabic nouns where stress fell on the penultimate syllable. Target fragments representing a CV or CVC syllable—“ba” or “bal” respectively—will also be recorded separately. All critical items and fillers will be recorded by a single Spanish native speaker in a quiet sound booth in the Arizona Applied Phonetics Laboratory at the University of Arizona. 

There will be four different versions of the experiment for counter-balancing purposes, which is accomplished by balancing across participants. For the example word-pair “balada”–”baldosa”, condition 1 would search for “ba” in “balada”, condition 2 would search for “bal” in “balada”, condition 3 would search for “ba” in “baldosa” and condition 4 would search for “bal” in “baldosa”. Each participant will be randomly assigned to one of the four versions of the experiment, which means that no participant will see both critical words from the critical word pairs more than once during the experiment. For example, if participant 1 is assigned to a version of the experiment where condition 1 (find “ba” in “balada”) is presented to them for the critical word pair “balada–baldosa” then participant 1 would not see conditions 2, 3 or 4 in their experiment. Each version of the experiment presents 12 CV and 12 CVC critical trials where half of each type of syllable structure contains a match between the syllable structure of the target and critical item while the other half are mismatched. Each block presented to the participant contains 1 critical trial and 11 filler trials which are also balanced for CV and CVC syllable structures.

\textbf{DELETE ME Give an accurate idea of how the overall project was designed, what previous studies is it based off, theoretical principles, etc.
DO NOT write about what participants do in this section.}

%-----------------------------------
%	SUBSECTION 2-3
%-----------------------------------
\subsection{Instrumentation}

The LexTALE and the LexTALE-Esp are tasks used to correlate vocabulary knowledge and language proficiency in English and Spanish respectively (Izura, Cuetos, \& Brysbaert, 2014; Lemhöfer \& Broersma, 2012). The LexTALE-Esp has also been shown to discriminate well between highly proficient Spanish speaking participants with different language dominances (Ferré \& Brysbaert, 2017). The LexTALE and LexTALE-Esp tests will be used in order to group participants into the appropriate bilingual population—Spanish–dominant bilinguals of English (L2), English–dominant bilinguals of Spanish (L2) or Balanced bilinguals (exposed to both English and Spanish since early childhood). Again, the LexTALE is available online and has been designed to run in PRAAT, Matlab and Presentations, but the participants will take it using PsychoPy at the experiment testing location immediately following the experimental tasks. 
A second instrument used in this study is the Bilingual Language Profile (BLP), which is a survey based assessment tool for determining language dominance (Birdsong, Gertken, \& Amengual, 2012). It assesses language history, use, proficiency and attitudes of participants in less than 10 minutes. This assessment tool is commonly used in language studies with a focus on bilingualism and is available for free under the creative commons license. This tool allows researchers to collect information about participants demographics—name, age, sex, place of residence and educational background. It also allows for the participant to talk about their language history, use, proficiency and attitudes. For the current dissertation project, this survey will be administered while the participants are in the experiment testing location through Google Forms as designed by the creators of the BLP. Since no modification will be made to the BLP survey, the scores will be automatically calculated as designed by the creators. In all cases, the BLP survey will be completed after the experimental trials to avoid any confounding factors of language activation, Spanish or English, the participants will be able to choose whether they would like to take the survey with Spanish or English instructions.


\textbf{DELETE ME Any special instrumentation could be included here (I am not sure that mine deserves a devoted section to instrumentation.}


%-----------------------------------
%	SUBSECTION 2-4
%-----------------------------------

\subsection{Procedure}
Participants were seated in front of a laptop computer with a USB button box in order to complete the experiment in PsychoPy. At the beginning of the experiment, participants entered in demographic information as asked in the basic language profile (BLP). Once they had entered the demographic information, the participants completed the Spanish version of LexTALE-ESP---a Spanish vocabulary test. 

Participants were instructed that they would be presented with a sequence of letters of for which they were to find in a list of words that would appear on the screen one by one. They were instructed to respond only if they had identified the sequence of letters in the word on the screen and to do nothing otherwise. The participants were also instructed to respond as fast as possible and were reminded with feedback screens staggered throughout the trials encouraging faster response times. The participants were first presented with 8 practice trials that followed the same criteria and procedure as the 48 experimental trials. Each trial began with text "Encuentre" above the sequence of letters, henceforth the target, which was presented in all capital letters in the center of the screen. The initial trial screen contained the target for 4 seconds before returning to a blank screen for 500 milliseconds. Following the blank screen, a list of ten words was presented randomly one at a time for 2000 milliseconds each with a 150 millisecond interstimulus interval. Only one word, the carrier item, in each list of ten words contained the target while the other nine words were simply filler items. The target was always found at the beginning of the carrier item. None of the filler items shared any of its first three letters with the target. The search target remained in the upper right hand portion of the screen to serve as a reminder while all ten words from the list were presented. Even when a response was made, only the first response was recorded, but the experiment did not progress until the 2000 millisecond presentation time had passed. Once all ten words from the list had been presented, the next trial began with a new target for participants to find in the next set list of ten words. No carrier or filler items were repeated through out the entire experiment. Participants were also given an optional 2 minute break halfway through the experimental trials.

Following the experiment participants also completed the English version of the LexTALE vocabulary test. 

Following the LexTALE task, participants also completed the bilingual language profile (BLP).

\textbf{combine this paragraph with the first one in this section}
The participants are given on screen instructions in Spanish that tell them that they will be presented with an auditory sound fragment—the target—followed by words that may or may not contain the sound fragment for which they are listening. They will be told to respond as quickly and accurately as possible when they hear the target sound fragment in the speech to which they are listening. Following the target presentation, words will be presented auditorily in sets of 12 carrier items, including only one of the critical words, with each carrier item being separated by 0.5 second interstimulus intervals (ISI). For example, the participants are instructed to find an auditorily presented fragment (“ba” or “bal”) in the following set of 12 auditorily presented words (“sotana”, “sonido”, “picota”, “torpeza”, “balada”, “semilla”, “rendija”, “renombre”, “sordera”, “tortuga”, “tersura”, and “sortija”). Each carrier item, including the critical item only has a 2 second response window from the onset of stimulus presentation before continuing on to the next carrier item in the list. The participants are instructed to press a single response button on a button response box using their preferred or dominant hand as soon as they have identified the target in one of the carrier items and are instructed to do nothing when the fragment is not present. The syllable monitoring experiment will include a practice run of 2 blocks containing 12 trials followed by 24 separate blocks of 12 experimental trials with each set of blocks—practice versus experimental—having their own directions that differ only in indicating whether the coming blocks will be practice trials or actual experimental trials.

\textbf{DELETE ME WHEN FINISHED
The first refer to mainly scope and size of research project while the procedure section is usually much more in-depth. Here, you connect your procedures to those in already published articles when possible and given detailed descriptions of classifications and scales used in procedure. Essentially ensure the reader does not suspect anything is being hidden and the researcher is honest. Do not repeat any unnecessary information in subsequent experiments of the paper.}


%----------------------------------------------------------------------------------------
%	SECTION 3
%----------------------------------------------------------------------------------------

\section{Results}

The filler trials will be removed from the data collected and so that only the critical trials remain. Participants who have less than a 90 percent success rate on critical trials will be removed from the data (Number of participants will be reported). Once the fillers and participants who have not completed the task successfully have been removed, a second pass will remove any individual participant responses under 200 milliseconds following the lower criteria range used by Bradley et al. (1993). (The percentage of trials removed will be reported here). For the latency data, only the correct responses to critical trials will be analyzed.

\textbf{DELETE ME describe the analyses the researcher has done, but do not overload. Instead of creating a laundry list of statistics, create the story you want to tell using only the statistics that are related to addressing your problem. 
For each task, review the hypothesis, give the statistics and say what the result of the test means. 
Do not discuss the findings until you reach the discussion session. 
This is where tables and figures can help keep the paper looking clean and crisp instead of cluttered unorganized statistical test lists that are hard to follow. 
Figures show patterns while tables give details.}


%----------------------------------------------------------------------------------------
%	SECTION 4
%----------------------------------------------------------------------------------------

\section{Discussion}

split into several small sections following each individual experiment’s results sections where applicable. 
The General Discussion steps back and begins with an overview of the problem and then the findings. A general rule of thumb is to keep this section shorter than the introduction. Only give limitation directly related to the current study, not the general limitation of the research or the field as a whole and be sure to give a good reason for why these limitations are not as bad as they sound on the surface.


%----------------------------------------------------------------------------------------
%	SECTION 5
%----------------------------------------------------------------------------------------

\section{References}

Insert references for this chapter here.
%% Chapter Template

\chapter{Visual Word Recognition} % Main chapter title

\label{Chapter4} % Change X to a consecutive number; for referencing this chapter elsewhere, use \ref{Chapter4}
%----------------------------------------------------------------------------------------
%	SECTION 0
%----------------------------------------------------------------------------------------

\section{Abstract}

Give abstract here

Keywords: (list all words necessary)

%----------------------------------------------------------------------------------------
%	SECTION 1
%----------------------------------------------------------------------------------------

\section{Introduction}

Write my introduction to the syllabification article here (Will write the introduction LAST)
The section of your article most likely to be read, not skimmed or skipped. The first paragraph or two is the overview of the article: describe the problem, question or theory motivating the research. 

%-----------------------------------
%	SUBSECTION 1-1
%-----------------------------------
\subsection{Background}

In the second part, describe relevant theories, review past research and give more details on the current research question. Do not forget about signposting, which headings and subheadings can naturally create for the reader.

%-----------------------------------
%	SUBSECTION 1-2
%-----------------------------------

\subsection{Present Study}
The third section titled, “The Present Experiment” or “The Present Research”, follows and contains experimental descriptions and how they address the questions being asked.
For most articles, keep your introduction under 10 pages

%----------------------------------------------------------------------------------------
%	SECTION 2
%----------------------------------------------------------------------------------------

\section{Methods}

Write this section FIRST
is the section that describes how the research was conducted. A good one shows how well thought out the experiment design is and allows other researchers to easily replicate it. This section also follows a formula:

%-----------------------------------
%	SUBSECTION 2-1
%-----------------------------------
\subsection{Participants}

Give accurate descriptions of participant groups, how they were classified, number of participants, etc.

%-----------------------------------
%	SUBSECTION 2-2
%-----------------------------------

\subsection{Design}

Give an accurate idea of how the overall project was designed, what previous studies is it based off, theoretical principles, etc.

DO NOT write about what participants do in this section.

%-----------------------------------
%	SUBSECTION 2-3
%-----------------------------------
\subsection{Instrumentation}

Any special instrumentation could be included here (I am not sure that mine deserves a devoted section to instrumentation.


%-----------------------------------
%	SUBSECTION 2-4
%-----------------------------------

\subsection{Procedure}

The first refer to mainly scope and size of research project while the procedure section is usually much more in-depth. Here, you connect your procedures to those in already published articles when possible and given detailed descriptions of classifications and scales used in procedure. Essentially ensure the reader does not suspect anything is being hidden and the researcher is honest. Do not repeat any unnecessary information in subsequent experiments of the paper.


%----------------------------------------------------------------------------------------
%	SECTION 3
%----------------------------------------------------------------------------------------

\section{Results}

describe the analyses the researcher has done, but do not overload. Instead of creating a laundry list of statistics, create the story you want to tell using only the statistics that are related to addressing your problem. 

For each task, review the hypothesis, give the statistics and say what the result of the test means. 

Do not discuss the findings until you reach the discussion session. 

This is where tables and figures can help keep the paper looking clean and crisp instead of cluttered unorganized statistical test lists that are hard to follow. 

Figures show patterns while tables give details.


%----------------------------------------------------------------------------------------
%	SECTION 4
%----------------------------------------------------------------------------------------

\section{Discussion}

split into several small sections following each individual experiment’s results sections where applicable. 
The General Discussion steps back and begins with an overview of the problem and then the findings. A general rule of thumb is to keep this section shorter than the introduction. Only give limitation directly related to the current study, not the general limitation of the research or the field as a whole and be sure to give a good reason for why these limitations are not as bad as they sound on the surface.


%----------------------------------------------------------------------------------------
%	SECTION 5
%----------------------------------------------------------------------------------------

\section{References}

Insert references for this chapter here. 
%% Chapter Template

\chapter{Conclusion} % Main chapter title

\label{Chapter5} % Change X to a consecutive number; for referencing this chapter elsewhere, use \ref{Chapter5}

%----------------------------------------------------------------------------------------
%	SECTION 1
%----------------------------------------------------------------------------------------

\section{Main Section 1}

This proposal has provided a basic background and introduction to the project that will be extended in the formal dissertation. This project will encompass the more in-depth the background knowledge of the research related to the syllable’s role in speech processing. The dissertation will not only attempt to replicate previous findings with a different speaker population, but also fill in some of the research veins that have yet to be investigated. For example, French–English early bilinguals have been tested in a word segmentation paradigm, but a comparative group, Spanish–English early bilinguals, have yet to investigated. Given that Spanish and French share many similarities—i.e. Syllable timed and clear, unambiguous syllable boundaries—replicating the results of the French–English bilinguals with this new population would give additional support for a syllable-based segmentation strategy. In addition, the research proposed for this dissertation would give new insight on late bilinguals—speakers of English who learned Spanish as adults—which will elucidate several uses of the syllable. First, it will provide research data on different bilingual speaker population, and secondly, it will allow for a comparison of degrees of bilingualism. In other words, it will shed light on whether or not native speaker of a language such as English that is known not to implement a syllable-based segmentation strategy ever learn to utilize a syllable-based segmentation strategy when becoming bilingual in a language where native speakers do make productive use of the strategy.
The second main thread of research contained in the forthcoming dissertation will be on the syllable’s role in lexical access. Up to this point, it has mainly been tested in the visual priming paradigm, but Italian has also been studied under a cross-modal fragment priming paradigm. The dissertation will add to this literature by again investigating whether or not the path to lexical access reflects a syllabic representation as well as if the degree of bilingualism is a pertinent factor of its usefulness. In addition, it will expand our knowledge base by replicating the the Italian results in Spanish. 
Both lines of research will lean towards a better understanding how the critical age hypothesis and the role of schooling may relate to speech processing. For example, if a difference is found between the early bilingual group and the late bilingual groups, it would provide evidence that once a person has passed through puberty, learning new segmentation or lexical access strategies are no longer acquirable for speakers learning a language as an adult. If on the other hand, no difference is found between the early bilingual group and late bilingual groups, it would provide evidence that these strategies are still available and able to be learned by adult language learners. A third possibility exists in that a difference will be found between late learners of English bilingual population and the other two bilingual groups—balanced bilinguals and late learners of Spanish bilinguals. If this were the case, then a case could be drawn for the role of schooling in which syllables are taught as a means of breaking words into smaller chunks for Spanish native speakers by Spanish native speakers as the reason for the development of the syllable-based processing strategy. This case can only be built when the early bilingual group consists of speakers who grew up using both Spanish and English, but received formal education only in Spanish.
No results have been report as part of this proposal since no data has been collected under the provisions of the Internal Review Board (IRB) at the University of Arizona. However, the final dissertation project will collect data from participants through the use of several psycholinguistic methodologies. In chapter 2 of the dissertation, a syllable monitoring task will be used in the auditory modality to determine whether or not the syllable has some representation in the minds of Spanish speakers that can be utilized to segment spoken speech. A second task will also be used in the second chapter, which will fall under a syllabification task. This task will ask participants to indicate the first syllable of the word in a two-choice forced decision task. In chapter 3, a visual priming experiment will be used in combination with a lexical decision task in order to determine whether or not the syllable plays a role in the facilitation of lexical retrieval by bilingual speakers of Spanish. Finally in chapter 4, a cross-modal auditory fragment priming in combination with a visually display word lexical decision task will be used to further test the role of the syllable in lexical access of bilingual Spanish–English speakers. In all experiments, reaction times captured in the responses to monitoring and lexical decisions tasks will be the variable of interest in determining whether or not the syllable has a facilitatory effect for word segmentation or lexical access.


%-----------------------------------
%	SUBSECTION 1
%-----------------------------------
\subsection{Subsection 1}

may not need

%-----------------------------------
%	SUBSECTION 2
%-----------------------------------

\subsection{Subsection 2}
probably need something here

%----------------------------------------------------------------------------------------
%	SECTION 2
%----------------------------------------------------------------------------------------

\section{Main Section 2}

may not need

%----------------------------------------------------------------------------------------
%	SECTION 3
%----------------------------------------------------------------------------------------

\section{Main Section 3}

may not need

%-----------------------------------
%	SUBSECTION 1
%-----------------------------------
\subsection{Subsection 1}

May not need

%-----------------------------------
%	SUBSECTION 2
%-----------------------------------

\subsection{Subsection 2}
May not need
%----------------------------------------------------------------------------------------
%	SECTION 4
%----------------------------------------------------------------------------------------

\section{Main Section 4}

May not need 
%\include{Chapters/Chapter6}

%----------------------------------------------------------------------------------------
%	THESIS CONTENT - APPENDICES
%----------------------------------------------------------------------------------------
%
% Appendices Requirements (if used)
% Follow content and precede References, unless used for manuscript/previous published article disserations
% are distinguished by capital letters, Appendix A – title, Appendix B – title, etc.
% must be correctly listed in Table of Contents
% Pages must be numbered in manner consistent with the rest of the document
% Permissions for including previously published articles are included
%
%If no appendices are used, comment out all active lines in this section
%\pagestyle{thesis}

\renewcommand\thechapter{}

\appendix % Cue to tell LaTeX that the following "chapters" are Appendices

% Include the appendices of the thesis as separate files from the Appendices folder
% Uncomment the lines as you write the Appendices

\include{Appendices/AppendixA}
%\include{Appendices/AppendixB}
%\include{Appendices/AppendixC}

%----------------------------------------------------------------------------------------
%	BIBLIOGRAPHY
%----------------------------------------------------------------------------------------
%
% References must be included in dissertation
% References must be consisent (check your citation style formatting

\printbibliography[heading=bibintoc]

%----------------------------------------------------------------------------------------

\end{document}  
