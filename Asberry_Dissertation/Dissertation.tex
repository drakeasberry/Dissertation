%%%%%%%%%%%%%%%%%%%%%%%%%%%%%%%%%%%%%%%%%
% Masters/Doctoral Thesis 
% LaTeX Template
% Version 2.5 (27/8/17)
%
% This template was downloaded from:
% http://www.LaTeXTemplates.com
%
% Version 2.x major modifications by:
% Vel (vel@latextemplates.com)
%
% This template is based on a template by:
% Steve Gunn (http://users.ecs.soton.ac.uk/srg/softwaretools/document/templates/)
% Sunil Patel (http://www.sunilpatel.co.uk/thesis-template/)
%
% Template license:
% CC BY-NC-SA 3.0 (http://creativecommons.org/licenses/by-nc-sa/3.0/)
%
% Modifications made by Drake Asberry to make document compliant with 
% University of Arizona dissertation formatting Guide 2019-01-29
% https://grad.arizona.edu/gsas/dissertations-theses/dissertation-and-thesis-formatting-guides
%
%%%%%%%%%%%%%%%%%%%%%%%%%%%%%%%%%%%%%%%%%

%----------------------------------------------------------------------------------------
%	PACKAGES AND OTHER DOCUMENT CONFIGURATIONS
%----------------------------------------------------------------------------------------

\documentclass[
12pt, % The default document font size, options: 10pt, 11pt, 12pt
%oneside, % Two side (alternating margins) for binding by default, uncomment to switch to one side
english, % ngerman for German
doublespacing, % Single line spacing (singlespacing), alternatives: onehalfspacing or doublespacing
%draft, % Uncomment to enable draft mode (no pictures, no links, overfull hboxes indicated)
nolistspacing, % If the document is onehalfspacing or doublespacing, uncomment this to set spacing in lists to single
liststotoc, % Uncomment to add the list of figures/tables/etc to the table of contents
%toctotoc, % Uncomment to add the main table of contents to the table of contents
%parskip, % Uncomment to add space between paragraphs
%nohyperref, % Uncomment to not load the hyperref package
headsepline, % Uncomment to get a line under the header
chapterinoneline, % Uncomment to place the chapter title next to the number on one line
%consistentlayout, % Uncomment to change the layout of the declaration, abstract and acknowledgements pages to match the default layout
openany, % eliminates all extra blank pages from entire document
]{DoctoralThesis}\usepackage[]{graphicx}\usepackage[]{color}
% maxwidth is the original width if it is less than linewidth
% otherwise use linewidth (to make sure the graphics do not exceed the margin)
\makeatletter
\def\maxwidth{ %
  \ifdim\Gin@nat@width>\linewidth
    \linewidth
  \else
    \Gin@nat@width
  \fi
}
\makeatother

\definecolor{fgcolor}{rgb}{0.345, 0.345, 0.345}
\newcommand{\hlnum}[1]{\textcolor[rgb]{0.686,0.059,0.569}{#1}}%
\newcommand{\hlstr}[1]{\textcolor[rgb]{0.192,0.494,0.8}{#1}}%
\newcommand{\hlcom}[1]{\textcolor[rgb]{0.678,0.584,0.686}{\textit{#1}}}%
\newcommand{\hlopt}[1]{\textcolor[rgb]{0,0,0}{#1}}%
\newcommand{\hlstd}[1]{\textcolor[rgb]{0.345,0.345,0.345}{#1}}%
\newcommand{\hlkwa}[1]{\textcolor[rgb]{0.161,0.373,0.58}{\textbf{#1}}}%
\newcommand{\hlkwb}[1]{\textcolor[rgb]{0.69,0.353,0.396}{#1}}%
\newcommand{\hlkwc}[1]{\textcolor[rgb]{0.333,0.667,0.333}{#1}}%
\newcommand{\hlkwd}[1]{\textcolor[rgb]{0.737,0.353,0.396}{\textbf{#1}}}%
\let\hlipl\hlkwb

\usepackage{framed}
\makeatletter
\newenvironment{kframe}{%
 \def\at@end@of@kframe{}%
 \ifinner\ifhmode%
  \def\at@end@of@kframe{\end{minipage}}%
  \begin{minipage}{\columnwidth}%
 \fi\fi%
 \def\FrameCommand##1{\hskip\@totalleftmargin \hskip-\fboxsep
 \colorbox{shadecolor}{##1}\hskip-\fboxsep
     % There is no \\@totalrightmargin, so:
     \hskip-\linewidth \hskip-\@totalleftmargin \hskip\columnwidth}%
 \MakeFramed {\advance\hsize-\width
   \@totalleftmargin\z@ \linewidth\hsize
   \@setminipage}}%
 {\par\unskip\endMakeFramed%
 \at@end@of@kframe}
\makeatother

\definecolor{shadecolor}{rgb}{.97, .97, .97}
\definecolor{messagecolor}{rgb}{0, 0, 0}
\definecolor{warningcolor}{rgb}{1, 0, 1}
\definecolor{errorcolor}{rgb}{1, 0, 0}
\newenvironment{knitrout}{}{} % an empty environment to be redefined in TeX

\usepackage{alltt} % The class file specifying the document structure

\usepackage[utf8]{inputenc} % Required for inputting international characters

\usepackage[T1]{fontenc} % Output font encoding for international characters

\usepackage{mathpazo} % Use the Palatino font by default

\usepackage[backend=bibtex,style=authoryear,natbib=true]{biblatex} % Use the bibtex backend with the authoryear citation style (which resembles APA)

\addbibresource{references.bib} % The filename of the bibliography

\usepackage[autostyle=true]{csquotes} % Required to generate language-dependent quotes in the bibliography

\usepackage{todonotes} % Make renderable notes in pdf
%----------------------------------------------------------------------------------------
%	MARGIN SETTINGS
%----------------------------------------------------------------------------------------

\geometry{
	paper = letterpaper, % Change to a4paper for international letter
	inner = 1in, % Inner margin
	outer=1in, % Outer margin
	bindingoffset=.5in, % Binding offset
	top=1in, % Top margin
	bottom=1in, % Bottom margin
	%showframe, % Uncomment to show how the type block is set on the page
}

%----------------------------------------------------------------------------------------
%	THESIS INFORMATION
%----------------------------------------------------------------------------------------
% Replace content within green curly brackets to reflect your own work and names

\thesistitle{Syllabification, Visual Segmentation and Visual Word Recognition in Spanish–English Bilinguals} % Your thesis title, this is used in the title, committee approval and abstract pages, print it elsewhere with \ttitle

\chair{Dr. Miquel Simonet} % Your dissertation chair's name, this is used in the title, abstract and committee approval pages, print it elsewhere with \chairname

\cochair{Co-Chair Name} % Your dissertation co-chair's name, this is not currently used anywhere in the template, but would be used in the title and committee approval page, print it elsewhere with \cochairname

\examiner{Examiner Name} % Your examiner's name, this is not currently used anywhere in the template, print it elsewhere with \examname

\degree{Doctor of Philosophy} % Your degree name, this is used in the title page and abstract, print it elsewhere with \degreename

\author{Drake Asberry} % Your name, this is used in the title page and abstract, print it elsewhere with \authorname

\addresses{Your Address} % Your address, this is not currently used anywhere in the template, print it elsewhere with \addressname

\subject{Hispanic Linguistics} % Your subject area, this is not currently used anywhere in the template, print it elsewhere with \subjectname

\keywords{} % Keywords for your thesis, this is not currently used anywhere in the template, print it elsewhere with \keywordnames

\university{University of Arizona} % Your university's name and URL, this is used in the title page and abstract, print it elsewhere with \univname

\department{Graduate Interdisciplinary Program in Second Language Acquisition and Teaching} % Your department's name and URL, this is used in the title page and abstract, print it elsewhere with \deptname

\group{\href{http://researchgroup.university.com}{Research Group Name}} % Your research group's name and URL, this is not currently used anywhere in the template, print it elsewhere with \groupname

\facultyA{Dr. Michael Hammond} % Your first faculty member's name and URL(can be added between green curly brackets before the member's name. Do not use this for your COMMITTEE CHAIR This is used in the committee approval page, print it elsewhere with \facnameA

\facultyB{Dr. Adam Ussishkin} % Your first faculty member's name and URL(can be added between green curly brackets before the member's name. This is used in the committee approval page, print it elsewhere with \facnameB

\facultyC{Dr. Maryia Fedzechkina} % Your first faculty member's name and URL(can be added between green curly brackets before the member's name. This may optionally be added to the committee approval page when number of committee members requires it, print it elsewhere with \facnameC

\facultyD{5th Committee Member Name} % Your first faculty member's name and URL(can be added between green curly brackets before the member's name. This may optionally be added to the committee approval page when number of committee members requires it, print it elsewhere with \facnameD

\defense{April 15, 2020} % This should have the long date format of your scheduled defense, print it elsewhere with \defensedate

\AtBeginDocument{
\hypersetup{pdftitle=\ttitle} % Set the PDF's title to your title
\hypersetup{pdfauthor=\authorname} % Set the PDF's author to your name
\hypersetup{pdfkeywords=\keywordnames} % Set the PDF's keywords to your keywords
\hypersetup{hidelinks} % Set all hyperlinks standard color text
}

%\usepackage{times} % Uncomment to use Times New Roman font
\IfFileExists{upquote.sty}{\usepackage{upquote}}{}
\begin{document}

%\frontmatter % Uncomment to use roman page numbering style (i, ii, iii, iv...) for the pre-content pages

\pagestyle{plain} % Default to the plain heading style until the thesis style is called for the body content


%-------------------------------------------------------------------------------------------------------------------------------
%	Pre-Thesis Content
%-------------------------------------------------------------------------------------------------------------------------------

%----------------------------------------------------------------------------------------
%	TITLE PAGE
%----------------------------------------------------------------------------------------
%
% All requirements of the graduate college as of 2019-01-30
% The title page must be the first page of your document (All pages must be numbered and match 
% the numbers listed in Table of Contents. However, a page number is not required to be printed
% on the actual title page).
% The title page must meet the following requirements:
% Title is set in ALL CAPS
% Student name matches official name in UAccess
% Rule line appears
% Official Department Name is Used
% Degree is indicated correctly
% Copyright year matches year of graduation on page

\begin{titlepage}
\begin{singlespacing} % needed for documents set to 1.5 or 2.0 spacing, Comment out otherwise
\begin{center}

\vfill

\MakeUppercase{\ttitle}\\ %Thesis Title in ALL CAPS
\vspace{0.4in}
by\\ \vspace{0.4in}
{\authorname}\\ % Places author name as specified in preamble
\vspace{0.6in}
\HRule \\[0.1cm] % Horizontal line
Copyright \textcopyright\space\authorname\space{\the\year}\\ % Copyright Date

\vspace{0.4in}

A Dissertation Submitted to the Faculty of the\\ % University required text
\vspace{0.4in}
\MakeUppercase{\deptname} \\  % Department name in Small Caps
\vspace{0.4in}
In Partial Fulfillment of the Requirements \\ \medskip % University required text
For the Degree of \\  % University required text
\vspace{0.4in}
\MakeUppercase{\degreename} \\ % Thesis type
\vspace{0.4in} 
In the Graduate College \\  % University required text
\vspace{0.4in}
\MakeUppercase{The \univname} \\ % University name in Small Caps
\vspace{0.6in}
%\normalsize
{\the\year}\\[4cm] % date
%\includegraphics{Logo} % University/department logo - uncomment to place it

\vfill
\end{center}
\end{singlespacing}% needed for documents set to 1.5 or 2.0 spacing, Comment out otherwise
\end{titlepage}

%\cleardoublepage %Uncomment to add blank page after Title page.


\setcounter{page}{2} % Starts pagination at 2 on the Committee Approval Form with no page number displayed on Title page.

%----------------------------------------------------------------------------------------
%	COMMITTEE APPROVAL PAGE
%----------------------------------------------------------------------------------------
%
% All requirements of the graduate college as of 2019-01-30
% The committee approval page must be the second page of your document
% The committee approval page must meet the following requirements:
% Title on approval page matches title on page 1 (Title Page)
% Dissertation chair (or co-chair) is indicated
% All members and chair (or co-chairs) have signed the approval page
% Date of defense is listed

%\addchaptertocentry{Committee Approval Page} % Add the committee approval page to the table of contents
\begin{singlespacing} % needed for documents set to 1.5 or 2.0 spacing, Comment out otherwise
\begin{center}
%\large

THE \MakeUppercase{\univname} \\
GRADUATE COLLEGE
\end{center}

\vspace*{0.3in}

\noindent As members of the Dissertation Committee, we certify that we have read the dissertation prepared by \authorname \space entitled "\ttitle "\space and recommend that it be accepted as fulfilling the dissertation requirement for the Degree of \degreename.

\vspace*{0.3in}

\noindent\underline{\makebox[4.0in][r]{}} \hspace{0.4in} Date: \defensedate \\
{\bfseries\chairname}\\
\emph{(Chair)}
\vspace*{0.3in}

\noindent\underline{\makebox[4.0in][r]{}} \hspace{0.4in} Date: \defensedate \\
{\bfseries\facnameA}\\
\emph{(Member)}
\vspace*{0.3in}

\noindent\underline{\makebox[4.0in][r]{}} \hspace{0.4in} Date: \defensedate \\
{\bfseries\facnameB}\\
\emph{(Member)}
\vspace*{0.3in}

\noindent\underline{\makebox[4.0in][r]{}} \hspace{0.4in} Date: \defensedate \\
{\bfseries\facnameC}\\
\emph{(Member)}
\vspace*{0.5in}

% If 4th committee member is needed, copy the preceding 4 lines, change to facnameD in copied lines
% You will then need to adjust vertical spacing to keep committee approval page to 1 page length

\noindent Final approval and acceptance of this dissertation is contingent upon the candidate's submission of the final copies of the dissertation to the Graduate College.

\vspace*{0.2in}

\noindent I hereby certify that I have read this dissertation prepared under my direction and recommend that it be accepted as fulfilling the dissertation requirement.
\vspace*{0.5in}

\noindent\underline{\makebox[4.0in][r]{}} \hspace{0.4in} Date: \defensedate \\
Dissertation Director: \chairname \\
%{\bfseries \emph{Instructor \\ Hispanic Linguistics}} % Update hard-coded to job title and department
\vfill
\end{singlespacing}% needed for documents set to 1.5 or 2.0 spacing, Comment out otherwise



%----------------------------------------------------------------------------------------
%	STATEMENT BY AUTHOR
%----------------------------------------------------------------------------------------
%
% No longer required for the graduate college as of 2019-01-30
% Uncomment all lines in this section  with "%%" at the beginning if your document requires it

%%\begin{statement}
%%\begin{singlespacing} % needed for documents set to 1.5 or 2.0 spacing, Comment out otherwise
%%\addchaptertocentry{\authorshipname} % Add the declaration to the table of contents

%The following block of text was the required text of the Graduate College (2019-02-01)
%%This dissertation has been submitted in partial fulfillment of the requirements for an advanced degree at the \univname\space and is deposited in the University Library to be made available to borrowers under rules of the Library. \\ \smallskip 

%%Brief quotations from this dissertation are allowable without special permission, provided that an accurate acknowledgement of the source is made. Requests for permission for extended quotation from or reproduction of this manuscript in whole or in part may be granted by the copyright holder.

%%\vspace*{0.3in}
%%\begin{center} 
%%SIGNED: \authorname
%%\end{center}
%%\end{singlespacing}% needed for documents set to 1.5 or 2.0 spacing, Comment out otherwise
%%\end{statement}


%----------------------------------------------------------------------------------------
%	ACKNOWLEDGEMENTS
%----------------------------------------------------------------------------------------
%
% Acknowledgements are not a necessary item. Comment out if not being used 
% This calls the Acknowledgements.rnw

% Acknowledgements with LaTeX code only

%----------------------------------------------------------------------------------------

%----------------------------------------------------------------------------------------


%\addchaptertocentry{\acknowledgementname} % Add the acknowledgements to the table of contents

\begin{center}
\MakeUppercase{Acknowledgements}\\ \bigskip
\end{center}

I would like to thank all the professors/educators I have had up to this point that have encouraged me to pursue my dreams and reach my goals. It all first began with a wonderful German teacher in high school who not only challenged me, but also helped me overcome the challenges that faced me. Debbie Williams-Aurthur insprired me to learn language, learn culture and be open to diversity and other cultural perspectives\ldots


%----------------------------------------------------------------------------------------
%	DEDICATION
%----------------------------------------------------------------------------------------
%
% Dedications are not a necessary item. Comment out to remove from document
\dedicatory{For my family and friends. \\\bigskip Dedicated to my Granny and Pa who I lost during my time in Arizona. They have always supported me and rejoiced in my accomplishments. I only wish that they were still here today to celebrate the completion of this dissertation.} 


%----------------------------------------------------------------------------------------
%	Quotation
%----------------------------------------------------------------------------------------
%
% This page is not really necessary, but if you feel the need to include some quote here is your
% chance. 
%
% Uncomment to use. This calls the Quotation.rnw
%knit_child('FrontBackMatter/Quotation.rnw')

%----------------------------------------------------------------------------------------
%	LIST OF CONTENTS/FIGURES/TABLES PAGES
%----------------------------------------------------------------------------------------
%
% Table of Contents (TOC) must include:
% a: all major sections with the document in a consistent manner
% b: section headings in document must match their listings (exact words) in TOC
%
\tableofcontents % Prints the main table of contents
%
% Lists of figures and tables must include accurate page numbers
\listoffigures % Prints the list of figures

\listoftables % Prints the list of tables


%----------------------------------------------------------------------------------------
%	ABBREVIATIONS
%----------------------------------------------------------------------------------------
%
% Abbreviations are not a necessary item. Comment out the line below to remove from document
%
% This calls the Abbreviations.rnw Uncomment to include
%knit_child('FrontBackMatter/Abbvreviations.rnw')


%----------------------------------------------------------------------------------------
%	PHYSICAL CONSTANTS/OTHER DEFINITIONS
%----------------------------------------------------------------------------------------
%
% Constants are not a necessary item. Comment out the line below to remove from document
%
% This calls the Constants.rnw Uncomment to include
%knit_child('FrontBackMatter/Constants.rnw')


%----------------------------------------------------------------------------------------
%	SYMBOLS
%----------------------------------------------------------------------------------------
%
% Symbols are not a necessary item. 
%
% This calls the Symbols.rnw Uncomment to include
%knit_child('FrontBackMatter/Symbols.rnw')


%----------------------------------------------------------------------------------------
%	Abstract
%----------------------------------------------------------------------------------------
% This is required to appear before the first chapter of the dissertation

% This calls the Abstract.rnw

% Abstract with LaTeX code only

%----------------------------------------------------------------------------------------

%----------------------------------------------------------------------------------------

%
% UA Requirements for Abstract:
% Must follow Table of Contents OR List of Figures / List of Tables (when present)
% Required in final document for all submissions; Must precede Chapter 1 even in the case of 
% 3-article dissertations that may include a separate abstract specific to each article later in document.
% English version must included if the document is submitted in a language other than English
% When two language versions of the abstract are included, order does not matter
% Length is reasonable (at least 150 words, generally no more than 500 words)


\addchaptertocentry{\abstractname} % Add the abstract to the table of contents

\begin{center}
\MakeUppercase{Abstract}\\ \bigskip
\end{center}

The Thesis Abstract is written here (and usually kept to just this page). The page is kept centered vertically so can expand into the blank space above the title too\ldots


%----------------------------------------------------------------------------------------
%	THESIS CONTENT - CHAPTERS
%----------------------------------------------------------------------------------------

%\mainmatter % Begin numeric (1,2,3...) page numbering Uncomment if using roman numerals in front matter


\pagestyle{thesis} % Return the page headers back to the "thesis" style

% Include the chapters of the thesis as separate files from the Chapters folder
% Uncomment the lines as you write the chapters
%\renewcommand{\chaptermarkformat}{\thechapter}


% Chapter 1 with LaTeX code only

%----------------------------------------------------------------------------------------

%----------------------------------------------------------------------------------------


%----------------------------------------------------------------------------------------
% Load statistics in memory from separate file
%<<content1, child='stat.rnw'>>=
@
%----------------------------------------------------------------------------------------


\chapter{Introduction} % Main chapter title

\label{Chapter1} % Change X to a consecutive number; for referencing this chapter elsewhere, use \ref{Chapter2}

%----------------------------------------------------------------------------------------
%	SECTION 0
%----------------------------------------------------------------------------------------
% Not needed for introduction
%\section{Abstract}


%----------------------------------------------------------------------------------------
%	SECTION 1
%----------------------------------------------------------------------------------------

\section{Introduction}

This three article dissertation was written in order to complete the degree requirements of the Second Language Acquisition and Teaching program (SLAT) at the University of Arizona. The participant population for this dissertation are three distinct groups early and late Spanish–English bilingual speakers. The early bilingual group consists of speakers who began learning before the age of %INSERT AGE%. 
One of the late bilingual groups consists of native English speakers who were L2 learners of Spanish while the other consists of native Spanish speakers who were L2 learners of English. The dissertation investigates the ability of these three speaker populations and their ability to use the linguistic unit, a syllable, in their language processing strategies for Spanish. The three articles that compose this dissertation---chapters 2, 3 and 4---each correspond to one independent article and utiliize the same three bilingual speaker populations discussed above.

%-----------------------------------
%	SECTION 2
%-----------------------------------
\section{Background}
%\emph{this may be too direct of a start for the introduction, maybe something a little lighter or broad scheme (funnel approach).}
A syllable is a pronounceable linguistic unit of a given language that generally contains a highly sonorant sound as it nucleus—typically a vowel in most languages. Early research sought to discover the \emph{minimal perceptual unit} and the syllable was a logical and testable linguistic unit. In spoken language processing, the syllable made its highlight when it was found that participants could detect syllables faster than they could detect individual phonemes of which the syllable was comprised \citep*{Savin1970-oy}. This finding spurred interest in researchers focus on the syllable as well as other linguistic units such as word, phrases and sentences. The findings of these additional studies revealed that the processes involved in parsing spoken language were complex and a minimal perceptual unit was unlikely to be found. %\textbf{Should probably give more details about these studies here}.%
Specifically, one linguistic unit—phoneme, syllable, word, etc.—could not be the sole mechanism in which listeners of language break the speech stream into smaller or processable chunks \citep{Foss1973-ll,Healy1976-js,McNeill1973-bo}. 

Mehler and Hayes \parencite*{Mehler1981-wp} captured the need for a change in the direction of research regarding the syllable, “Traditionally, psycholinguistics research has invested the bulk of its efforts into uncovering the units used in speech processing. Although it is currently fashionable to claim that such work is pointless since it has no very clear outcome, many of the more meaningful advances in the field have come from projects whose framework included the problem of processing units.” They went on to delineate two different levels in which the research around the syllable could move forward: (1) The syllable as a phonological unit of the language which can efficiently explain the grammar of language and (2) The syllable as a unit which aides speech perception and language comprehension. As a result in the early 1980s, several researchers began refocusing their own investigations in accordance to this second vein of syllable research. %\textbf{Should cite and describe additional studies here}.% 
Even as researchers conducted more pointed research on the syllable's role in language processing strategies, the results continued to suggest research questions needed further subdivision. Ultimately, two distinct subprocesses of language processing in which the syllable may play a role---segmentation and lexical access---were proposed. 
% discuss the differences between segmentation and lexical access here.


%-----------------------------------
%	SECTION 3
%-----------------------------------

\section{Present Study}

This dissertation utilizes several different methodologies as a means to investigate research questions that fall under Mehler and Hayes \parencite*{Mehler1981-wp} second level of research. There are three overarching questions that underlie this dissertation project as a whole:
\begin{enumerate}
%revisit these research questions later
\item Whether or not the common linguistic unit—the syllable—is available to Spanish–English bilinguals when processing the Spanish language?
\item Whether or not the syllable is a strategy used by Spanish–English bilinguals in language segmentation and/or lexical access?
\item Does the age of acquisition, early versus late bilinguals, affect the ability and efficiency in which Spanish–English bilinguals can make use of the syllable?
\end{enumerate}

The three articles in this dissertation provide additional information about syllable structure and the representation of the syllable in Spanish to the knowledge base of the field. Within the segmentation strain of research, investigating differing types of bilingualism of the speaker who have the same two languages at their disposal provides new information to the fields of bilingualism and second language acquisition. At the time of writing this dissertation, the majority of research studies have been conducted with monolingual speakers. When studies have investigated the role of the syllable in language processing by bilinguals, they have generally compared the bilinguals against monolingual speakers of the two respective languages. However, previous studies suggest that bilinguals are not simply the summation of two independent monolingual speakers. For example, %Cutler citation% 
found that French--English bilinguals listening to French responded in the same manner as French monolinguals, but when listening to English, these speakers did not respond similarly to English monolinguals. Therefore, this dissertation does not utilize monolingual Spanish or monolingual English speaker controls, but instead utilizes three distinct bilingual populations of Spanish and English which are compared against each other. %\textbf{This type of setup allows for the control group to be the Spanish native L2 English speaker group and two test groups be early and late learners of Spanish.}% 
Many previous studies on lexical access have generally been conducted using the visual word recognition paradigm to study speech processing in conjunction with the syllable. Since the visual word recognition tasks have been conducted with both monolingual and bilingual populations, this dissertation seeks to add to the knowledge base of the syllable’s role through the replication of previous findings while testing a different population of bilingual speakers under the visual word recognition paradigm.

The format for the remaining chapters of this dissertation project will be as follows in order to address the main overarching questions: chapter 2 explores the representation of the syllable in the minds of Spanish–English bilinguals with a two option forced-choice syllabic intuition task. Chapter 3 utilizes a visual word segmentation task to compare the efficiency of the processes employed by the bilingual speakers. Chapter 4 includes a visual priming experiment with a lexical decision task that explores the syllable’s role in lexical access by Spanish–English bilinguals. Chapter 5 then concludes the dissertation project by drawing overall conclusions and how all three individual studies were necessary to draw the conclusions that were borne out through the various testing methodologies used to explore the role of the syllable in Spanish language processing in the three articles of the dissertation.


%----------------------------------------------------------------------------------------
%	SECTION 2
%----------------------------------------------------------------------------------------

%\section{Methods}

%Not needed

%-----------------------------------
%	SUBSECTION 2-1
%-----------------------------------
%\subsection{Participants}

%Probably still important here to some degree

%Give accurate descriptions of participant groups, how they were classified, number of participants, etc. 


%----------------------------------------------------------------------------------------
%	SECTION 3
%----------------------------------------------------------------------------------------

%\section{Results}

%Not needed


%----------------------------------------------------------------------------------------
%	SECTION 4
%----------------------------------------------------------------------------------------

%\section{Discussion}
%Not needed


%----------------------------------------------------------------------------------------
%	SECTION 5
%----------------------------------------------------------------------------------------

\section{References}

%Insert references for this chapter here.










% Chapter 2 with LaTeX code only

%----------------------------------------------------------------------------------------

%----------------------------------------------------------------------------------------


%----------------------------------------------------------------------------------------
% Load statistics in memory from separate file
%<<content2, child='stat.rnw'>>=
@
%----------------------------------------------------------------------------------------

\chapter{Syllabification} % Main chapter title

\label{Chapter2} % Change X to a consecutive number; for referencing this chapter elsewhere, use \ref{Chapter2}

%----------------------------------------------------------------------------------------
%	SECTION 0
%----------------------------------------------------------------------------------------

\section{Abstract}

Give Syllabification abstract here

Keywords: (list all words necessary)

%----------------------------------------------------------------------------------------
%	SECTION 1
%----------------------------------------------------------------------------------------

\section{Introduction}

%Talk about syllabificationdifferences in syllabification between spanish and english

The syllable intuition experiment is an important step in the process because how the three groups of participants syllabify the words will have a direct effect on how fast or useful the syllable is in their segmentation strategy. It is expected that syllabic intuitions will vary based on being a native speaker of Spanish, an early learner of Spanish or a late learner of Spanish. These differences are likely to stem from age of acquisition, language dominance or type of schooling. This experiment will begin to build a data source of Spanish–English bilinguals syllabification of Spanish words that will help to determine whether or not age of acquisition or language dominance are sources are different syllabification patterns. 


%\textbf{DELETE ME Write my introduction to the syllabification article here (Will write the introduction LAST) The section of your article most likely to be read, not skimmed or skipped. The first paragraph or two is the overview of the article: describe the problem, question or theory motivating the research.} 

%-----------------------------------
%	SUBSECTION 1-1
%-----------------------------------
\subsection{Background}

%In the second part, describe relevant theories, review past research and give more details on the current research question. Do not forget about signposting, which headings and subheadings can naturally create for the reader.

%-----------------------------------
%	SUBSECTION 1-2
%-----------------------------------

\subsection{Present Study}
%Treiman


\begin{figure}[th]
\centering
\includegraphics{Figures/three_way_test}
\decoRule
\caption[A test of figures]{testing 1, 2, 3}
\label{fig:Test}
\end{figure}


%\textbf{DELETE ME The third section titled, “The Present Experiment” or “The Present Research”, follows and contains experimental descriptions and how they address the questions being asked. For most articles, keep your introduction under 10 pages}

%----------------------------------------------------------------------------------------
%	SECTION 2
%----------------------------------------------------------------------------------------

\section{Methods}

%Write this section FIRST is the section that describes how the research was conducted. A good one shows how well thought out the experiment design is and allows other researchers to easily replicate it. This section also follows a formula:

%-----------------------------------
%	SUBSECTION 2-1
%-----------------------------------
\subsection{Participants}

The participants were split into three distinct populations of Spanish–English bilinguals. The Spanish–dominant group consisted of native speakers of Spanish that lived in Sonora, Mexico and learned English after %INSERT AGE%. 
The English–dominant group were native English speakers that lived in Tucson, Arizona and learned Spanish after%INSERT AGE%. 
The early bilingual group consisted of participants who lived in Tucson, Arizona, but were exposed to both English and Spanish before the %INSERT AGE%.
\emph{The participants will be placed into their appropriate group based off of the combined results of Bilingual Language Profile (BLP), the LexTALE and the LexTALE-Esp vocabulary tests.} 

Based off of responses in the BLP, participants who report proficiency in a language other than English and Spanish will be removed from the data analysis. According to the findings of Lemhöfer \& Broersma \parencite*{Lemhofer2012-hz}, the LexTALE vocabulary tests can distinguish between lower intermediate (up to 59 percent), upper intermediate (60–80 percent) and advanced (above 80 percent) levels of proficiency based on average percent correct responses. Participants that score lower than %INSERT CUTOFF PERCENTAGE HERE% 
were excluded due to having too low of a proficiency score in one of the two languages. The Spanish–dominant group had a %INSERT CUTOFF PERCENTAGE HERE% 
success rate in Spanish and it was higher than their English success rate. The English-dominant group scored %INSERT CUTOFF PERCENTAGE HERE%
or higher in English and it was higher than their Spanish scores. The early bilingual group had a correctness score of %INSERT CUTOFF PERCENTAGE HERE% 
or more in both English and Spanish. %INSERT NUMBER OF PARTICIPANTS% 
participants were excluded from the analysis due to falling below the minimum standards used to describe each group. %INSERT NUMBER OF PARTICIPANTS%
participants were recruited from each bilingual population and are reported in the analysis section. 
\emph{During data collection, more participants than reported here were collected since it was expected that several participants would be removed for one or more of the reasons listed above. When more than %INSERT NUMBER OF PARTICIPANTS% 
participants remained eligible after removing participants who did not fit the population criteria, a random sampling of %INSERT NUMBER OF PARTICIPANTS% 
participants were selected from eligible pool of participants.}

%\textbf{DELETE ME Give accurate descriptions of participant groups, how they were classified, number of participants, etc.}

%-----------------------------------
%	SUBSECTION 2-2
%-----------------------------------
%participants lexTALE BLP 
% Computers, box in instrumentation
\subsection{Instrumentation}
The LexTALE and the LexTALE-Esp are tasks used to correlate vocabulary knowledge and language proficiency in English and Spanish respectively \citep{Izura2014-yw,Lemhofer2012-hz}. The LexTALE-Esp has also been shown to discriminate well between highly proficient Spanish speaking participants with different language dominances \citep{Ferre2017-jq}. The LexTALE and LexTALE-Esp tests will be used in order to group participants into the appropriate bilingual population—Spanish–dominant bilinguals of L2 English, English–dominant bilinguals of L2 Spanish or early bilinguals who were exposed to both English and Spanish before %INSERT AGE%. 
LexTALE is publicly available online and has been designed to run in PRAAT, Matlab and Presentations. For data collection purposes in the current study, participants completed both the LexTALE and LexTALE-Esp using PsychoPy. 

A second instrument used in this study was the Bilingual Language Profile (BLP), which is a survey based assessment tool for determining language dominance \citep{Birdsong2012-wd}. It assesses language history, use, proficiency and attitudes of participants in less than 10 minutes. This assessment tool has been used in numerous language studies with a focus on bilingualism and is available for free under the creative commons license. This tool allowed for the collection of information about participants demographics---name, age, sex, place of residence and educational background---which took place at experiment setup in the current study. It also allowed for participants to indicate their language history, use, proficiency and attitudes and was the last task complete during the current study. Since the BLP survey was completed after all experimental trials, the participants were able to choose whether they received Spanish or English instructions for the survey. The BLP is publicly available in a paper-based format or electronic format through the use of Google Forms. In an effort to make the experiment seamless as possible, the participants in the current study took the BLP within the PsychoPy platform as well. 

%Button box
The button box contained five different color buttons---left to right (white, green, blue, yellow and red)---which were referenced in all instruction 

%\textbf{DELETE ME Any special instrumentation could be included here (I am not sure that mine deserves a devoted section to instrumentation.}


%-----------------------------------
%	SUBSECTION 2-3
%-----------------------------------

\subsection{Design}
The syllabic intuitions were elicited from participants through a forced two-choice task and were given as much time as needed to respond. Participants were shown a real word in Spanish on the screen and then were presented with two options below that represented a CV or CVC structure. There were 24 critical word pairs that were used as the stimuli in a previous experiment. Each word pair shared the first three letters, but differed in initial syllable structure. Using the example word pair \emph{balada–baldosa} to illustrate, it can easily been seen that both words begin with the letters \emph{bal}. However, the initial syllable structure of \emph{balada} is \emph{ba}, a CV structure, while the initial syllable structure of \emph{baldosa} is \emph{bal}, a CVC structure, when following standard Spanish syllabification. All other word pairs followed the same pattern where one word had a CV word-initial syllable structure and the other word began with a CVC syllable---where C represents a consonant and V represents a vowel. The order of presentation for the two syllabic options, CV or CVC, remained the same for the entire experiment for the participant, but was counterbalanced across participants. In order to counterbalance for hand dominance and visual presentation in this task, half of the participants used the left response button to indicate a CVC response and the right response button to indicate a CV response while the other half of the participants did the opposite. Since the position of response buttons aligned with screen position, half of the participants visually saw the orthographic representation of the CVC syllable to the left of screen center and the CV syllable orthographic representation on the right while the other half of the participants saw the opposite visual displays for syllable orthographic representations.

%\textbf{DELETE ME Give an accurate idea of how the overall project was designed, what previous studies is it based off, theoretical principles, etc. DO NOT write about what participants do in this section.}


%-----------------------------------
%	SUBSECTION 2-4
%-----------------------------------

\subsection{Procedure}
Participants were seated in front of a laptop computer with a USB button box in order to complete the experiment in PsychoPy. At the beginning of the experiment, participants entered in demographic information as asked in the Basic Language Profile (BLP). Once they had entered the demographic information, all participants recruited from for this study completed the current experiment as an additional task of two additional ongoing experiments. Some participants completed this syllable intuition experiment following a visual word segmentation experiment while others completed it following a lexical decision experiment. Depending on the prior experiment completed, the overall procedural order of the time in the lab changed.

When the syllable intuition experiment was followed by the visual segmentation experiment, %INSERT NUMBER OF PARTICIPANTS%
participants began the experimental session with the Spanish vocabulary task---LexTALE-Esp. Immediately upon the completion of the LexTALE-Esp, participants were given instructions for the practice portion of the visual segmentation experiment. Following the practice portion, participants were given an instruction screen that indicated they had completed the practice portion, reminded about the controls needed for the experiment, allowed to ask any remaining question about the process. When the participant was ready to begin the actual experiment, they pressed the white button on the response box to begin. Once the participants had finished the entire visual segmentation experiment, they were presented with instructions for the practice trials of the syllable intuition experiment. Once they completed the practice trials, they were given a second instruction screen indicating they had completed the practice portion and were about to begin the real experiment. Once participants had completed the syllabic intuition experiment, they completed the LexTALE-Eng followed by the BLP. 

When the syllable intuition experiment was followed lexical decision based priming experiment, the order differed slightly. In this scenario, participants were immediately presented with instructions for practice trials for the lexical decision priming experiment. Once they had completed the practice trials, a new instruction screen appeared stating that they were getting ready to start the actual experimental trials. Then participants received instructions for the syllable intuition experiment of which was immediately followed the three remaining tasks---LexTALE-Esp, LexTALE-Eng and the BLP.

The participants were instructed that they would see Spanish words presented individually on the computer screen and that each word would be followed by two options for the first syllable of the presented word. There was no time constraint for the participant to select the option that they considered to be the first syllable of the presented word and the experiment would not progress until the participant had indicated a response. %In the second experiment (Lexical Access), this was initially ran with 4 second time limit. It affected participants 062, 065-067, 069-072, 074-076 so these 11 participants should be checked for missing values.% 
In total, there were 10 practice trials that followed the same criteria and procedure as the 48 experimental trials presented to the participants. %same 11 participants had timing issue in practice trials, check if possible but values may not have been stored.% 
For each trial, the word in lowercase letters appeared in the center of the screen for 1500 milliseconds before being replaced by the two syllable options---one a CV syllable and the other a CVC syllable---which remained on the screen, in all capital letters, until a response was entered. Following each response, the screen remained blank for 500 milliseconds before the next word appeared. %It may be good to create a visual representation of the experiment here to illustrate what each participant saw.%
For half of the participants, the CV syllable option was on the left side of the screen while the CVC syllable option was on the right. For the other half of the participants, the CVC syllable option was on the left side of the screen while the CV syllable option was on the right. Participants were given separate instruction screens for practice and experimental trials that differed only in letting the participant know whether they are about to enter a practice or experimental phase. %\emph{Is this too repetitive? Should this be described here or in the Design section, currently it is in both} In this task it is worth noting that timing was not being measured but following the training session, most participants were trying to indicate their answer before the 1.5 second display time of the word since the syllables were always on the same side of the screen.

%\textbf{DELETE ME WHEN FINISHED The first refer to mainly scope and size of research project while the procedure section is usually much more in-depth. Here, you connect your procedures to those in already published articles when possible and given detailed descriptions of classifications and scales used in procedure. Essentially ensure the reader does not suspect anything is being hidden and the researcher is honest. Do not repeat any unnecessary information in subsequent experiments of the paper.}

%----------------------------------------------------------------------------------------
%	SECTION 3
%----------------------------------------------------------------------------------------

\section{Results}

The practice trials for syllabification were not recorded and therefore are not represented in the reported data below. Participants syllabification patterns are first analyzed by group and then are compared across groups. 


%Discuss the deviation by group from the standard Spanish syllabification pattern. (Error rates)
%Discuss whether significant differences between groups are found. (anova between subjects)

%Are there patterns related to syllable structure CV or CVC.

%stats from Miquel: 
% 2 choice forced decision task (A or B) 
% logistic regression
% log Anova (Arcsin or logit)
Columns needed include:
\begin{enumerate}
\item{participant}
\item{correct syllable}
\item{correct answer}
\item{left key}
\item{right key}
\item{condition}
\item{response}
\item{correct response}
\item{response time}
\end{enumerate}

%\textbf{DELETE ME describe the analyses the researcher has done, but do not overload. Instead of creating a laundry list of statistics, create the story you want to tell using only the statistics that are related to addressing your problem. For each task, review the hypothesis, give the statistics and say what the result of the test means. Do not discuss the findings until you reach the discussion session. This is where tables and figures can help keep the paper looking clean and crisp instead of cluttered unorganized statistical test lists that are hard to follow. Figures show patterns while tables give details.}

%----------------------------------------------------------------------------------------
%	SECTION 4
%----------------------------------------------------------------------------------------

\section{Discussion}

%\textbf{split into several small sections following each individual experiment’s results sections where applicable. The General Discussion steps back and begins with an overview of the problem and then the findings. A general rule of thumb is to keep this section shorter than the introduction. Only give limitation directly related to the current study, not the general limitation of the research or the field as a whole and be sure to give a good reason for why these limitations are not as bad as they sound on the surface.}


%----------------------------------------------------------------------------------------
%	SECTION 5
%----------------------------------------------------------------------------------------

\section{References}

%Insert references for this chapter here










% Chapter 3 with LaTeX code only

%----------------------------------------------------------------------------------------

%----------------------------------------------------------------------------------------


%----------------------------------------------------------------------------------------
% Load statistics in memory from separate file
%<<content3, child='stat.rnw'>>=
@
%----------------------------------------------------------------------------------------


\chapter{Visual Word Segementation} % Main chapter title

\label{Chapter3} % Change X to a consecutive number; for referencing this chapter elsewhere, use \ref{Chapter3}
%----------------------------------------------------------------------------------------
%	SECTION 0
%----------------------------------------------------------------------------------------

\section{Abstract}

Give Segmentation abstract here

Keywords: (list all words necessary)

%----------------------------------------------------------------------------------------
%	SECTION 1
%----------------------------------------------------------------------------------------

\section{Introduction}

People break down speech and language on a daily basis in both verbal and written communication. It seems that most people do not have a problem accomplishing this task with little effort although the task is actually quite complex. \todo{This sentence may be stronger than it should be.} Previous studies have investigated the underlying mechanisms in an attempt to better understand how these processes are handled in the brain. While some researchers conducted studies using auditory stimuli, many have resorted to visual stimuli in the methodologies. This is the case of the current study investigating word segmentation through a visual paradigm. Word segmentation is thought to be a separate process from accessing the lexicon. Word segmentation does not need to contact the lexicon before processes can be kicked off which is in direct contrast to making a lexical decision where you must search through the mental lexicon before deciding whether or not a word is valid in the language. In other words, segmentation processes are thought to be pre-lexical and often occur much quicker after the stimuli presentation that post-lexical decisions. In order to segment language, meaning does not have to be attached to the segmented chunks of sounds or letters. This means a speaker can segment language by isolating or detecting certain language units in languages for which they have no knowledge—a common practice in psycholinguistic research to ensure pre-lexical decisions are being made \parencite{Cutler1986-zl}. Understanding that people can successfully segment language of an unknown language, having naïve and native listeners of a language search language for a particular sequence of sounds or graphemes tells researchers several things. First, it allows for comparisons of different languages and testing hypotheses about language segmentation strategies depending on experimental design. Furthermore, it can give additional support the effects that may or may not be found are not due to experimental design or items. Several linguistic units were previously investigated for their roles in language processing which include the phoneme, syllable, word, and sentence. 


%\textbf{DELETE ME Write my introduction to the syllabification article here (Will write the introduction LAST) The section of your article most likely to be read, not skimmed or skipped. The first paragraph or two is the overview of the article: describe the problem, question or theory motivating the research.} 

%-----------------------------------
%	SUBSECTION 1-1
%-----------------------------------
\subsection{Background}
%There is not much in the way of speech segmentation in the visual paradigm
The most relevant unit to the current study is the syllable which served as the focus of nearly three decades of research that fell into the second level of investigation of the syllable discussed by Mehler and Hayes \parencite*{Mehler1981-wp}---a unit which aides speech perception and comprehension. Researchers abandoned the search for the minimal perceptual unit after many attempts to identify a single clear linguistic unit of perception and turned their focus towards the syllable’s role in speech perception despite the fact it was not likely to be a minimal perceptual unit for speech processing. While the monitoring paradigm was not going to help researchers discover the minimal perceptual unit used to process spoken language, this methodology served them well to test the syllable’s role in speech processing. In the syllable monitoring paradigm a target syllable such as \emph{pa, pal, ba, bal, ca} or \emph{car} is presented to the participant. Then a participant must identify the target syllable which is embedded in language---spoken or written. The language is presented as a list of words, carrier items, rather than connected and continuous speech. French subjects in a French syllable monitoring experiment were asked to find CV or CVC syllables in lists of bisyllabic words, which had initial syllable structures of CV or CVC, consonant-vowel or consonant-vowel-consonant, where both the target and carrier items were presented auditorily \parencite{Mehler1981-vi}. For example, an experimental trial started with the target presentation “The target is pa”, which was immediately followed by the list of words, ranging from two to five words. Each list may or may not contain a critical word as the last item of the list. Only one out of every ten sequences of words contained a critical carrier item. The critical trials contained only one carrier item of a word pair \emph{PA.LA.CE–PAL.MIER} or \emph{CAR.TON–CA.ROTTE} that appeared as the last item of the word sequence, where both words always shared the initial three phonemes, but they differed in syllable structure as denoted by the period indicating standard French syllabification. The experimental conditions varied based on the target syllable structure and initial syllable structure of the carrier item. The structures of the target and carrier item could match (find “pa” in PA.LACE, “pal” in PAL.MIER, “ca” in CA.ROTTE or “car” in CAR.TON) or mismatch (find “pal” in PA.LACE, “pa” in PAL.MIER, “ca” in CAR.TON or “car” in CA.ROTTE). In order to collect the reaction time data the researchers needed, participants were asked to press a response key as soon as they had heard their target sound fragment in the speech they were hearing, French speakers found the target syllable faster when the initial syllable structure of the carrier item matched the syllable structure of the target. The real French nouns used as critical items showed a faster detection time of the target \emph{pa} when \emph{PALACE} rather than \emph{PALMIER} was the critical item. Likewise, the target \emph{pal} was found faster when the critical item was \emph{PALMIER} rather than \emph{PALACE}\citep{Mehler1981-vi}. This finding was named the “crossover effect” and was found by Mehler, Dommergues, Frauenfelder and Segui \parencite*{Mehler1981-vi} which can be described as faster detection times when target syllable structure matched initial syllable of carrier word. The crossover effect was not replicated by native English speakers who monitored real English words where one word in each pairing had an unclear syllable boundary---an ambisyllabic [l] \parencite{Cutler1986-zl}. In other words, English speakers did not show differences in the speed in which they identified the target syllable according to the syllable structure of the carrier item. Finding \emph{pa} in \emph{PA.LACE} was not easier than finding \emph{pal} in \emph{PA.LACE} nor was finding \emph{pal} in \emph{PAL.PI.TATE} easier than finding \emph{pa} in \emph{PAL.PI.TATE}. Researchers speculated that something in the auditory stimuli may have been the culprit of the inconsistent finding. They followed up on this hypothesis by having native English speakers monitor the French nouns used by Mehler et al. \parencite*{Mehler1981-vi} where syllable boundary ambiguity was not present in the carrier items. Given that these were naive listeners of French, the segmentation process is being examined here because these listeners would have had no lexical entries for any of the French words they heard. The English listeners again did not show the crossover effect which suggested that syllable based segmentation strategy used by the the native French listeners was not a strategy used in speech segmentation by English monolinguals even when the language stimuli supported the use of a syllable based segmenation strategy \parencite{Cutler1986-zl}. Importantly, this finding suggested that language segmentation strategies are not universal, but are language specific. In order to give additional strength to their argument, the researchers used the English experimental items that did not produce the crossover effect with native English speakers, and ran the same experiment with native speakers of French who had not learned English. The French listeners still showed the crossover effect even when listening to an unknown language—in this case, English, a language that does not support a syllable based segmentation strategy. 
% stopped editing here (10/1/2020)
Given the nature of the differences found between English and French monolingual speakers’ strategy for speech segmentation, the next logical step while remaining in the same of vein of research was to investigate the segmentation strategy used by bilinguals. In a series of experiments, both English and French language modes of French–English balanced bilingual speakers were tested \parencite{Cutler1992-qq}. The French mode of the experiment used the same stimuli as Mehler et al. \parencite*{Mehler1981-vi} which had showed a strong syllabic effect with French monolinguals, but showed no such effect with the bilingual population. Post-hoc analysis split the bilinguals into two groups according to language dominance as reported by the participant in response to the following question, “Suppose you developed a serious disease, and your life could only be saved by a brain operation which would unfortunately have the side effect of removing one of your languages. Which language would you choose to keep?” This manner of determining language dominance was then used in all remaining experiments of the study. Under this post-hoc analysis where language dominance was considered for the experiments conducted in French, French-dominant listeners patterned like French monolinguals. Likewise, English dominant listeners patterned like English monolinguals from previous studies for the experiments conducted in French. In the French versions of the experiments, French-dominant participants exhibited the syllable-based segmentation strategy while the English-dominant participants did not show evidence of a syllable-based segmentation.  In the English versions of the experiments, French dominant speakers patterned unlike the French monolinguals while the English dominant speakers continued to pattern like the English monolinguals. Like previous studies with English monolinguals and the English-dominant bilinguals of the Cutler et al. \parencite*{Cutler1986-zl}, the French-dominant bilingual were not able to find “pa” in PALACE easier than “pal” in PALACE nor were they able to find “pal” in PALPITATE faster than “pa” in PALPITATE. Where French was considered the dominant language of the speaker, it appeared that they had identified the ineffectiveness of the syllable based segmentation strategy and were able to inhibit its application while listening to English. When English was considered the dominant language of the speaker, it appeared that these speakers could not utilize a syllable based segmentation strategy despite the fact that one of their two languages could use a syllabic segmentation strategy.  This suggested that bilinguals—even in their most balanced form—are not two monolinguals within a bilingual mind.
English and French are quite different in their phonological structure: (1) French is syllable-timed while English is stressed-timed, (2) French has fixed stress while English has variable stress, (3) French has no vowel reduction while English has rampant vowel reduction and (4) French has clear syllable boundaries while English has ambiguous syllable boundaries. As a result of the phonological structures of French and English, in the previous experiments stressed syllables were always the carrier syllables for English speakers while unstressed syllables were always the carrier syllables for the French speakers. This stress factor was considered as a potential confound in that stressed syllables are known to carry more phonetic details and last longer than unstressed syllables. To overcome this difference, a different group of bilingual speakers were recruited—Catalan–Spanish bilinguals—because stress is variable in both languages and can therefore be controlled \parencite{Sebastian-Galles1992-xd}. Unlike French and English that varied on multiple factors, Catalan and Spanish only differ in vowel reduction—Catalan has vowel reduction while Spanish does not. With stress being controlled across both languages, vowel reduction can be isolated as it is allowed in Catalan, like in English, but it is not in French or Spanish. They found that Catalan dominant speakers monitoring Catalan speech produced the crossover effect only when the initial syllable of the carrier item was unstressed. Spanish was also considered in the same study where the researchers found no crossover effect by Spanish dominant speakers monitoring in Spanish. Given the similarities between Spanish and French, both have clear and unambiguous syllable boundaries nor vowel reduction, it is surprising that they did not find a crossover effect with stressed or unstressed initial syllables of the carrier items as they did with Catalan speakers. Sebastián-Gallés et al. \parencite*{Sebastian-Galles1992-xd} attempted to force a post-lexical decision by the Spanish dominant speakers with the incorporation of an additional semantic relatedness task. This succeeded in slowing the reaction time by an average of 250 milliseconds and found a syllabic effect in both initially stressed and unstressed Spanish words. The syllabic effect similar to the those found in Sebastián-Gallès et al. \parencite*{Sebastian-Galles1992-xd} was later replicated in Italian by Tabossi et al. \parencite*{Tabossi2000-xn}.
In contrast to the findings of Sebastián-Gallés et al. \parencite*{Sebastian-Galles1992-xd} , Bradley, Sánchez-Casas and García-Albea \parencite*{Bradley1993-qq}  found a crossover effect when Spanish speakers monitored Spanish carrier items. They also found no crossover effect for English speakers monitoring English carrier items. Similar to Cutler et al. \parencite*{Cutler1986-zl}, Bradley et al. \parencite*{Bradley1993-qq}  tested naive listeners with the same material. They found no crossover effect for English monolinguals monitoring Spanish carrier items, which was comparable to the French findings. However, unlike the French findings, the Spanish monolinguals monitoring in English also showed no syllabic segmentation effect. Bradley et al. \parencite*{Bradley1993-qq} then turned to Spanish–English bilinguals where they again found no crossover effects when monitoring Spanish carrier items. This suggests that these native speakers of Spanish and English L2 speakers have abandoned their native segmentation strategy even when listening to one of their native languages (Spanish). This result again differs from the Cutler et al. \parencite*{Cutler1986-zl} French–English bilinguals because the French kept the native syllable-based segmentation strategy when listening to French and abandoned it only when listening to English where it was no longer effective.


%In the second part, describe relevant theories, review past research and give more details on the current research question. Do not forget about signposting, which headings and subheadings can naturally create for the reader.

%-----------------------------------
%	SUBSECTION 1-2
%-----------------------------------

\subsection{Present Study}
The second article of this dissertation will ground itself here in the previous literature by first attempting to replicate the Spanish findings of Bradley et al. \parencite*{Bradley1993-qq}. with the Spanish-dominant bilingual group, which will serve as the control, and the English-dominant bilingual group, which will serve as the control floor. The third bilingual group in the experiment will provide new knowledge on the syllable’s role in speech processing.  The first group, henceforth the Spanish-dominant group, will consist of participants who grew up in Mexico speaking only Spanish and later learned their L2 English after the age of %INSERT AGE.% 
%The participants in this group will be recruited from Guanajuato, Mexico and will have received their formal grade-level schooling in Spanish. The Spanish-dominant group will serve as the control in this experiment. The second group, the English-dominant group or late learners of Spanish, will consist of participants who grew up in Arizona speaking only English and later learned their L2 Spanish after the age of %INSERT AGE%. 
%The participants for this group will be recruited from Tucson, Arizona and will have received their formal grade-level schooling in English. This group will serve as the first of two experimental groups. The last group, the Balanced-bilingual group or early learners of Spanish, will consist of participants that were exposed to both Spanish and English from infancy. These participants will have acquired both Spanish and English simultaneously, but unlike the Spanish-dominant group, they will have received their formal grade-level schooling in English. The Balanced-bilingual group participants will also be recruited from Tucson, Arizona and serve as the second experimental group. 
Utilizing three distinct groups of bilingual Spanish--English speaker populations, it is possible isolate differences in segmentation strategies that are due to the age of acquisition. The experiment is run completely in Spanish mode and the expectation is that the Spanish-dominant group will employ a syllabic based segmentation strategy while the English-dominant group will not. The heritage speaker group could go in one of two ways since the experiment is conducted in Spanish---a language that encourages the use of a syllabic segmentation strategy. The first possible outcome is that these early learners of Spanish will exhibit the same pattern as the Spanish-dominant group. If this is the outcome of the segmentation experiment, then it would provide stronger evidence for a syllable-based segmentation strategy for speakers of Spanish. The second option is that the these participants will pattern like their English-dominant counterparts and fail to employ the syllable-based segmentation despite the fact that the language input supports such an approach. This would contradict the findings of monolingual and Spanish-dominant bilingual speakers of English, who appear to use this syllable-based strategy. If these were to be the results of the first experiment, it would give additional support to the findings of Bradley et al. \parencite{Bradley1993-qq}. 
The main purpose of this study is to determine whether a representation of the syllable is a represented linguistic unit in the minds of bilingual Spanish speakers. This article addresses the following questions:
\begin{enumerate}
\item{Does a representation of the syllable exist in minds of Spanish–English bilinguals which is available to aide in pre- or post-lexical levels of segmenting spoken Spanish?}
\item{Does the age of acquisition of Spanish in Spanish–English bilinguals determine whether or not syllabic intuitions of Spanish match the intuitions of native Spanish speakers?}
\end{enumerate}
%I am not sure that this can be discussed since IRB was not obtained for this:
%This first question on the representation of the syllable has been one with unclear results in previous research. As a precaution to building this entire dissertation on less-than-stable previous findings, a quasi-pilot study was completed in the summer of 2018. One PsychoPy experiment that included two separate tasks—an identification task and a syllabification task—was designed and conducted completely in Spanish. Eight (8) native speakers of Spanish who attended school in a Spanish speaking country and learned English as adults were recruited from the University of Arizona main campus in Tucson, AZ. All participants were highly proficient in their English (L2) as they were currently enrolled in a masters or doctorate program or had just recently completed their graduate degree at the University of Arizona. This group showed a consistent syllabification pattern and showed a significant interaction between the visually presented targets and carrier items in the identification task. Participants responded faster when the target syllable structure matched the initial syllable structure of carrier item than when they did not coincide.
%Finding this effect in the population of Spanish dominant bilingual speakers of English, which was representative of one of the three populations being investigated, gave the confidence needed to proceed with the dissertation project. Therefore, the second chapter of the dissertation will use PsychoPy to conduct two experiments—an identification task using a syllable monitoring paradigm and a syllabic intuition task using a two option forced-choice methodology. Three additional instruments, Bilingual Language Profile (BLP), the LexTALE English vocabulary test and the LexTALE-Esp Spanish vocabulary test, will also be used to collect demographic information and language proficiency data that will be used for placement of participants into the appropriate bilingual group or to exclude them altogether from the data analysis. The nature of these tasks is discussed in the instruments section of the experiment 1. However, it is important to note that both of these tasks will take place following the completion of experiments 1 and 2, syllable monitoring and syllabic intuition experiments, in an effort to avoid biasing participants on the nature of the investigations in which they are participating.


%\textbf{DELETE ME The third section titled, “The Present Experiment” or “The Present Research”, follows and contains experimental descriptions and how they address the questions being asked. For most articles, keep your introduction under 10 pages}

%----------------------------------------------------------------------------------------
%	SECTION 2
%----------------------------------------------------------------------------------------

\section{Methods}

% Write this section FIRST. It is the section that describes how the research was conducted. A good one shows how well thought out the experiment design is and allows other researchers to easily replicate it. This section also follows a formula:

%-----------------------------------
%	SUBSECTION 2-1
%-----------------------------------
\subsection{Participants}

The participants were split into three distinct populations of Spanish–English bilinguals. The Spanish–dominant group consisted of native speakers of Spanish that lived in Sonora, Mexico and learned English after %insert age%. 
The English–dominant group were native English speakers that lived in Tucson, Arizona and learned Spanish after %insert age%. 
The early bilingual group consisted of participants who lived in Tucson, Arizona, but were exposed to both English and Spanish before the %insert age%.
\emph{The participants will be placed into their appropriate group based off of the combined results of Bilingual Language Profile (BLP), the LexTALE and the LexTALE-Esp vocabulary tests.} 

Based off of responses in the BLP, participants who report proficiency in a language other than English and Spanish will be removed from the data analysis. According to the findings of Lemhöfer \& Broersma \parencite*{Lemhofer2012-hz}, the LexTALE vocabulary tests can distinguish between lower intermediate (up to 59 percent), upper intermediate (60–80 percent) and advanced (above 80 percent) levels of proficiency based on average percent correct responses. Participants that score lower than %INSERT CUTOFF PERCENTAGE HERE%
were excluded due to having too low of a proficiency score in one of the two languages. The Spanish–dominant group had a %INSERT CUTOFF PERCENTAGE HERE% 
success rate in Spanish and it was higher than their English success rate. The English-dominant group scored %INSERT CUTOFF PERCENTAGE HERE% 
or higher in English and it was higher than their Spanish scores. The early bilingual group had a correctness score of %INSERT CUTOFF PERCENTAGE HERE% 
or more in both English and Spanish. %INSERT NUMBER OF PARTICIPANTS% 
participants were excluded from the analysis due to falling below the minimum standards used to describe each group.%INSERT NUMBER OF PARTICIPANTS%
participants were recruited from each bilingual population and are reported in the analysis section. \emph{During data collection, more participants than reported here were collected since it was expected that several participants would be removed for one or more of the reasons listed above. When more than %INSERT NUMBER OF PARTICIPANTS% 
participants remained eligible after removing participants who did not fit the population criteria, a random sampling of%INSERT NUMBER OF PARTICIPANTS%
participants were selected from eligible pool of participants.}

%\textbf{DELETE ME Give accurate descriptions of participant groups, how they were classified, number of participants, etc.}


%-----------------------------------
%	SUBSECTION 2-2
%-----------------------------------
\subsection{Instrumentation}

The LexTALE and the LexTALE-Esp are tasks used to correlate vocabulary knowledge and language proficiency in English and Spanish respectively \parencite{Izura2014-yw,Lemhofer2012-hz}. The LexTALE-Esp has also been shown to discriminate well between highly proficient Spanish speaking participants with different language dominances \parencite{Ferre2017-jq}. The LexTALE and LexTALE-Esp tests will be used in order to group participants into the appropriate bilingual population—Spanish–dominant bilinguals of L2 English, English–dominant bilinguals of L2 Spanish or early bilinguals who were exposed to both English and Spanish before %INSERT AGE HERE%. 
LexTALE is publicly available online and has been designed to run in PRAAT, Matlab and Presentations. For data collection purposes in the current study, participants completed both the LexTALE and LexTALE-Esp using PsychoPy. 
A second instrument used in this study was the Bilingual Language Profile (BLP), which is a survey based assessment tool for determining language dominance \parencite{Birdsong2012-wd}. It assesses language history, use, proficiency and attitudes of participants in less than 10 minutes. This assessment tool has been used in numerous language studies with a focus on bilingualism and is available for free under the creative commons license. This tool allowed for the collection of information about participants demographics---name, age, sex, place of residence and educational background---which took place at experiment setup in the current study. It also allowed for participants to indicate their language history, use, proficiency and attitudes and was the last task complete during the current study. Since the BLP survey was completed after all experimental trials, the participants were able to choose whether they received Spanish or English instructions for the survey. The BLP is publicly available in a paper-based format or electronic format through the use of Google Forms. In an effort to make the experiment seamless as possible, the participants in the current study took the BLP within the PsychoPy platform as well. 


%\textbf{DELETE ME Any special instrumentation could be included here (I am not sure that mine deserves a devoted section to instrumentation.}
%-----------------------------------
%	SUBSECTION 2-3
%-----------------------------------

\subsection{Design}

There were 24 real word pairs and 24 nonword pairs selected as critical items where the initial syllable structure varies between a CV and a CVC structure while the first three phonemes are shared between the two. Example real word pair included: ba.la.da–bal.do.sa (\emph{ballad–floor tile}), cu.le.bra–cul.pa.ble(\emph{snake–culprit}), mo.re.ra–mor.ci.llo(\emph{mulberry–beef shank}), and jo.ro.ba–jor.na.da(\emph{hump–day}). Example non word pairs included: ba.le.ga–bal.bu.sa, cu.li.tra–cul.se.ble, mo.ri.pa–mor.bo.llo, and jo.ru.ma–jor.te.da. In addition to the 24 critical real word pairs, another 294 real Spanish words were selected to use as fillers and are also balanced according to initial syllable structure—147 start with a CV syllable and 147 start with a CVC syllable. Likewise, in addition to the 24 critical nonword pairs, 294 nonwords were selected and counterbalanced for initial syllable structure. All critical items and fillers were trisyllabic and stressed on the penultimate syllable. In order to create the nonwords, the 48 real Spanish words were submitted to a nonword generator called Wuggy (see Appendix A). %INSERT Wuggy Citation. I would also like to make footnote or appendix giving all the parameters and instructions for this.% 

There were four different versions of the experiment necessary for counter-balancing purposes, which was accomplished by balancing across participants. For the example word-pair \emph{balada–baldosa}, participants in condition 1 searched for \emph{BA} in \emph{balada}, participants in condition 2 searched for \emph{BAL} in \emph{balada}, participants in condition 3 searched for \emph{BA} in \emph{baldosa} and participants in condition 4 searched for \emph{BAL} in \emph{baldosa}. Each participant was pseudorandomly assigned to one of the four versions of the experiment in the order in which they arrived to the experimental location. Since data collection took place where participants could not be easily identified into one population, the researchers best guess given the short introduction preceding the experiment. For example, the first native Spanish–L2 English speaker was assigned to condition A, the second to condition B, the third to condition C and the fourth to condition D. However, there were participants who were initially misrepresented causing the data to analyzed in the paper to not follow the order exactly. Since the participants were assigned to only one condition, no participant saw both critical words from any single critical word pair during the experiment. For example, if participant 1 is assigned to a version of the experiment where condition 1 (find “ba” in “balada”) is presented to them for the critical word pair “balada–baldosa” then participant 1 would not see conditions 2, 3 or 4 in their experiment. Each version of the experiment presents 24 CV and 24 CVC critical trials where half of each type of syllable structure contained a match between the syllable structure of the target and critical item while the other half are mismatched. Each block presented to the participant contains 1 critical trial and 9 filler trials which were also balanced for CV and CVC syllable structures.

%\textbf{DELETE ME Give an accurate idea of how the overall project was designed, what previous studies is it based off, theoretical principles, etc. DO NOT write about what participants do in this section.}

%-----------------------------------
%	SUBSECTION 2-4
%-----------------------------------

\subsection{Procedure}
Participants were seated in front of a laptop computer with a USB button box in order to complete the experiment in PsychoPy. At the beginning of the experiment, participants entered in demographic information as asked in the Basic Language Profile (BLP). Once they had entered the demographic information, they were asked if they had any question about the procedure and informed that each section of the experimental process would have instructions that always referenced the color of the button(s) needed to complete the next section.

All participants began the experimental session with the Spanish vocabulary task---LexTALE-Esp. %Do I make all these appendices: instructions% 
Immediately upon the completion of the LexTALE-Esp, participants were given instructions for the practice portion of the visual segmentation experiment. Following the practice portion, participants were given a new screen of instructions that indicated they had completed the practice portion, reminded about the controls needed for the segmentation experiment, and were allowed to ask any remaining question about the process. When the participant was ready to begin the actual experiment, they pressed the white button on the response box to begin. Once the participants had finished the entire visual segmentation experiment, they were presented with instructions for the practice trials of the syllable intuition experiment. Similar to the segmentation experiment, a new set of instructions came on the screen indicating that the practice trials were finished and that they were about to start the actual experiment. Following the syllable intuition experiment, participants completed the LexTALE-Eng and the BLP.

Participants were instructed that they would be presented with a sequence of letters of for which they were to find in a list of words that would appear on the screen one by one. They were instructed to respond only if they had identified the sequence of letters in the word on the screen and to do nothing otherwise. The participants were also instructed to respond as fast and accurate as possible and they were reminded with feedback screens staggered throughout the trials encouraging faster response times. The participants were first presented with 8 practice trials that followed the same criteria and procedure as the 48 blocks of experimental trials. Each trial began with text "Encuentre" above the sequence of letters, henceforth the target, which was presented in all capital letters in the center of the screen. The initial trial screen contained the target for 4 seconds before returning to a blank screen for 500 milliseconds. Following the blank screen, a list of ten words was presented randomly one at a time for 2000 milliseconds each with a 150 millisecond interstimulus interval (ISI). Only one word, the carrier item, in each list of ten words contained the target while the other nine words were simply filler items. The target was always found at the beginning of the carrier item. None of the filler items shared any of its first three letters with the target. The search target remained in the upper right hand portion of the screen to serve as a reminder while all ten words from the list were presented. When a response was made, only the first response was recorded, but the experiment did not progress until the 2000 millisecond presentation time had passed. Once all ten words from the list had been presented, the next block of trials began with a new target for participants to find in the next set list of ten words. For example, the participants are instructed to find a visually presented fragment \emph{BA} in the following set of 10 visually presented words \emph{sotana, sonido, picota, torpeza, balada, semilla, rendija, renombre, sordera, tortuga, tersura and sortija}. The participants were instructed to press a single response button on a button response box using their preferred or dominant hand as soon as they have identified the target in one of the carrier items and are instructed to do nothing when the fragment is not present. Participants were also given an optional 2 minute break halfway through the experimental trials.

%\textbf{DELETE ME WHEN FINISHED The first refer to mainly scope and size of research project while the procedure section is usually much more in-depth. Here, you connect your procedures to those in already published articles when possible and given detailed descriptions of classifications and scales used in procedure. Essentially ensure the reader does not suspect anything is being hidden and the researcher is honest. Do not repeat any unnecessary information in subsequent experiments of the paper.}


%----------------------------------------------------------------------------------------
%	SECTION 3
%----------------------------------------------------------------------------------------

\section{Results}
\section{Analysis Process}
The initial analysis did a quick run through for all participants that had completed the study. In total, there were 77 participants that completed the study, but 2 participants were removed due to language backgrounds. One native English speaking participant %part044
was removed because they reported being fluent in languages other than Spanish and English. One native Spanish participant %part047
was removed because they reported being born and raised outside the state of Sonora, Mexico. 

%This paragraph includes part044 & part047 data
The 77 participants provided a total of 36960 data points of which 3696 were in response to critical items while 33264 were in response to filler items. The first step in the analysis checked to ensure the data provided by participants were valid and that no participants had error rates greater than 10 percent for their responses to critical and filler items. In responses to filler items, 52 of the 77 participants committed 1 or more errors resulting in a total of 722 errors. Of the 722 errors, 475 of the errors were produced by participants with less than a 200 ms reaction time. It is thought that the participant was not intending to respond to these stimuli, but were the result of resting a hand on the button box or responding too late to the previous stimulus trial. As a result 247 responses (0.74\%) were actual errors made by participants and no participant was removed from analysis for having more than a 10 percent error rate to filler items. %add detail about tech errors in each subgroup high and low error committing folks

In terms of critical items, the error here made by participants is that they failed to respond to stimulus presented. There were only 44 (1.19\%) missed critical items committed by 28 different participants. One participant %part020
committed 6 errors (12.5\%) by not responding to critical items and was removed from the analysis. That left only 3648 critical item responses across 76 participants meaning that 27 people committed a total of 38 errors or 1.04\%.

 
\section{Initial Ideas Probably Delete later}
The filler trials will be removed from the data collected and so that only the critical trials remain. Participants who have less than a 90 percent success rate on critical trials will be removed from the data %INSERT NUMBER OF PARTICIPANTS%. 
Once the fillers and participants who have not completed the task successfully have been removed, a second pass will remove any individual participant responses under 200 milliseconds following the lower criteria range used by Bradley et al. \parencite*{Bradley1993-qq}. %INSERT NUMBER AND PERCENTAGE OF TRIALS REMOVED%. 
For the latency data, only the correct responses to critical trials will be analyzed.

Columns needed:
\begin{enumerate}
\item{participant}
\item{reaction time}
\item{match/mismatch}
\item{target-carrier}
\item{group}
\end{enumerate}

%linear regression/ANOVA
%Accuracy
% reaction time
%AorB logistic regression

		CV	   |	CVC	
			
-----------------------------------------

CV	|     match	   |  mismatch |

-----------------------------------------

CVC |  mismatch |     match    |

-----------------------------------------


%\textbf{DELETE ME describe the analyses the researcher has done, but do not overload. Instead of creating a laundry list of statistics, create the story you want to tell using only the statistics that are related to addressing your problem. For each task, review the hypothesis, give the statistics and say what the result of the test means. Do not discuss the findings until you reach the discussion session. This is where tables and figures can help keep the paper looking clean and crisp instead of cluttered unorganized statistical test lists that are hard to follow. Figures show patterns while tables give details.}


%----------------------------------------------------------------------------------------
%	SECTION 4
%----------------------------------------------------------------------------------------

\section{Discussion}

If Bradley et al. \parencite{Bradley1993-qq}. are replicated, there would be several possible factors that would warrant further investigation. Since the population of balanced bilinguals will be recruited from Tucson, Arizona and have completed their schooling in English, it would be a worthwhile endeavor to find a comparable balanced bilingual population who received their schooling in Spanish. This would allow for a comparison of the effect of schooling and explicit teaching of syllables, which typically occurs when children are taught to read and its relation to speech segmentation strategies. 
Another avenue to investigate would be the syllabic intuitions of participants, which could be a factor given that English and Spanish differ in ways similar to English and French participants of previous research studies. It may be that the three bilingual populations do not agree on the syllabic structures of the speech they are segmenting as a result of language background profiles. It will be possible to look at syllabic intuitions from the data that will be collected in a syllabic intuition task conducted in the second experiment of the first article in the proposed dissertation project. 
% Split into several small sections following each individual experiment’s results sections where applicable. The General Discussion steps back and begins with an overview of the problem and then the findings. A general rule of thumb is to keep this section shorter than the introduction. Only give limitation directly related to the current study, not the general limitation of the research or the field as a whole and be sure to give a good reason for why these limitations are not as bad as they sound on the surface.


%----------------------------------------------------------------------------------------
%	SECTION 5
%----------------------------------------------------------------------------------------

\section{References}

% Insert references for this chapter here.






% Chapter 4 with LaTeX code only

%----------------------------------------------------------------------------------------

%----------------------------------------------------------------------------------------


%----------------------------------------------------------------------------------------
% Load statistics in memory from separate file
%<<content4, child='stat.rnw'>>=
@
%----------------------------------------------------------------------------------------


\chapter{Visual Word Recognition} % Main chapter title

\label{Chapter4} % Change X to a consecutive number; for referencing this chapter elsewhere, use \ref{Chapter4}
%----------------------------------------------------------------------------------------
%	SECTION 0
%----------------------------------------------------------------------------------------

\section{Abstract}

Give Word Recognition abstract here

Keywords: (list all words necessary)

%----------------------------------------------------------------------------------------
%	SECTION 1
%----------------------------------------------------------------------------------------

\section{Introduction}

How humans understand and process spoken speech has been the subject investigation for decades in psycholinguistics. Processing spoken speech is a difficult task because there is a continuous and variable stream of language input, yet most speakers of a language appear to complete this complex task with little effort. Two specific processes used in spoken language processing are generally discussed in the literature—speech segmentation and lexical access. 
Speech segmentation addresses the strategies used by listeners when trying to parse spoken language input into processable chunks of speech regardless of whether the chunks have actual meaning or are from a language the listener understands (see Chapter 2 of the current dissertation, \citep{Cutler2002-ge, Cutler1986-zl, Cutler1992-qq, Dumay2002-hx, Finney1996-fw, Mehler1981-le, Segui1981-uf, Tabossi2000-xn}). Lexical access on the other hand, addresses the retrieval of meanings from the mental lexicon represented by phonological sequences. The mental lexicon is the warehouse of all lexical items—vocabulary words—that are represented in the memory of the language user. Unlike speech segmentation, accessing the lexicon in order to find words in a language does require a link to semantic meaning of a given entry. This implies that the listener has knowledge of what words are and are not in the language. Once the language user has broken the speech stream into manageable chunks, they can begin searching their mental lexicon for items that match the auditory input. This process is typically considered to be a combination of the activation of lexical items followed by a competition process that leads the listener to the top match stored in the lexicon or to the conclusion that input is not a real-word in the language.
Several studies have utilized visual word recognition to investigate the role of the syllable in accessing the mental lexicon. One such study investigated French native speakers’ ability to use the syllable to gain access to the mental lexicon with a lexical decision task under a masked priming paradigm \parencite{Ferrand1996-vu}. They found no significant results with their French participants, but only used a 29 millisecond prime display followed by a 14 millisecond backwards mask for a total stimulus onset asynchrony (SOA). Following the suggestion of Ferrand et al. \parencite*{Ferrand1996-vu} that the short SOA may have resulted in the inability of French speakers to use the syllable in lexical access, Carreiras and Perea \parencite*{Carreiras2002-mp} investigated syllable congruency as a means of measuring the facilitation or inhibition in the lexical access of Spanish words utilizing a longer SOA. While they found the a crossover effect using six letter disyllabic Spanish words (CV-CV-CV or CVC-CVC in syllabic structure), they also found a speed–accuracy trade-off effect with the two different SOAs—116 milliseconds versus 166 milliseconds. The participants responded faster to the lexical decision task when they had the additional 50 milliseconds of processing time in 166 msec SOA, but also made significantly more errors than participants who completed the same task with the 116 msec SOA.
The previous results had no way of disentangling the orthographic syllable from the phonological syllable. Álvarez, Carreiras and Perea \parencite*{Alvarez2004-nd} set out to investigate whether or not the phonological syllable was a unit used during visual word recognition. In their first experiment, they use disyllabic words with initial CV or CVC syllables allowing them to look at the differences between segmental and syllabic overlap by having the prime and targets sharing the first three segments, but varying syllabic structure between CV and CVC—i.e. ju.nas-JU.NIO and jun.tu-JU.NIO appeared in separate lists for counterbalancing across participants. They found that CV targets were responded to faster when the prime had a CV initial structure as well when compared to primes that had an initial CVC syllable structure. However, the same pattern was not found for CVC targets. Participants also made more errors on CVC target than CV targets. These findings do support that fact that syllabic priming effects can be found in lexical decision tasks at short SOAs, 64 millisecond in this study. It is important to note that only a 19 millisecond addition is added to the SOA used by Ferrand et al. \parencite*{Ferrand1996-vu} who did not find syllabic priming in French lexical decision task—also a Romance language like Spanish that has clear syllable boundaries. This suggests that very small adjustments in experimental design could affect the results of the experiment and should be considered carefully The second experiment focuses in more on the question of distinguishing between the phonological and orthographic syllable \parencite{Alvarez2004-nd}. The researchers here utilize the clear and unambiguous phoneme–grapheme correspondence and the fact that several graphemes map onto the same phoneme. For example, “b” and “v” graphemes both map onto the phoneme /b/ in Spanish. This study used four conditions to capture the effects of phonological versus orthographic syllables—vi.rel-VI.RUS is the same orthographic and phonological syllable; bi.rel-VI.RUS is a different orthographic syllable, but the same phonological syllable; lastly, vir.ga-VI.RUS and bir.ga-VI.RUS were set as control primes that shared the first three phonemes, but differed in syllable structure. They found faster responses were elicited when overlap included both orthographic and phonological representations than when it only had phonological overlap. Given the speed–accuracy trade-off between the reaction times and error rates of the orthographic-phonological and phonological only conditions, it suggested that phonological activation is occurring during the visual word recognition experiments. The final experiment of Álverez, Carreiras and Perea \parencite*{Alvarez2004-nd} showed that the initial syllable is the better key to lexical retrieval as phonological rime priming was not significantly different than primes sharing first three phoneme, but not the initial syllable structure. Additional evidence for the syllable congruency effect and the initial syllable being an essential key to lexical access comes from the event-related potential (ERP) experiments that presented word in two colors where one color represented the initial syllable or misrepresented it \citep{Carreiras2005-us}. They found syllable congruency effects in Spanish for low-frequency real words and pseudowords, but not for high-frequency words. This congruency effect found also interacted the lexicality judgements, but the time frames in which these two processes started differed in time. The congruency effect was available much earlier than the actual lexicality judgement suggesting that segmentation and lexical access are different processes that may work in conjunction with one another. Several years later, an interesting study introduced age and a medical condition, Alzheimer’s disease, into their experiment design to further investigate the syllable congruency effect \citep{Carreiras2008-ar}. They replicated the fourth experiment of Carreiras and Perea \parencite*{Carreiras2002-mp} and were able to find the syllable congruency effect in the control group—elderly people without Alzheimer’s disease—as well as the Alzheimer patients despite large differences in latencies across groups. Carreiras et al. \parencite*{Carreiras2008-ar} also tested syllable frequency effect in a second experiment. The syllable frequency effect was found in the young controls whereas the older age group appears to have a deterioration in the ability to inhibit lexical competitors.

% \textbf{DELETE ME Write my introduction to the syllabification article here (Will write the introduction LAST) The section of your article most likely to be read, not skimmed or skipped. The first paragraph or two is the overview of the article: describe the problem, question or theory motivating the research. }

%-----------------------------------
%	SUBSECTION 1-1
%-----------------------------------
\subsection{Background}

%In the second part, describe relevant theories, review past research and give more details on the current research question. Do not forget about signposting, which headings and subheadings can naturally create for the reader.

%-----------------------------------
%	SUBSECTION 1-2
%-----------------------------------

\subsection{Present Study}
The current study seeks to replicate the previous findings and expanding on the existing literature by including the three varying degrees of Spanish–English bilingual populations. Given the nature of three bilingual populations, the Spanish-dominant group should replicate the previous findings that Álvarez et al. \parencite*{Alvarez2004-nd} and Carreiras and Perea \parencite*{Carreiras2002-mp} found with monolingual Spanish speakers. Given the Canary Island dialect of Spanish used in the Álvarez et al. \parencite*{Alvarez2004-nd} where letters “z, s, and c” when followed by “e” and “i” are all pronounced as an /s/, all of the experimental items are transferable to the Mexican dialects of Spanish being studied in the current article. Due to syllabification differences between Spanish and English, the English dominant group will not likely show the an effect of syllable structure in contacting the lexicon which has been shown in previous studies. The third group and possibly the most interesting bilingual population being studied is the balanced bilingual group. Given that this group has learned both Spanish and English simultaneously, it is hard to predict whether or not this group will use a syllabic key to lexical access. Nonetheless, either result for this group provides interesting results. If they follow the pattern of the Spanish-dominant group, it will be the first finding of its kind in pairings where English was one of the two languages of the bilinguals to reveal a syllabic effect. If the balanced bilingual group patterns alongside the English-dominant group, then further evidence could be drawn for the access of to one and only one strategy for lexical access. In this case, the syllabic strategy could be argued to be a default strategy for language users until it proves ineffective in which case listeners abandon it for segmental access to the lexicon. In experiment 1, testing this effect at the subconscious level will be done through masked priming \citep{Forster1984-sf}. 
Transitioning from the findings of Chapter 2, which investigates the syllable’s role in speech segmentation, this investigation goes into the role that the syllable plays in lexical access. The study will address the following questions:
\begin{enumerate}
\item{Is the structure of the syllable represented in the organization of the mental lexicon?}
\item{If there is evidence of syllable structures in the mental lexicon, what are the differences, if any that exist between the three types of bilingual populations?}
\item{Using a masked prime—subconscious processing—and visually presented target; Is the phonological syllable activated as a means to kick off a lexical search?}
\end{enumerate}

%\textbf{DELETE ME The third section titled, “The Present Experiment” or “The Present Research”, follows and contains experimental descriptions and how they address the questions being asked. For most articles, keep your introduction under 10 pages}

%----------------------------------------------------------------------------------------
%	SECTION 2
%----------------------------------------------------------------------------------------

\section{Methods}

%Write this section FIRST. It is the section that describes how the research was conducted. A good one shows how well thought out the experiment design is and allows other researchers to easily replicate it. This section also follows a formula:

%-----------------------------------
%	SUBSECTION 2-1
%-----------------------------------
\subsection{Participants}

The participants will be recruited from the same three populations as those represented in chapter 2, but no participant that completed experiments for chapter 2 or the quasi pilot study will participate in the experiments of chapter 3. The participants will make up three distinct populations of Spanish–English bilinguals, which will remain the same throughout all the experiments carried out in the following three chapters. The first group, henceforth the Spanish-dominant group, will consist of participants who grew up in Mexico speaking only Spanish and later learned their L2 English after the age of %INSERT AGE%. 
The participants in this group will be recruited from Guanajuato, Mexico and will have received their formal grade-level schooling in Spanish. The second group, the English-dominant group, will consist of participants who grew up in Arizona speaking only English and later learning their L2 Spanish after the age of %INSERT AGE%. 
The participants for this group will be recruited from Tucson, Arizona and will have received their formal grade-level schooling in English. The last group, the Balanced-bilingual group, will consist of participants that were exposed to both Spanish and English from infancy. These participants will have acquired both Spanish and English simultaneously, but unlike the Spanish-dominant group, they will have received their formal grade-level schooling in English. The Balanced-bilingual group participants will also be recruited from Tucson, Arizona. The participants will be placed into their appropriate group based off of the combined results of BLP and LexTALE tasks. In addition, based off of responses in the BLP, participants who report proficiency in a language other than English and Spanish will be removed from the data analysis. Due to the likely fact that several participants will necessarily be excluded from the analysis, 30 participants will be recruited for each population in each chapter. If more than twenty participants remain eligible after removing participants who do not fit the population criteria, a random sampling of 20 participants will be selected from eligible pool of participants.

%\textbf{alternate participant description to work with}
The participants will make up three distinct populations of Spanish–English bilinguals, which will remain the same throughout all the experiments carried out in the following three chapters. The first group, henceforth the Spanish-dominant group, will consist of participants who grew up in Mexico speaking only Spanish and later learned their L2 English after the age of %INSERT AGE%. 
The participants in this group will be recruited from Guanajuato, Mexico and will have received their formal grade-level schooling in Spanish. The second group, the English-dominant group, will consist of participants who grew up in Arizona speaking only English and later learning their L2 Spanish after the age of %INSERT AGE%. 
The participants for this group will be recruited from Tucson, Arizona and will have received their formal grade-level schooling in English. The last group, the Balanced-bilingual group, will consist of participants that were exposed to both Spanish and English from %INSERT AGE%. 
These participants will have acquired both Spanish and English simultaneously, but unlike the Spanish-dominant group, they will have received their formal grade-level schooling in English. The Balanced-bilingual %Don't use term balanced bilingual
group participants will also be recruited from Tucson, Arizona. The participants will be placed into their appropriate group based off of the combined results of BLP and LexTALE tasks. In addition, based off of responses in the BLP, participants who report proficiency in a language other than English and Spanish will be removed from the data analysis. Due to the likely fact that several participants will necessarily be excluded from the analysis, 30 participants will be recruited for each population in each chapter. If more than twenty participants remain eligible after removing participants who do not fit the population criteria, a random sampling of 20 participants will be selected from eligible pool of participants. The participants will be recruited from the same three populations as those represented in chapters 2 and 3, but no participant that completed experiments for chapters 2 or 3 or the quasi pilot study will participate in the experiments of chapter 4. 

%\textbf{DELETE ME Give accurate descriptions of participant groups, how they were classified, number of participants, etc.}
%-----------------------------------
%	SUBSECTION 2-2
%-----------------------------------
\subsection{Instrumentation}

LexTALE and LexTALE-Esp are tasks used to correlate vocabulary knowledge and language proficiency in English and Spanish respectively \citep{Izura2014-yw, Lemhofer2012-hz}. The LexTALE-Esp has also been shown to discriminate well between highly proficient Spanish speaking participants with different language dominances \citep{Ferre2017-lp}. The LexTALE and LexTALE-Esp tests will be used in order to group participants into the appropriate bilingual population—Spanish–dominant bilinguals of English (L2), English–dominant bilinguals of Spanish (L2) or Balanced bilinguals (exposed to both English and Spanish since early childhood). Again, the LexTALE is available online and has been designed to run in PRAAT, Matlab and Presentations, but the participants will take it using PsychoPy at the experiment testing location immediately following the experimental tasks. 
A second instrument used in this study is the Bilingual Language Profile (BLP) is survey based assessment tool for determining language dominance \citep{Birdsong2012-wd}. It assesses language history, use, proficiency and attitudes of participants in less than 10 minutes. This assessment tool is commonly used in language studies with a focus on bilingualism and is available for free under the creative commons license. This tool allows researchers to collect information about participants demographics—name, age, sex, place of residence and educational background. It also allows for the participant to talk about the language history, use, proficiency and attitudes. For the current dissertation project, this survey will be administered through Google Forms as designed by the creators of the BLP while they are in the experiment testing location. Since no modification will be made to the BLP survey, the scores will be automatically calculated as designed by the creators. In all cases, the BLP survey will be completed after the experimental trials to avoid any confounding factors of language activation, Spanish or English, the participants will be able to choose whether they would like to take the survey with Spanish or English instructions. The participants will be recruited from the same three populations as those represented in chapters 2, but no participant that completed experiments for chapters 2 or the quasi pilot study will participate in the experiments of chapter 3.

%\textbf{DELETE ME Any special instrumentation could be included here (I am not sure that mine deserves a devoted section to instrumentation.}


%-----------------------------------
%	SUBSECTION 2-3
%-----------------------------------

\subsection{Design}

%\textbf{This paragraph needs major revision, no longer using Álvarez}
Following the methods used to disentangle phonological and orthographic syllables by Álvarez et al. \parencite*{Alvarez2004-nd}, 80 Spanish words that contain unstressed initial syllables will be chosen as targets for the lexical decision task. 40 of the words began with a CV syllable structure while 40 words begin with a CVC syllable structure. The CVC masked primes will be composed of syllables where half match the first three phonemes in orthography and phonology (bal-BAL.CÓN) and the other half match in the phonology of the first three phonemes, but differed in orthography (val-BAL.CÓN). The CV masked will mimic the pairings of the CVC masked primes with the exception that only the first two phonemes are considered—ba-BAL.CÓN and va-BAL.CÓN. In addition to the 80 real Spanish words, 80 nonce words will be created that all abide by orthographic constraints of Spanish.

%\textbf{DELETE ME Give an accurate idea of how the overall project was designed, what previous studies is it based off, theoretical principles, etc. DO NOT write about what participants do in this section.}


%-----------------------------------
%	SUBSECTION 2-4
%-----------------------------------

\subsection{Procedure}

Participants were seated in front a laptop computer and given a button box in which to respond with following the entering of demographic information at the beginning of the experiment. Once the participant was ready to begin, the first task they completed was the LexTALE-Esp---the Spanish version of the vocabulary test. This served to place the bilingual participants in Spanish mode before proceeding the lexical decision task.

For the lexical decision task, all instructions were given in Spanish regardless of the response they gave to "preferred language" in the demographic section. The instructions told participants that they would see a series of pound signs on the screen followed by a word. When the word appeared on the screen, the participants were instructed to respond as quickly and accurately as possible to whether or not the word was a real Spanish word. The overall experimental trial presented a forward mask "\#\#\#\#\#\#" of six pound signs for 500 milliseconds. The forward mask was replaced by a prime revealing 2 or 3 lowercase letters followed by either 4 or 3 pound signs, respectively. The prime remained on the screen for 116 milliseconds before being replaced by the actual word in all capital letters on which the participants had to make a lexical decision. The prime always shared the same sequence of letters as the beginning of the word. The word remained on the screen for 2000 milliseconds or the participant entered a response before moving on to the next trial. In total, there were 320 trials where half the words were real Spanish words and the other half were noncewords that followed all Spanish orthographic rules. For trials with real words, half of the words began had a word initial stressed syllable while the other half did not. The 20 of the words containing a word initial stressed syllable began with a CV syllable structure---where C is a consonant and V is a vowel---while the remaining 20 words began with a CVC syllable structure. Half of both the syllable initial stressed words and syllable initial unstressed words, the sequence of letters revealed by the prime matched the initial syllable structure of the presented word. The other half of the words, the sequence of letters revealed in the prime did not match the initial syllable structure of the word. In some cases, the prime revealed one letter too many---the third letter would be represented as the first letter of the second syllable of the word---while in other cases, the prime did not reveal the entire first syllable---the third letter of the initial CVC syllable was not revealed until the entire word appeared on the screen. The nonword trials followed the same pattern as the real word trials.

Following the lexical decision task, participants then took part in the LexTALE task---the English version of a vocabulary test.

Following the LexTALE task, participants also completed the bilingual language profile (BLP).

%\textbf{this paragraph also needs major revisions}
As in experiment 1 of chapter 2, There will necessarily be four different versions of the current experiment for counter-balancing purposes, which is accomplished by balancing across participant. For example, the word “balcón” (balcony) will be the target one time for each participant who each receive it under a different condition. Condition 1 would prime BALCÓN with bal, which matches the initial syllable structure, the orthography of first three phonemes and the first three phonological segments. Condition 2 will prime BALCÓN with val, which matches the initial syllable structure, the first three phonological segments, but does not share the first three graphemes. Condition 3 uses the prime ba for the target BALCÓN, which does not match the initial syllable structure of the target, but shares the first two graphemes and phonological segments. The final condition, Condition 4, will use the prime va for the target BALCÓN, which does not match the initial syllable structure of the target, nor the first two graphemes, but does match the first two phonological segments. “Balcón” illustrates target words with a CVC initial syllable, but the experiment will also have targets with CV initial syllable structures. For example, balón (ball) would have four separate conditions: ba-BALÓN, va-BALÓN, bal-BALÓN and val-BALÓN.
Each participant will be randomly assigned to one of the four versions of the experiment, which means that no participant will see any target word more than once during the experiment. For example, if participant 1 is assigned to a version of the experiment where condition 1 (bal-BALCÓN) is presented to them then participant 1 would not see the target—BALCÓN—in conditions 2, 3 or 4 in their experiment. Each version of the experiment presents 40 CV and 40 CVC trials where half of each type of syllable structure contains a match between the syllable structure of the prime-TARGET while the other half are mismatched. In each version of the experiment half of the CV and CVC trials share orthographic and phonological segments while the other half only share phonological segments. In order to control for hand dominance, half of the participants will indicate a real Spanish word with the left button on the response box while the other half of participants will indicate a nonce word with the left button.

%\textbf{this paragraph should be combined with the first one of this section}
Participants will be instructed that a word will appear in the center of the screen and that they will need to indicate whether the word is a real word of Spanish as quickly and accurately as possible. Participants will be instructed to press one button on the response box if the word is a real Spanish word and will press the other button on the response box if the word is not a real Spanish word. Participants in experiment 1 will complete a visual word recognition task, which will require a lexical decision under the following experimental conditions. At the beginning of trial a series of six pounds signs (\#\#\#\#\#\#) will appear as a forward mask in the center of the screen for 500 milliseconds. The forward mask will immediately be followed by a masked lexical fragment—the prime—for 116 milliseconds which will be immediately followed by a real or nonce word—the target. 116 milliseconds for the forward mask was chosen following the findings of Carreiras and Perea \parencite*{Carreiras2002-mp} who showed that this timeframe was sufficient to show the syllabic effect and extending it to 166 milliseconds resulted in an accuracy–speed trade off. The trial will end once the participant has made their response which will clear the screen for one second before the forward mask of the following trial appears. The participants will see 16 practice trials that match the procedure described above for the experimental trials. Following the practice trials and the 160 experimental trials, the participants will complete the LexTALE and LexTALE-Esp vocabulary tasks using PsychoPy. The last task that participants will complete in their one laboratory visit is the bilingual language profile (BLP) in a Google Form developed by the creators.



%\textbf{DELETE ME The first refer to mainly scope and size of research project while the procedure section is usually much more in-depth. Here, you connect your procedures to those in already published articles when possible and given detailed descriptions of classifications and scales used in procedure. Essentially ensure the reader does not suspect anything is being hidden and the researcher is honest. Do not repeat any unnecessary information in subsequent experiments of the paper.}


%----------------------------------------------------------------------------------------
%	SECTION 3
%----------------------------------------------------------------------------------------

\section{Results}

Participants who have less than a %INSERT CUTOFF PERCENTAGE% 
percent success rate on real word trials will be removed from the data %INSERT NUMBER OF PARTICIPANTS%. 
For the latency data, incorrect responses for remaining participants will be removed. Then a second pass will remove any correct individual participant responses under 300 milliseconds and over 2000 milliseconds. Finally, correct responses with latency data more than two standard deviations away from that participants average will also be removed %INSERT NUMBER AND PERCENTAGE OF TRIALS.


Columns needed include:
\begin{enumerate}
\item{participant}
\item{correct answer}
\item{condition}
\item{response}
\item{correct response}
\item{response time}
\end{enumerate}

%linear regression/ANOVA
% reaction time
%Accuracy
%AorB logistic regression
		CV	   |	CVC	
			
-----------------------------------------

CV	|     match	   |  mismatch |

-----------------------------------------

CVC |  mismatch |     match    |

-----------------------------------------
%\textbf{DELETE ME describe the analyses the researcher has done, but do not overload. Instead of creating a laundry list of statistics, create the story you want to tell using only the statistics that are related to addressing your problem. For each task, review the hypothesis, give the statistics and say what the result of the test means. Do not discuss the findings until you reach the discussion session. This is where tables and figures can help keep the paper looking clean and crisp instead of cluttered unorganized statistical test lists that are hard to follow. Figures show patterns while tables give details.}


%----------------------------------------------------------------------------------------
%	SECTION 4
%----------------------------------------------------------------------------------------

\section{Discussion}

%split into several small sections following each individual experiment’s results sections where applicable. The General Discussion steps back and begins with an overview of the problem and then the findings. A general rule of thumb is to keep this section shorter than the introduction. Only give limitation directly related to the current study, not the general limitation of the research or the field as a whole and be sure to give a good reason for why these limitations are not as bad as they sound on the surface.


%----------------------------------------------------------------------------------------
%	SECTION 5
%----------------------------------------------------------------------------------------

\section{References}

%Insert references for this chapter here.





% Chapter 5 with LaTeX code only

%----------------------------------------------------------------------------------------

%----------------------------------------------------------------------------------------


%----------------------------------------------------------------------------------------
% Load statistics in memory from separate file
%<<content5, child='stat.rnw'>>=
@
%----------------------------------------------------------------------------------------


\chapter{Conclusion} % Main chapter title

\label{Chapter5} % Change X to a consecutive number; for referencing this chapter elsewhere, use \ref{Chapter5}

%----------------------------------------------------------------------------------------
%	SECTION 1
%----------------------------------------------------------------------------------------

\section{Main Section 1}

This proposal has provided a basic background and introduction to the project that will be extended in the formal dissertation. This project will encompass the more in-depth the background knowledge of the research related to the syllable’s role in speech processing. The dissertation will not only attempt to replicate previous findings with a different speaker population, but also fill in some of the research veins that have yet to be investigated. For example, French–English early bilinguals have been tested in a word segmentation paradigm, but a comparative group, Spanish–English early bilinguals, have yet to investigated. Given that Spanish and French share many similarities—i.e. Syllable timed and clear, unambiguous syllable boundaries—replicating the results of the French–English bilinguals with this new population would give additional support for a syllable-based segmentation strategy. In addition, the research proposed for this dissertation would give new insight on late bilinguals—speakers of English who learned Spanish as adults—which will elucidate several uses of the syllable. First, it will provide research data on different bilingual speaker population, and secondly, it will allow for a comparison of degrees of bilingualism. In other words, it will shed light on whether or not native speaker of a language such as English that is known not to implement a syllable-based segmentation strategy ever learn to utilize a syllable-based segmentation strategy when becoming bilingual in a language where native speakers do make productive use of the strategy.
The second main thread of research contained in the forthcoming dissertation will be on the syllable’s role in lexical access. Up to this point, it has mainly been tested in the visual priming paradigm, but Italian has also been studied under a cross-modal fragment priming paradigm. The dissertation will add to this literature by again investigating whether or not the path to lexical access reflects a syllabic representation as well as if the degree of bilingualism is a pertinent factor of its usefulness. In addition, it will expand our knowledge base by replicating the the Italian results in Spanish. 
Both lines of research will lean towards a better understanding how the critical age hypothesis and the role of schooling may relate to speech processing. For example, if a difference is found between the early bilingual group and the late bilingual groups, it would provide evidence that once a person has passed through puberty, learning new segmentation or lexical access strategies are no longer acquirable for speakers learning a language as an adult. If on the other hand, no difference is found between the early bilingual group and late bilingual groups, it would provide evidence that these strategies are still available and able to be learned by adult language learners. A third possibility exists in that a difference will be found between late learners of English bilingual population and the other two bilingual groups—balanced bilinguals and late learners of Spanish bilinguals. If this were the case, then a case could be drawn for the role of schooling in which syllables are taught as a means of breaking words into smaller chunks for Spanish native speakers by Spanish native speakers as the reason for the development of the syllable-based processing strategy. This case can only be built when the early bilingual group consists of speakers who grew up using both Spanish and English, but received formal education only in Spanish.
No results have been report as part of this proposal since no data has been collected under the provisions of the Internal Review Board (IRB) at the University of Arizona. However, the final dissertation project will collect data from participants through the use of several psycholinguistic methodologies. In chapter 2 of the dissertation, a syllable monitoring task will be used in the auditory modality to determine whether or not the syllable has some representation in the minds of Spanish speakers that can be utilized to segment spoken speech. A second task will also be used in the second chapter, which will fall under a syllabification task. This task will ask participants to indicate the first syllable of the word in a two-choice forced decision task. In chapter 3, a visual priming experiment will be used in combination with a lexical decision task in order to determine whether or not the syllable plays a role in the facilitation of lexical retrieval by bilingual speakers of Spanish. Finally in chapter 4, a cross-modal auditory fragment priming in combination with a visually display word lexical decision task will be used to further test the role of the syllable in lexical access of bilingual Spanish–English speakers. In all experiments, reaction times captured in the responses to monitoring and lexical decisions tasks will be the variable of interest in determining whether or not the syllable has a facilitatory effect for word segmentation or lexical access.


%-----------------------------------
%	SUBSECTION 1
%-----------------------------------
%\subsection{Subsection 1}

%may not need

%-----------------------------------
%	SUBSECTION 2
%-----------------------------------

%\subsection{Subsection 2}
%probably need something here

%----------------------------------------------------------------------------------------
%	SECTION 2
%----------------------------------------------------------------------------------------

%\section{Main Section 2}

%may not need

%----------------------------------------------------------------------------------------
%	SECTION 3
%----------------------------------------------------------------------------------------

%\section{Main Section 3}

%may not need

%-----------------------------------
%	SUBSECTION 1
%-----------------------------------
%\subsection{Subsection 1}

%May not need

%-----------------------------------
%	SUBSECTION 2
%-----------------------------------

%\subsection{Subsection 2}
%May not need
%----------------------------------------------------------------------------------------
%	SECTION 4
%----------------------------------------------------------------------------------------

%\section{Main Section 4}

%May not need





% Chapter Template

\chapter{Chapter Title Here} % Main chapter title

\label{ChapterX} % Change X to a consecutive number; for referencing this chapter elsewhere, use \ref{ChapterX}

%----------------------------------------------------------------------------------------
%	SECTION 1
%----------------------------------------------------------------------------------------

\section{Main Section 1}

Lorem ipsum dolor sit amet, consectetur adipiscing elit. Aliquam ultricies lacinia euismod. Nam tempus risus in dolor rhoncus in interdum enim tincidunt. Donec vel nunc neque. In condimentum ullamcorper quam non consequat. Fusce sagittis tempor feugiat. Fusce magna erat, molestie eu convallis ut, tempus sed arcu. Quisque molestie, ante a tincidunt ullamcorper, sapien enim dignissim lacus, in semper nibh erat lobortis purus. Integer dapibus ligula ac risus convallis pellentesque.

%-----------------------------------
%	SUBSECTION 1
%-----------------------------------
\subsection{Subsection 1}

Nunc posuere quam at lectus tristique eu ultrices augue venenatis. Vestibulum ante ipsum primis in faucibus orci luctus et ultrices posuere cubilia Curae; Aliquam erat volutpat. Vivamus sodales tortor eget quam adipiscing in vulputate ante ullamcorper. Sed eros ante, lacinia et sollicitudin et, aliquam sit amet augue. In hac habitasse platea dictumst.

%-----------------------------------
%	SUBSECTION 2
%-----------------------------------

\subsection{Subsection 2}
Morbi rutrum odio eget arcu adipiscing sodales. Aenean et purus a est pulvinar pellentesque. Cras in elit neque, quis varius elit. Phasellus fringilla, nibh eu tempus venenatis, dolor elit posuere quam, quis adipiscing urna leo nec orci. Sed nec nulla auctor odio aliquet consequat. Ut nec nulla in ante ullamcorper aliquam at sed dolor. Phasellus fermentum magna in augue gravida cursus. Cras sed pretium lorem. Pellentesque eget ornare odio. Proin accumsan, massa viverra cursus pharetra, ipsum nisi lobortis velit, a malesuada dolor lorem eu neque.

%----------------------------------------------------------------------------------------
%	SECTION 2
%----------------------------------------------------------------------------------------

\section{Main Section 2}

Sed ullamcorper quam eu nisl interdum at interdum enim egestas. Aliquam placerat justo sed lectus lobortis ut porta nisl porttitor. Vestibulum mi dolor, lacinia molestie gravida at, tempus vitae ligula. Donec eget quam sapien, in viverra eros. Donec pellentesque justo a massa fringilla non vestibulum metus vestibulum. Vestibulum in orci quis felis tempor lacinia. Vivamus ornare ultrices facilisis. Ut hendrerit volutpat vulputate. Morbi condimentum venenatis augue, id porta ipsum vulputate in. Curabitur luctus tempus justo. Vestibulum risus lectus, adipiscing nec condimentum quis, condimentum nec nisl. Aliquam dictum sagittis velit sed iaculis. Morbi tristique augue sit amet nulla pulvinar id facilisis ligula mollis. Nam elit libero, tincidunt ut aliquam at, molestie in quam. Aenean rhoncus vehicula hendrerit.

% Chapter 1

\chapter{How to use this template} % Main chapter title

\label{Chapter6} % For referencing the chapter elsewhere, use \ref{Chapter1} 

%----------------------------------------------------------------------------------------

% Define some commands to keep the formatting separated from the content 
\newcommand{\keyword}[1]{\textbf{#1}}
\newcommand{\tabhead}[1]{\textbf{#1}}
\newcommand{\code}[1]{\texttt{#1}}
\newcommand{\file}[1]{\texttt{\bfseries#1}}
\newcommand{\option}[1]{\texttt{\itshape#1}}

%----------------------------------------------------------------------------------------

\section{Welcome and Thank You}
Welcome to this \LaTeX{} Thesis Template, a beautiful and easy to use template for writing a thesis using the \LaTeX{} typesetting system.

If you are writing a thesis (or will be in the future) and its subject is technical or mathematical (though it doesn't have to be), then creating it in \LaTeX{} is highly recommended as a way to make sure you can just get down to the essential writing without having to worry over formatting or wasting time arguing with your word processor.

\LaTeX{} is easily able to professionally typeset documents that run to hundreds or thousands of pages long. With simple mark-up commands, it automatically sets out the table of contents, margins, page headers and footers and keeps the formatting consistent and beautiful. One of its main strengths is the way it can easily typeset mathematics, even \emph{heavy} mathematics. Even if those equations are the most horribly twisted and most difficult mathematical problems that can only be solved on a super-computer, you can at least count on \LaTeX{} to make them look stunning.

%----------------------------------------------------------------------------------------

\section{Learning \LaTeX{}}

\LaTeX{} is not a \textsc{wysiwyg} (What You See is What You Get) program, unlike word processors such as Microsoft Word or Apple's Pages. Instead, a document written for \LaTeX{} is actually a simple, plain text file that contains \emph{no formatting}. You tell \LaTeX{} how you want the formatting in the finished document by writing in simple commands amongst the text, for example, if I want to use \emph{italic text for emphasis}, I write the \verb|\emph{text}| command and put the text I want in italics in between the curly braces. This means that \LaTeX{} is a \enquote{mark-up} language, very much like HTML.

\subsection{A (not so short) Introduction to \LaTeX{}}

If you are new to \LaTeX{}, there is a very good eBook -- freely available online as a PDF file -- called, \enquote{The Not So Short Introduction to \LaTeX{}}. The book's title is typically shortened to just \emph{lshort}. You can download the latest version (as it is occasionally updated) from here:
\url{http://www.ctan.org/tex-archive/info/lshort/english/lshort.pdf}

It is also available in several other languages. Find yours from the list on this page: \url{http://www.ctan.org/tex-archive/info/lshort/}

It is recommended to take a little time out to learn how to use \LaTeX{} by creating several, small `test' documents, or having a close look at several templates on:\\ 
\url{http://www.LaTeXTemplates.com}\\ 
Making the effort now means you're not stuck learning the system when what you \emph{really} need to be doing is writing your thesis.

\subsection{A Short Math Guide for \LaTeX{}}

If you are writing a technical or mathematical thesis, then you may want to read the document by the AMS (American Mathematical Society) called, \enquote{A Short Math Guide for \LaTeX{}}. It can be found online here:
\url{http://www.ams.org/tex/amslatex.html}
under the \enquote{Additional Documentation} section towards the bottom of the page.

\subsection{Common \LaTeX{} Math Symbols}
There are a multitude of mathematical symbols available for \LaTeX{} and it would take a great effort to learn the commands for them all. The most common ones you are likely to use are shown on this page:
\url{http://www.sunilpatel.co.uk/latex-type/latex-math-symbols/}

You can use this page as a reference or crib sheet, the symbols are rendered as large, high quality images so you can quickly find the \LaTeX{} command for the symbol you need.

\subsection{\LaTeX{} on a Mac}
 
The \LaTeX{} distribution is available for many systems including Windows, Linux and Mac OS X. The package for OS X is called MacTeX and it contains all the applications you need -- bundled together and pre-customized -- for a fully working \LaTeX{} environment and work flow.
 
MacTeX includes a custom dedicated \LaTeX{} editor called TeXShop for writing your `\file{.tex}' files and BibDesk: a program to manage your references and create your bibliography section just as easily as managing songs and creating playlists in iTunes.

%----------------------------------------------------------------------------------------

\section{Getting Started with this Template}

If you are familiar with \LaTeX{}, then you should explore the directory structure of the template and then proceed to place your own information into the \emph{THESIS INFORMATION} block of the \file{main.tex} file. You can then modify the rest of this file to your unique specifications based on your degree/university. Section \ref{FillingFile} on page \pageref{FillingFile} will help you do this. Make sure you also read section \ref{ThesisConventions} about thesis conventions to get the most out of this template.

If you are new to \LaTeX{} it is recommended that you carry on reading through the rest of the information in this document.

Before you begin using this template you should ensure that its style complies with the thesis style guidelines imposed by your institution. In most cases this template style and layout will be suitable. If it is not, it may only require a small change to bring the template in line with your institution's recommendations. These modifications will need to be done on the \file{MastersDoctoralThesis.cls} file.

\subsection{About this Template}

This \LaTeX{} Thesis Template is originally based and created around a \LaTeX{} style file created by Steve R.\ Gunn from the University of Southampton (UK), department of Electronics and Computer Science. You can find his original thesis style file at his site, here:
\url{http://www.ecs.soton.ac.uk/~srg/softwaretools/document/templates/}

Steve's \file{ecsthesis.cls} was then taken by Sunil Patel who modified it by creating a skeleton framework and folder structure to place the thesis files in. The resulting template can be found on Sunil's site here:
\url{http://www.sunilpatel.co.uk/thesis-template}

Sunil's template was made available through \url{http://www.LaTeXTemplates.com} where it was modified many times based on user requests and questions. Version 2.0 and onwards of this template represents a major modification to Sunil's template and is, in fact, hardly recognisable. The work to make version 2.0 possible was carried out by \href{mailto:vel@latextemplates.com}{Vel} and Johannes Böttcher.

%----------------------------------------------------------------------------------------

\section{What this Template Includes}

\subsection{Folders}

This template comes as a single zip file that expands out to several files and folders. The folder names are mostly self-explanatory:

\keyword{Appendices} -- this is the folder where you put the appendices. Each appendix should go into its own separate \file{.tex} file. An example and template are included in the directory.

\keyword{Chapters} -- this is the folder where you put the thesis chapters. A thesis usually has about six chapters, though there is no hard rule on this. Each chapter should go in its own separate \file{.tex} file and they can be split as:
\begin{itemize}
\item Chapter 1: Introduction to the thesis topic
\item Chapter 2: Background information and theory
\item Chapter 3: (Laboratory) experimental setup
\item Chapter 4: Details of experiment 1
\item Chapter 5: Details of experiment 2
\item Chapter 6: Discussion of the experimental results
\item Chapter 7: Conclusion and future directions
\end{itemize}
This chapter layout is specialised for the experimental sciences, your discipline may be different.

\keyword{Figures} -- this folder contains all figures for the thesis. These are the final images that will go into the thesis document.

\subsection{Files}

Included are also several files, most of them are plain text and you can see their contents in a text editor. After initial compilation, you will see that more auxiliary files are created by \LaTeX{} or BibTeX and which you don't need to delete or worry about:

\keyword{example.bib} -- this is an important file that contains all the bibliographic information and references that you will be citing in the thesis for use with BibTeX. You can write it manually, but there are reference manager programs available that will create and manage it for you. Bibliographies in \LaTeX{} are a large subject and you may need to read about BibTeX before starting with this. Many modern reference managers will allow you to export your references in BibTeX format which greatly eases the amount of work you have to do.

\keyword{MastersDoctoralThesis.cls} -- this is an important file. It is the class file that tells \LaTeX{} how to format the thesis. 

\keyword{main.pdf} -- this is your beautifully typeset thesis (in the PDF file format) created by \LaTeX{}. It is supplied in the PDF with the template and after you compile the template you should get an identical version.

\keyword{main.tex} -- this is an important file. This is the file that you tell \LaTeX{} to compile to produce your thesis as a PDF file. It contains the framework and constructs that tell \LaTeX{} how to layout the thesis. It is heavily commented so you can read exactly what each line of code does and why it is there. After you put your own information into the \emph{THESIS INFORMATION} block -- you have now started your thesis!

Files that are \emph{not} included, but are created by \LaTeX{} as auxiliary files include:

\keyword{main.aux} -- this is an auxiliary file generated by \LaTeX{}, if it is deleted \LaTeX{} simply regenerates it when you run the main \file{.tex} file.

\keyword{main.bbl} -- this is an auxiliary file generated by BibTeX, if it is deleted, BibTeX simply regenerates it when you run the \file{main.aux} file. Whereas the \file{.bib} file contains all the references you have, this \file{.bbl} file contains the references you have actually cited in the thesis and is used to build the bibliography section of the thesis.

\keyword{main.blg} -- this is an auxiliary file generated by BibTeX, if it is deleted BibTeX simply regenerates it when you run the main \file{.aux} file.

\keyword{main.lof} -- this is an auxiliary file generated by \LaTeX{}, if it is deleted \LaTeX{} simply regenerates it when you run the main \file{.tex} file. It tells \LaTeX{} how to build the \emph{List of Figures} section.

\keyword{main.log} -- this is an auxiliary file generated by \LaTeX{}, if it is deleted \LaTeX{} simply regenerates it when you run the main \file{.tex} file. It contains messages from \LaTeX{}, if you receive errors and warnings from \LaTeX{}, they will be in this \file{.log} file.

\keyword{main.lot} -- this is an auxiliary file generated by \LaTeX{}, if it is deleted \LaTeX{} simply regenerates it when you run the main \file{.tex} file. It tells \LaTeX{} how to build the \emph{List of Tables} section.

\keyword{main.out} -- this is an auxiliary file generated by \LaTeX{}, if it is deleted \LaTeX{} simply regenerates it when you run the main \file{.tex} file.

So from this long list, only the files with the \file{.bib}, \file{.cls} and \file{.tex} extensions are the most important ones. The other auxiliary files can be ignored or deleted as \LaTeX{} and BibTeX will regenerate them.

%----------------------------------------------------------------------------------------

\section{Filling in Your Information in the \file{main.tex} File}\label{FillingFile}

You will need to personalise the thesis template and make it your own by filling in your own information. This is done by editing the \file{main.tex} file in a text editor or your favourite LaTeX environment.

Open the file and scroll down to the third large block titled \emph{THESIS INFORMATION} where you can see the entries for \emph{University Name}, \emph{Department Name}, etc \ldots

Fill out the information about yourself, your group and institution. You can also insert web links, if you do, make sure you use the full URL, including the \code{http://} for this. If you don't want these to be linked, simply remove the \verb|\href{url}{name}| and only leave the name.

When you have done this, save the file and recompile \code{main.tex}. All the information you filled in should now be in the PDF, complete with web links. You can now begin your thesis proper!

%----------------------------------------------------------------------------------------

\section{The \code{main.tex} File Explained}

The \file{main.tex} file contains the structure of the thesis. There are plenty of written comments that explain what pages, sections and formatting the \LaTeX{} code is creating. Each major document element is divided into commented blocks with titles in all capitals to make it obvious what the following bit of code is doing. Initially there seems to be a lot of \LaTeX{} code, but this is all formatting, and it has all been taken care of so you don't have to do it.

Begin by checking that your information on the title page is correct. For the thesis declaration, your institution may insist on something different than the text given. If this is the case, just replace what you see with what is required in the \emph{DECLARATION PAGE} block.

Then comes a page which contains a funny quote. You can put your own, or quote your favourite scientist, author, person, and so on. Make sure to put the name of the person who you took the quote from.

Following this is the abstract page which summarises your work in a condensed way and can almost be used as a standalone document to describe what you have done. The text you write will cause the heading to move up so don't worry about running out of space.

Next come the acknowledgements. On this page, write about all the people who you wish to thank (not forgetting parents, partners and your advisor/supervisor).

The contents pages, list of figures and tables are all taken care of for you and do not need to be manually created or edited. The next set of pages are more likely to be optional and can be deleted since they are for a more technical thesis: insert a list of abbreviations you have used in the thesis, then a list of the physical constants and numbers you refer to and finally, a list of mathematical symbols used in any formulae. Making the effort to fill these tables means the reader has a one-stop place to refer to instead of searching the internet and references to try and find out what you meant by certain abbreviations or symbols.

The list of symbols is split into the Roman and Greek alphabets. Whereas the abbreviations and symbols ought to be listed in alphabetical order (and this is \emph{not} done automatically for you) the list of physical constants should be grouped into similar themes.

The next page contains a one line dedication. Who will you dedicate your thesis to?

Finally, there is the block where the chapters are included. Uncomment the lines (delete the \code{\%} character) as you write the chapters. Each chapter should be written in its own file and put into the \emph{Chapters} folder and named \file{Chapter1}, \file{Chapter2}, etc\ldots Similarly for the appendices, uncomment the lines as you need them. Each appendix should go into its own file and placed in the \emph{Appendices} folder.

After the preamble, chapters and appendices finally comes the bibliography. The bibliography style (called \option{authoryear}) is used for the bibliography and is a fully featured style that will even include links to where the referenced paper can be found online. Do not underestimate how grateful your reader will be to find that a reference to a paper is just a click away. Of course, this relies on you putting the URL information into the BibTeX file in the first place.

%----------------------------------------------------------------------------------------

\section{Thesis Features and Conventions}\label{ThesisConventions}

To get the best out of this template, there are a few conventions that you may want to follow.

One of the most important (and most difficult) things to keep track of in such a long document as a thesis is consistency. Using certain conventions and ways of doing things (such as using a Todo list) makes the job easier. Of course, all of these are optional and you can adopt your own method.

\subsection{Printing Format}

This thesis template is designed for double sided printing (i.e. content on the front and back of pages) as most theses are printed and bound this way. Switching to one sided printing is as simple as uncommenting the \option{oneside} option of the \code{documentclass} command at the top of the \file{main.tex} file. You may then wish to adjust the margins to suit specifications from your institution.

The headers for the pages contain the page number on the outer side (so it is easy to flick through to the page you want) and the chapter name on the inner side.

The text is set to 11 point by default with single line spacing, again, you can tune the text size and spacing should you want or need to using the options at the very start of \file{main.tex}. The spacing can be changed similarly by replacing the \option{singlespacing} with \option{onehalfspacing} or \option{doublespacing}.

\subsection{Using US Letter Paper}

The paper size used in the template is A4, which is the standard size in Europe. If you are using this thesis template elsewhere and particularly in the United States, then you may have to change the A4 paper size to the US Letter size. This can be done in the margins settings section in \file{main.tex}.

Due to the differences in the paper size, the resulting margins may be different to what you like or require (as it is common for institutions to dictate certain margin sizes). If this is the case, then the margin sizes can be tweaked by modifying the values in the same block as where you set the paper size. Now your document should be set up for US Letter paper size with suitable margins.

\subsection{References}

The \code{biblatex} package is used to format the bibliography and inserts references such as this one \parencite{Reference1}. The options used in the \file{main.tex} file mean that the in-text citations of references are formatted with the author(s) listed with the date of the publication. Multiple references are separated by semicolons (e.g. \parencite{Reference2, Reference1}) and references with more than three authors only show the first author with \emph{et al.} indicating there are more authors (e.g. \parencite{Reference3}). This is done automatically for you. To see how you use references, have a look at the \file{Chapter1.tex} source file. Many reference managers allow you to simply drag the reference into the document as you type.

Scientific references should come \emph{before} the punctuation mark if there is one (such as a comma or period). The same goes for footnotes\footnote{Such as this footnote, here down at the bottom of the page.}. You can change this but the most important thing is to keep the convention consistent throughout the thesis. Footnotes themselves should be full, descriptive sentences (beginning with a capital letter and ending with a full stop). The APA6 states: \enquote{Footnote numbers should be superscripted, [...], following any punctuation mark except a dash.} The Chicago manual of style states: \enquote{A note number should be placed at the end of a sentence or clause. The number follows any punctuation mark except the dash, which it precedes. It follows a closing parenthesis.}

The bibliography is typeset with references listed in alphabetical order by the first author's last name. This is similar to the APA referencing style. To see how \LaTeX{} typesets the bibliography, have a look at the very end of this document (or just click on the reference number links in in-text citations).

\subsubsection{A Note on bibtex}

The bibtex backend used in the template by default does not correctly handle unicode character encoding (i.e. "international" characters). You may see a warning about this in the compilation log and, if your references contain unicode characters, they may not show up correctly or at all. The solution to this is to use the biber backend instead of the outdated bibtex backend. This is done by finding this in \file{main.tex}: \option{backend=bibtex} and changing it to \option{backend=biber}. You will then need to delete all auxiliary BibTeX files and navigate to the template directory in your terminal (command prompt). Once there, simply type \code{biber main} and biber will compile your bibliography. You can then compile \file{main.tex} as normal and your bibliography will be updated. An alternative is to set up your LaTeX editor to compile with biber instead of bibtex, see \href{http://tex.stackexchange.com/questions/154751/biblatex-with-biber-configuring-my-editor-to-avoid-undefined-citations/}{here} for how to do this for various editors.

\subsection{Tables}

Tables are an important way of displaying your results, below is an example table which was generated with this code:

{\small
\begin{verbatim}
\begin{table}
\caption{The effects of treatments X and Y on the four groups studied.}
\label{tab:treatments}
\centering
\begin{tabular}{l l l}
\toprule
\tabhead{Groups} & \tabhead{Treatment X} & \tabhead{Treatment Y} \\
\midrule
1 & 0.2 & 0.8\\
2 & 0.17 & 0.7\\
3 & 0.24 & 0.75\\
4 & 0.68 & 0.3\\
\bottomrule\\
\end{tabular}
\end{table}
\end{verbatim}
}

\begin{table}
\caption{The effects of treatments X and Y on the four groups studied.}
\label{tab:treatments}
\centering
\begin{tabular}{l l l}
\toprule
\tabhead{Groups} & \tabhead{Treatment X} & \tabhead{Treatment Y} \\
\midrule
1 & 0.2 & 0.8\\
2 & 0.17 & 0.7\\
3 & 0.24 & 0.75\\
4 & 0.68 & 0.3\\
\bottomrule\\
\end{tabular}
\end{table}

You can reference tables with \verb|\ref{<label>}| where the label is defined within the table environment. See \file{Chapter1.tex} for an example of the label and citation (e.g. Table~\ref{tab:treatments}).

\subsection{Figures}

There will hopefully be many figures in your thesis (that should be placed in the \emph{Figures} folder). The way to insert figures into your thesis is to use a code template like this:
\begin{verbatim}
\begin{figure}
\centering
\includegraphics{Figures/Electron}
\decoRule
\caption[An Electron]{An electron (artist's impression).}
\label{fig:Electron}
\end{figure}
\end{verbatim}
Also look in the source file. Putting this code into the source file produces the picture of the electron that you can see in the figure below.

\begin{figure}[th]
\centering
\includegraphics{Figures/Electron}
\decoRule
\caption[An Electron]{An electron (artist's impression).}
\label{fig:Electron}
\end{figure}

Sometimes figures don't always appear where you write them in the source. The placement depends on how much space there is on the page for the figure. Sometimes there is not enough room to fit a figure directly where it should go (in relation to the text) and so \LaTeX{} puts it at the top of the next page. Positioning figures is the job of \LaTeX{} and so you should only worry about making them look good!

Figures usually should have captions just in case you need to refer to them (such as in Figure~\ref{fig:Electron}). The \verb|\caption| command contains two parts, the first part, inside the square brackets is the title that will appear in the \emph{List of Figures}, and so should be short. The second part in the curly brackets should contain the longer and more descriptive caption text.

The \verb|\decoRule| command is optional and simply puts an aesthetic horizontal line below the image. If you do this for one image, do it for all of them.

\LaTeX{} is capable of using images in pdf, jpg and png format.

\subsection{Typesetting mathematics}

If your thesis is going to contain heavy mathematical content, be sure that \LaTeX{} will make it look beautiful, even though it won't be able to solve the equations for you.

The \enquote{Not So Short Introduction to \LaTeX} (available on \href{http://www.ctan.org/tex-archive/info/lshort/english/lshort.pdf}{CTAN}) should tell you everything you need to know for most cases of typesetting mathematics. If you need more information, a much more thorough mathematical guide is available from the AMS called, \enquote{A Short Math Guide to \LaTeX} and can be downloaded from:
\url{ftp://ftp.ams.org/pub/tex/doc/amsmath/short-math-guide.pdf}

There are many different \LaTeX{} symbols to remember, luckily you can find the most common symbols in \href{http://ctan.org/pkg/comprehensive}{The Comprehensive \LaTeX~Symbol List}.

You can write an equation, which is automatically given an equation number by \LaTeX{} like this:
\begin{verbatim}
\begin{equation}
E = mc^{2}
\label{eqn:Einstein}
\end{equation}
\end{verbatim}

This will produce Einstein's famous energy-matter equivalence equation:
\begin{equation}
E = mc^{2}
\label{eqn:Einstein}
\end{equation}

All equations you write (which are not in the middle of paragraph text) are automatically given equation numbers by \LaTeX{}. If you don't want a particular equation numbered, use the unnumbered form:
\begin{verbatim}
\[ a^{2}=4 \]
\end{verbatim}

%----------------------------------------------------------------------------------------

\section{Sectioning and Subsectioning}

You should break your thesis up into nice, bite-sized sections and subsections. \LaTeX{} automatically builds a table of Contents by looking at all the \verb|\chapter{}|, \verb|\section{}|  and \verb|\subsection{}| commands you write in the source.

The Table of Contents should only list the sections to three (3) levels. A \verb|chapter{}| is level zero (0). A \verb|\section{}| is level one (1) and so a \verb|\subsection{}| is level two (2). In your thesis it is likely that you will even use a \verb|subsubsection{}|, which is level three (3). The depth to which the Table of Contents is formatted is set within \file{MastersDoctoralThesis.cls}. If you need this changed, you can do it in \file{main.tex}.

%----------------------------------------------------------------------------------------

\section{In Closing}

You have reached the end of this mini-guide. You can now rename or overwrite this pdf file and begin writing your own \file{Chapter1.tex} and the rest of your thesis. The easy work of setting up the structure and framework has been taken care of for you. It's now your job to fill it out!

Good luck and have lots of fun!

\begin{flushright}
Guide written by ---\\
Sunil Patel: \href{http://www.sunilpatel.co.uk}{www.sunilpatel.co.uk}\\
Vel: \href{http://www.LaTeXTemplates.com}{LaTeXTemplates.com}
\end{flushright}


%----------------------------------------------------------------------------------------
%	THESIS CONTENT - APPENDICES
%----------------------------------------------------------------------------------------
%
% Appendices Requirements (if used)
% Follow content and precede References, unless used for manuscript/previous published article disserations
% are distinguished by capital letters, Appendix A – title, Appendix B – title, etc.
% must be correctly listed in Table of Contents
% Pages must be numbered in manner consistent with the rest of the document
% Permissions for including previously published articles are included
%
%If no appendices are used, comment out all active lines in this section
%\pagestyle{thesis}

\renewcommand\thechapter{}

\appendix % Cue to tell LaTeX that the following "chapters" are Appendices

% Include the appendices of the thesis as separate files from the Appendices folder
% Uncomment the lines as you write the Appendices


% Appendix A with LaTeX code only

%----------------------------------------------------------------------------------------

%----------------------------------------------------------------------------------------


\chapter{Wuggy Nonword Generator} % Main appendix title

\label{AppendixA} % Change X to a consecutive letter; for referencing this appendix elsewhere, use \ref{AppendixX}

This list of 80 critical items were entered into Wuggy.app as words and the second column gave Wuggy.app the correction Spanish syllabification of each critical word—generated from within the application. I then used the verify feature to ensure all data was correct. This resulted in one error because "malena" was not found in the Wuggy.app dictionary. Therefore, I gave the correct syllabification of "ma-le-na" where hypens represent syllable boundaries. I then re-verified the data before runnning the analysis. Once I receive no errors, I ran the following analysis to generate new nonwords.

Parameters:
General Settings:

Language module: Orthographic Spanish
Output type: Only pseudowords
Maximal number of candidates: 10 per word
Maximal search time per word: 10 seconds
Output Restrictions:

Checked: Match length of subsyllabic segements
Checked: Match letter length
Checked: Match transition frequencies (concentric search)
Checked: Match subsyllabic segments: "2" out of "3"
Output Options:

Syllables chosen from dropdown box
Checked: Lexicality
Checked: OLD20
Checked: Neighbors at edit distance 1
Checked: Number of overlapping segments
Checked: Deviation statistics

Output file
Wuggy outputs a file that contains all the raw data that was then imported into Microsoft Excel to preserve all the unicode characters contained in the data. Do not simply open this from Windows Explorer or Finder as it will replace all encoding of the file eliminating special characters.

WARNING: Wuggy uses syllabification and it removes the accented vowels in Spanish. The only way to remedy this problem is to fix it once it has been imported into Excel.

Once in Excel, I renamed the worksheet WuggyRawOutput. Then I duplicated this worksheet and renamed the new worksheet WuggyBuilt. I then deleted all the columns except for columns A and B ("Word" and "Match"). I then added two columns called "Nonword\_Concat" and "Rand". In the Nonword\_Concat column, I entered the following formulas:

For CV-CV-CV words: \=CONCATENATE(LEFT(B2,2),MID(B2,4,2),MID(B2,7,2))
This formula combines the two leftmost characters, skips character 3 (the first syllable boundary marker), adds the 4th and 5th characters to the string store, skips the 6th character (the second syllable boundary marker), and finally adds the 7th and 8th characters from the left to create the six-letter nonword.
For CVC-CVC words: \=CONCATENATE(LEFT(B403,3),RIGHT(B403,3))
This formula combines the three leftmost and three rightmost characters to create the six-letter nonword excluding the one and only syllable boundary marker.

These two formulas are necessary in order to create nonwords without the syllable markers. Once these steps have been completed, the file is ready for the python code housed in the cell below.

%end wuggy stuff
%knit_child('Appendices/AppendixB.rnw')
%knit_child('Appendices/AppendixC.rnw')


%----------------------------------------------------------------------------------------
%	BIBLIOGRAPHY
%----------------------------------------------------------------------------------------
%
% References must be included in dissertation
% References must be consisent (check your citation style formatting

\printbibliography[heading=bibintoc]

%----------------------------------------------------------------------------------------

\end{document}  
