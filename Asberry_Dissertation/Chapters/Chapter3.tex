% Chapter Template

\chapter{Visual Word Segementation} % Main chapter title

\label{Chapter3} % Change X to a consecutive number; for referencing this chapter elsewhere, use \ref{Chapter3}
%----------------------------------------------------------------------------------------
%	SECTION 0
%----------------------------------------------------------------------------------------

\section{Abstract}

Give Segmentation abstract here

Keywords: (list all words necessary)

%----------------------------------------------------------------------------------------
%	SECTION 1
%----------------------------------------------------------------------------------------

\section{Introduction}

One of the processes necessary to process spoken speech is the ability to break up the continuous stream of spoken language into manageable chunks of information—i.e. segmentation. In order to segment speech, attaching meaning to the segmented chunks of sounds is not a prerequisite as it is in the retrieval of lexical items. This means a speaker can segment speech by isolating or detecting certain language sounds from a spoken language in which they have no knowledge—a common practice in psycholinguistic research to ensure prelexical decisions are being made \parencite{Cutler1986-zl}. Understanding that people can successfully segment speech of an unknown language, having naïve and native listeners of a language search spoken speech for a particular sequence of sounds tells researchers several things. First, it allows for comparisons of different languages and testing hypotheses about speech segmentation strategies depending on experimental design. Furthermore, it can give additional support the effects that may or may not be found are not due to experimental design or items.
It is important to note that this dissertation builds off of nearly three decades of research that also fell into the second level of investigation of the syllable discussed by Mehler and Hayes \parencite*{Mehler1981-wp}---a unit which aides speech perception and comprehension. These researchers abandoned the search for the minimal perceptual unit and turned their focus towards the syllable’s role in speech perception despite the fact it was not likely to be a minimal perceptual unit for speech processing. While the monitoring paradigm was not going to help researchers discover the minimal perceptual unit used to process spoken language, this methodology was able to be used for testing the syllable’s role in speech processing. The syllable monitoring paradigm presents a target syllable such as “pa, pal, ba, bal, ca or car”. Then a participant must then identify the target syllable that is embedded in speech, which is presented as a list of words, carrier items, rather than connected and continuous speech. Using French subjects in a French syllable monitoring experiment, participants were asked to find CV or CVC syllables in lists of bisyllabic words, which had initial syllable structures of CV or CVC where both the target and carrier items were presented auditorily \parencite{Mehler1981-vi}. For example, an experimental trial started with the target presentation “The target is “pa”, which was immediately followed by the list of words, ranging from two to five words, which may or may not contain a critical word as the last item of the list. Only one out of every ten sequences of words contained a critical carrier item. The critical trials contained only one carrier item of a pair \emph{PA.LA.CE–PAL.MIER} or \emph{CAR.TON–CA.ROTTE} that appeared as the last item of the word sequence, where both words always shared the initial three phonemes, but they differed in syllable structure. The experimental conditions varied on the target syllable and initial syllable of the carrier item could match could match (find “pa” in PA.LACE, “pal” in PAL.MIER, “ca” in CA.ROTTE or “car” in CAR.TON) or mismatch (find “pal” in PA.LACE, “pa” in PAL.MIER, “ca” in CAR.TON or “car” in CA.ROTTE).  In a French syllable monitoring experiment, participants were asked to press a response key as soon as they have heard their target sound fragment in the speech they were hearing, French speakers found the target syllable faster when the initial syllable structure of the carrier item matched the target. The real French nouns used as critical items showed a faster detection time of the target pa when PALACE rather than PALMIER was the critical item. Likewise, the target pal was found faster when the critical item was  PALMIER rather than PALACE \citep{Mehler1981-vi}. This “crossover effect” found by Mehler, Dommergues, Frauenfelder and Segui \parencite*{Mehler1981-vi}---faster detection times when target syllable structure matched initial syllable of carrier word---was not found by native English speakers who monitored real English words where one word in each pairing had an unclear syllable boundary---an ambisyllabic \[l\] \parencite{Cutler1986-zl}. In other words, English speakers did not show differences in the speed in which they identified the target syllable according to the syllable structure of the carrier item. Finding “pa” in PA.LACE was not easier than finding “pal” in PA.LACE nor was finding “pal” in PAL.PI.TATE easier than finding “pa” in PAL.PI.TATE. The follow up to this finding was to have native English speakers monitor the French noun used by Mehler et al. \parencite*{Mehler1981-vi} where syllable boundary ambiguity was not present in the carrier items. Given that these were naive listeners of French, the segmentation process is being examined here because these listeners will have no lexical entries for any of the French words they hear. The English listeners again did not show the crossover effect suggesting that syllable based segmentation is not a strategy used in speech segmentation by English monolinguals even when it would be beneficial to do so \parencite{Cutler1986-zl}. Importantly, this finding suggests that segmentation strategies are not universal, but are language specific. In order to give additional strength to their argument, the researchers used the English experimental items that did not produce the crossover effect with native English speakers, and ran the same experiment with native speakers of French who had not learned English. The French listeners still showed the crossover effect even when listening to an unknown language—in this case, English. 
Given the nature of the differences found between English and French monolingual speakers’ strategy for speech segmentation, the next logical step while remaining in the same of vein of research was to investigate the segmentation strategy used by bilinguals. In a series of experiments, both English and French language modes of French–English balanced bilingual speakers were tested \parencite{Cutler1992-qq}. The French mode of the experiment used the same stimuli as Mehler et al. \parencite*{Mehler1981-vi} which had showed a strong syllabic effect with French monolinguals, but showed no such effect with the bilingual population. Post-hoc analysis split the bilinguals into two groups according to language dominance as reported by the participant in response to the following question, “Suppose you developed a serious disease, and your life could only be saved by a brain operation which would unfortunately have the side effect of removing one of your languages. Which language would you choose to keep?” This manner of determining language dominance was then used in all remaining experiments of the study. Under this post-hoc analysis where language dominance was considered for the experiments conducted in French, French-dominant listeners patterned like French monolinguals. Likewise, English dominant listeners patterned like English monolinguals from previous studies for the experiments conducted in French. In the French versions of the experiments, French-dominant participants exhibited the syllable-based segmentation strategy while the English-dominant participants did not show evidence of a syllable-based segmentation.  In the English versions of the experiments, French dominant speakers patterned unlike the French monolinguals while the English dominant speakers continued to pattern like the English monolinguals. Like previous studies with English monolinguals and the English-dominant bilinguals of the Cutler et al. \parencite*{Cutler1986-zl}, the French-dominant bilingual were not able to find “pa” in PALACE easier than “pal” in PALACE nor were they able to find “pal” in PALPITATE faster than “pa” in PALPITATE. Where French was considered the dominant language of the speaker, it appeared that they had identified the ineffectiveness of the syllable based segmentation strategy and were able to inhibit its application while listening to English. When English was considered the dominant language of the speaker, it appeared that these speakers could not utilize a syllable based segmentation strategy despite the fact that one of their two languages could use a syllabic segmentation strategy.  This suggested that bilinguals—even in their most balanced form—are not two monolinguals within a bilingual mind.
English and French are quite different in their phonological structure: (1) French is syllable-timed while English is stressed-timed, (2) French has fixed stress while English has variable stress, (3) French has no vowel reduction while English has rampant vowel reduction and (4) French has clear syllable boundaries while English has ambiguous syllable boundaries. As a result of the phonological structures of French and English, in the previous experiments stressed syllables were always the carrier syllables for English speakers while unstressed syllables were always the carrier syllables for the French speakers. This stress factor was considered as a potential confound in that stressed syllables are known to carry more phonetic details and last longer than unstressed syllables. To overcome this difference, a different group of bilingual speakers were recruited—Catalan–Spanish bilinguals—because stress is variable in both languages and can therefore be controlled \parencite{Sebastian-Galles1992-xd}. Unlike French and English that varied on multiple factors, Catalan and Spanish only differ in vowel reduction—Catalan has vowel reduction while Spanish does not. With stress being controlled across both languages, vowel reduction can be isolated as it is allowed in Catalan, like in English, but it is not in French or Spanish. They found that Catalan dominant speakers monitoring Catalan speech produced the crossover effect only when the initial syllable of the carrier item was unstressed. Spanish was also considered in the same study where the researchers found no crossover effect by Spanish dominant speakers monitoring in Spanish. Given the similarities between Spanish and French, both have clear and unambiguous syllable boundaries nor vowel reduction, it is surprising that they did not find a crossover effect with stressed or unstressed initial syllables of the carrier items as they did with Catalan speakers. Sebastián-Gallés et al. \parencite*{Sebastian-Galles1992-xd} attempted to force a post-lexical decision by the Spanish dominant speakers with the incorporation of an additional semantic relatedness task. This succeeded in slowing the reaction time by an average of 250 milliseconds and found a syllabic effect in both initially stressed and unstressed Spanish words. The syllabic effect similar to the those found in Sebastián-Gallès et al. \parencite*{Sebastian-Galles1992-xd} was later replicated in Italian by Tabossi et al. \parencite*{Tabossi2000-xn}.
In contrast to the findings of Sebastián-Gallés et al. \parencite*{Sebastian-Galles1992-xd} , Bradley, Sánchez-Casas and García-Albea \parencite*{Bradley1993-qq}  found a crossover effect when Spanish speakers monitored Spanish carrier items. They also found no crossover effect for English speakers monitoring English carrier items. Similar to Cutler et al. \parencite*{Cutler1986-zl}, Bradley et al. \parencite*{Bradley1993-qq}  tested naive listeners with the same material. They found no crossover effect for English monolinguals monitoring Spanish carrier items, which was comparable to the French findings. However, unlike the French findings, the Spanish monolinguals monitoring in English also showed no syllabic segmentation effect. Bradley et al. \parencite*{Bradley1993-qq} then turned to Spanish–English bilinguals where they again found no crossover effects when monitoring Spanish carrier items. This suggests that these native speakers of Spanish and English L2 speakers have abandoned their native segmentation strategy even when listening to one of their native languages (Spanish). This result again differs from the Cutler et al. \parencite*{Cutler1986-zl} French–English bilinguals because the French kept the native syllable-based segmentation strategy when listening to French and abandoned it only when listening to English where it was no longer effective.


\textbf{DELETE ME Write my introduction to the syllabification article here (Will write the introduction LAST)
The section of your article most likely to be read, not skimmed or skipped. The first paragraph or two is the overview of the article: describe the problem, question or theory motivating the research.} 

%-----------------------------------
%	SUBSECTION 1-1
%-----------------------------------
\subsection{Background}

In the second part, describe relevant theories, review past research and give more details on the current research question. Do not forget about signposting, which headings and subheadings can naturally create for the reader.

%-----------------------------------
%	SUBSECTION 1-2
%-----------------------------------

\subsection{Present Study}
The first article of this dissertation will ground itself here in the previous literature by first attempting to replicate the Spanish findings of Bradley et al. \parencite*{Bradley1993-qq}. with the Spanish-dominant bilingual group, which will serve as the control, and the English-dominant bilingual group, which will serve as the control floor. The third bilingual group in the experiment will provide new knowledge on the syllable’s role in speech processing.  The first group, henceforth the Spanish-dominant group, will consist of participants who grew up in Mexico speaking only Spanish and later learned their L2 English after the age of twelve. The participants in this group will be recruited from Guanajuato, Mexico and will have received their formal grade-level schooling in Spanish. The Spanish-dominant group will serve as the control in this experiment. The second group, the English-dominant group or late learners of Spanish, will consist of participants who grew up in Arizona speaking only English and later learned their L2 Spanish after the age of twelve. The participants for this group will be recruited from Tucson, Arizona and will have received their formal grade-level schooling in English. This group will serve as the first of two experimental groups. The last group, the Balanced-bilingual group or early learners of Spanish, will consist of participants that were exposed to both Spanish and English from infancy. These participants will have acquired both Spanish and English simultaneously, but unlike the Spanish-dominant group, they will have received their formal grade-level schooling in English. The Balanced-bilingual group participants will also be recruited from Tucson, Arizona and serve as the second experimental group. Utilizing these three groups, it will be possible isolate differences in segmentation strategies that are due to the age of acquisition. The experiment will be run completely in Spanish mode and the expectation is that the Spanish-dominant group will employ a syllabic based segmentation strategy while the English-dominant group will not. The balanced bilingual group could go in one of two ways. Since the experiment is being conducted in Spanish—a language that encourages the use of a syllabic segmentation strategy—the first possible outcome is that the early learners of Spanish will exhibit the same pattern as the Spanish-dominant group. If this is the outcome of the segmentation experiment, then it would provide stronger evidence for a syllable-based segmentation strategy for speakers of Spanish. The second option is that the early learners will pattern like the English-dominant counterparts and fail to employ the syllable-based segmentation despite the fact that the language input supports it. This would contradict the findings of monolingual and Spanish-dominant bilingual speakers of English, who appear to use this syllable-based strategy. If these were to be the results of the first experiment, it would give additional support to the findings of Bradley et al. \parencite{Bradley1993-qq} and there would be several possible factors that would warrant further investigation. Since the population of balanced bilinguals will be recruited from Tucson, Arizona and have completed their schooling in English, it would be a worthwhile endeavor to find a comparable balanced bilingual population who received their schooling in Spanish. This would allow for a comparison of the effect of schooling and explicit teaching of syllables, which typically occurs when children are taught to read and its relation to speech segmentation strategies. 
Another avenue to investigate would be the syllabic intuitions of participants, which could be a factor given that English and Spanish differ in ways similar to English and French participants of previous research studies. It may be that the three bilingual populations do not agree on the syllabic structures of the speech they are segmenting as a result of language background profiles. It will be possible to look at syllabic intuitions from the data that will be collected in a syllabic intuition task conducted in the second experiment of the first article in the proposed dissertation project. 
The main purpose of this study is to determine whether a representation of the syllable is a represented linguistic unit in the minds of bilingual Spanish speakers. This chapter will address the following questions:
1.	Does a representation of the syllable exist in minds of Spanish–English bilinguals which is available to aide in pre- or post-lexical levels of segmenting spoken Spanish?
2.	Does the age of acquisition of Spanish in Spanish–English bilinguals determine whether or not syllabic intuitions of Spanish match the intuitions of native Spanish speakers?
This first question on the representation of the syllable has been one with unclear results in previous research. As a precaution to building this entire dissertation on less-than-stable previous findings, a quasi-pilot study was completed in the summer of 2018. One PsychoPy experiment that included two separate tasks—an identification task and a syllabification task—was designed and conducted completely in Spanish. Eight (8) native speakers of Spanish who attended school in a Spanish speaking country and learned English as adults were recruited from the University of Arizona main campus in Tucson, AZ. All participants were highly proficient in their English (L2) as they were currently enrolled in a masters or doctorate program or had just recently completed their graduate degree at the University of Arizona. This group showed a consistent syllabification pattern and showed a significant interaction between the visually presented targets and carrier items in the identification task. Participants responded faster when the target syllable structure matched the initial syllable structure of carrier item than when they did not coincide.
Finding this effect in the population of Spanish dominant bilingual speakers of English, which was representative of one of the three populations being investigated, gave the confidence needed to proceed with the dissertation project. Therefore, the second chapter of the dissertation will use PsychoPy to conduct two experiments—an identification task using a syllable monitoring paradigm and a syllabic intuition task using a two option forced-choice methodology. Three additional instruments, Bilingual Language Profile (BLP), the LexTALE English vocabulary test and the LexTALE-Esp Spanish vocabulary test, will also be used to collect demographic information and language proficiency data that will be used for placement of participants into the appropriate bilingual group or to exclude them altogether from the data analysis. The nature of these tasks is discussed in the instruments section of the experiment 1. However, it is important to note that both of these tasks will take place following the completion of experiments 1 and 2, syllable monitoring and syllabic intuition experiments, in an effort to avoid biasing participants on the nature of the investigations in which they are participating.


\textbf{DELETE ME The third section titled, “The Present Experiment” or “The Present Research”, follows and contains experimental descriptions and how they address the questions being asked.
For most articles, keep your introduction under 10 pages}

%----------------------------------------------------------------------------------------
%	SECTION 2
%----------------------------------------------------------------------------------------

\section{Methods}

Write this section FIRST
is the section that describes how the research was conducted. A good one shows how well thought out the experiment design is and allows other researchers to easily replicate it. This section also follows a formula:

%-----------------------------------
%	SUBSECTION 2-1
%-----------------------------------
\subsection{Participants}

The participants were split into three distinct populations of Spanish–English bilinguals. The Spanish–dominant group consisted of native speakers of Spanish that lived in Sonora, Mexico and learned English after \[INSERT AGE HERE\]. The English–dominant group were native English speakers that lived in Tucson, Arizona and learned Spanish after \[INSERT AGE HERE\]. The early bilingual group consisted of participants who lived in Tucson, Arizona, but were exposed to both English and Spanish before the \[INSERT AGE HERE\].
\emph{The participants will be placed into their appropriate group based off of the combined results of Bilingual Language Profile (BLP), the LexTALE and the LexTALE-Esp vocabulary tests.} 

Based off of responses in the BLP, participants who report proficiency in a language other than English and Spanish will be removed from the data analysis. According to the findings of Lemhöfer \& Broersma \parencite*{Lemhofer2012-hz}, the LexTALE vocabulary tests can distinguish between lower intermediate (up to 59 percent), upper intermediate (60–80 percent) and advanced (above 80 percent) levels of proficiency based on average percent correct responses. Participants that score lower than \[INSERT CUTOFF PERCENTAGE HERE\] were excluded due to having too low of a proficiency score in one of the two languages. The Spanish–dominant group had a  \[INSERT PERCENTAGE HERE\] success rate in Spanish and it was higher than their English success rate. The English-dominant group scored \[INSERT PERCENTAGE HERE\] or higher in English and it was higher than their Spanish scores. The early bilingual group had a correctness score of \[INSERT PERCENTAGE HERE\] or more in both English and Spanish. \[INSERT \# OF PARTICIPANTS\] participants were excluded from the analysis due to falling below the minimum standards used to describe each group. \[INSERT \# OF PARTICIPANTS\]  participants were recruited from each bilingual population and are reported in the analysis section. 
\emph{During data collection, more participants than reported here were collected since it was expected that several participants would be removed for one or more of the reasons listed above. When more than \[INSERT \# OF PARTICIPANTS\] participants remained eligible after removing participants who did not fit the population criteria, a random sampling of \[INSERT \# OF PARTICIPANTS\] participants were selected from eligible pool of participants.}

\textbf{DELETE ME Give accurate descriptions of participant groups, how they were classified, number of participants, etc.}

%-----------------------------------
%	SUBSECTION 2-2
%-----------------------------------

\subsection{Design}

There will be 24 word pairs selected as critical items where the initial syllable structure varies between a CV and a CVC structure while the first three phonemes are shared between the two—i.e. ba.la.da–bal-do-sa, cu.le.bra–cul.pa.ble, mo.re.ra–mor.ci.llo, jo.ro.ba–jor.na.da, etc. In addition to the 24 critical word pairs, another 294 real Spanish words will be selected to use as fillers and are also balanced according to initial syllable structure—147 start with a CV syllable and 147 start with a CVC syllable. Both critical items and fillers are all trisyllabic nouns where stress fell on the penultimate syllable. Target fragments representing a CV or CVC syllable—“ba” or “bal” respectively—will also be recorded separately. All critical items and fillers will be recorded by a single Spanish native speaker in a quiet sound booth in the Arizona Applied Phonetics Laboratory at the University of Arizona. 

There will be four different versions of the experiment for counter-balancing purposes, which is accomplished by balancing across participants. For the example word-pair “balada”–”baldosa”, condition 1 would search for “ba” in “balada”, condition 2 would search for “bal” in “balada”, condition 3 would search for “ba” in “baldosa” and condition 4 would search for “bal” in “baldosa”. Each participant will be randomly assigned to one of the four versions of the experiment, which means that no participant will see both critical words from the critical word pairs more than once during the experiment. For example, if participant 1 is assigned to a version of the experiment where condition 1 (find “ba” in “balada”) is presented to them for the critical word pair “balada–baldosa” then participant 1 would not see conditions 2, 3 or 4 in their experiment. Each version of the experiment presents 12 CV and 12 CVC critical trials where half of each type of syllable structure contains a match between the syllable structure of the target and critical item while the other half are mismatched. Each block presented to the participant contains 1 critical trial and 11 filler trials which are also balanced for CV and CVC syllable structures.

\textbf{DELETE ME Give an accurate idea of how the overall project was designed, what previous studies is it based off, theoretical principles, etc.
DO NOT write about what participants do in this section.}

%-----------------------------------
%	SUBSECTION 2-3
%-----------------------------------
\subsection{Instrumentation}

The LexTALE and the LexTALE-Esp are tasks used to correlate vocabulary knowledge and language proficiency in English and Spanish respectively \parencite{Izura2014-yw,Lemhofer2012-hz}. The LexTALE-Esp has also been shown to discriminate well between highly proficient Spanish speaking participants with different language dominances \parencite{Ferre2017-jq}. The LexTALE and LexTALE-Esp tests will be used in order to group participants into the appropriate bilingual population—Spanish–dominant bilinguals of L2 English, English–dominant bilinguals of L2 Spanish or early bilinguals who were exposed to both English and Spanish before \[INSERT AGE HERE\]. LexTALE is publicly available online and has been designed to run in PRAAT, Matlab and Presentations. For data collection purposes in the current study, participants completed both the LexTALE and LexTALE-Esp using PsychoPy. 
A second instrument used in this study was the Bilingual Language Profile (BLP), which is a survey based assessment tool for determining language dominance \parencite{Birdsong2012-wd}. It assesses language history, use, proficiency and attitudes of participants in less than 10 minutes. This assessment tool has been used in numerous language studies with a focus on bilingualism and is available for free under the creative commons license. This tool allowed for the collection of information about participants demographics---name, age, sex, place of residence and educational background---which took place at experiment setup in the current study. It also allowed for participants to indicate their language history, use, proficiency and attitudes and was the last task complete during the current study. Since the BLP survey was completed after all experimental trials, the participants were able to choose whether they received Spanish or English instructions for the survey. The BLP is publicly available in a paper-based format or electronic format through the use of Google Forms. In an effort to make the experiment seamless as possible, the participants in the current study took the BLP within the PsychoPy platform as well. 


\textbf{DELETE ME Any special instrumentation could be included here (I am not sure that mine deserves a devoted section to instrumentation.}

%-----------------------------------
%	SUBSECTION 2-4
%-----------------------------------

\subsection{Procedure}
Participants were seated in front of a laptop computer with a USB button box in order to complete the experiment in PsychoPy. At the beginning of the experiment, participants entered in demographic information as asked in the basic language profile (BLP). Once they had entered the demographic information, the participants completed the Spanish version of LexTALE-ESP---a Spanish vocabulary test. 

Participants were instructed that they would be presented with a sequence of letters of for which they were to find in a list of words that would appear on the screen one by one. They were instructed to respond only if they had identified the sequence of letters in the word on the screen and to do nothing otherwise. The participants were also instructed to respond as fast as possible and were reminded with feedback screens staggered throughout the trials encouraging faster response times. The participants were first presented with 8 practice trials that followed the same criteria and procedure as the 48 experimental trials. Each trial began with text "Encuentre" above the sequence of letters, henceforth the target, which was presented in all capital letters in the center of the screen. The initial trial screen contained the target for 4 seconds before returning to a blank screen for 500 milliseconds. Following the blank screen, a list of ten words was presented randomly one at a time for 2000 milliseconds each with a 150 millisecond interstimulus interval. Only one word, the carrier item, in each list of ten words contained the target while the other nine words were simply filler items. The target was always found at the beginning of the carrier item. None of the filler items shared any of its first three letters with the target. The search target remained in the upper right hand portion of the screen to serve as a reminder while all ten words from the list were presented. Even when a response was made, only the first response was recorded, but the experiment did not progress until the 2000 millisecond presentation time had passed. Once all ten words from the list had been presented, the next trial began with a new target for participants to find in the next set list of ten words. No carrier or filler items were repeated through out the entire experiment. Participants were also given an optional 2 minute break halfway through the experimental trials.

Following the experiment participants also completed the English version of the LexTALE vocabulary test. 

Following the LexTALE task, participants also completed the bilingual language profile (BLP).

\textbf{combine this paragraph with the first one in this section}
The participants are given on screen instructions in Spanish that tell them that they will be presented with an auditory sound fragment—the target—followed by words that may or may not contain the sound fragment for which they are listening. They will be told to respond as quickly and accurately as possible when they hear the target sound fragment in the speech to which they are listening. Following the target presentation, words will be presented auditorily in sets of 12 carrier items, including only one of the critical words, with each carrier item being separated by 0.5 second interstimulus intervals (ISI). For example, the participants are instructed to find an auditorily presented fragment (“ba” or “bal”) in the following set of 12 auditorily presented words (“sotana”, “sonido”, “picota”, “torpeza”, “balada”, “semilla”, “rendija”, “renombre”, “sordera”, “tortuga”, “tersura”, and “sortija”). Each carrier item, including the critical item only has a 2 second response window from the onset of stimulus presentation before continuing on to the next carrier item in the list. The participants are instructed to press a single response button on a button response box using their preferred or dominant hand as soon as they have identified the target in one of the carrier items and are instructed to do nothing when the fragment is not present. The syllable monitoring experiment will include a practice run of 2 blocks containing 12 trials followed by 24 separate blocks of 12 experimental trials with each set of blocks—practice versus experimental—having their own directions that differ only in indicating whether the coming blocks will be practice trials or actual experimental trials.

\textbf{DELETE ME WHEN FINISHED
The first refer to mainly scope and size of research project while the procedure section is usually much more in-depth. Here, you connect your procedures to those in already published articles when possible and given detailed descriptions of classifications and scales used in procedure. Essentially ensure the reader does not suspect anything is being hidden and the researcher is honest. Do not repeat any unnecessary information in subsequent experiments of the paper.}


%----------------------------------------------------------------------------------------
%	SECTION 3
%----------------------------------------------------------------------------------------

\section{Results}

The filler trials will be removed from the data collected and so that only the critical trials remain. Participants who have less than a 90 percent success rate on critical trials will be removed from the data (Number of participants will be reported). Once the fillers and participants who have not completed the task successfully have been removed, a second pass will remove any individual participant responses under 200 milliseconds following the lower criteria range used by Bradley et al. \parencite*{Bradley1993-qq}. (The percentage of trials removed will be reported here). For the latency data, only the correct responses to critical trials will be analyzed.

\textbf{DELETE ME describe the analyses the researcher has done, but do not overload. Instead of creating a laundry list of statistics, create the story you want to tell using only the statistics that are related to addressing your problem. 
For each task, review the hypothesis, give the statistics and say what the result of the test means. 
Do not discuss the findings until you reach the discussion session. 
This is where tables and figures can help keep the paper looking clean and crisp instead of cluttered unorganized statistical test lists that are hard to follow. 
Figures show patterns while tables give details.}


%----------------------------------------------------------------------------------------
%	SECTION 4
%----------------------------------------------------------------------------------------

\section{Discussion}

split into several small sections following each individual experiment’s results sections where applicable. 
The General Discussion steps back and begins with an overview of the problem and then the findings. A general rule of thumb is to keep this section shorter than the introduction. Only give limitation directly related to the current study, not the general limitation of the research or the field as a whole and be sure to give a good reason for why these limitations are not as bad as they sound on the surface.


%----------------------------------------------------------------------------------------
%	SECTION 5
%----------------------------------------------------------------------------------------

\section{References}

Insert references for this chapter here.