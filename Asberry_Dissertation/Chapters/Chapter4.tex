% Chapter Template

\chapter{Visual Word Recognition} % Main chapter title

\label{Chapter4} % Change X to a consecutive number; for referencing this chapter elsewhere, use \ref{Chapter4}
%----------------------------------------------------------------------------------------
%	SECTION 0
%----------------------------------------------------------------------------------------

\section{Abstract}

Give Word Recognition abstract here

Keywords: (list all words necessary)

%----------------------------------------------------------------------------------------
%	SECTION 1
%----------------------------------------------------------------------------------------

\section{Introduction}

How humans understand and process spoken speech has been the subject investigation for decades in psycholinguistics. Processing spoken speech is a difficult task because there is a continuous and variable stream of language input, yet most speakers of a language appear to complete this complex task with little effort. Two specific processes used in spoken language processing are generally discussed in the literature—speech segmentation and lexical access. 
Speech segmentation addresses the strategies used by listeners when trying to parse spoken language input into processable chunks of speech regardless of whether the chunks have actual meaning or are from a language the listener understands (see Chapter 2 of the current dissertation, \citep{Cutler2002-ge, Cutler1986-zl, Cutler1992-qq, Dumay2002-hx, Finney1996-fw, Mehler1981-le, Segui1981-uf, Tabossi2000-xn}). Lexical access on the other hand, addresses the retrieval of meanings from the mental lexicon represented by phonological sequences. The mental lexicon is the warehouse of all lexical items—vocabulary words—that are represented in the memory of the language user. Unlike speech segmentation, accessing the lexicon in order to find words in a language does require a link to semantic meaning of a given entry. This implies that the listener has knowledge of what words are and are not in the language. Once the language user has broken the speech stream into manageable chunks, they can begin searching their mental lexicon for items that match the auditory input. This process is typically considered to be a combination of the activation of lexical items followed by a competition process that leads the listener to the top match stored in the lexicon or to the conclusion that input is not a real-word in the language.
Several studies have utilized visual word recognition to investigate the role of the syllable in accessing the mental lexicon. One such study investigated French native speakers’ ability to use the syllable to gain access to the mental lexicon with a lexical decision task under a masked priming paradigm \parencite{Ferrand1996-vu}. They found no significant results with their French participants, but only used a 29 millisecond prime display followed by a 14 millisecond backwards mask for a total stimulus onset asynchrony (SOA). Following the suggestion of Ferrand et al. \parencite*{Ferrand1996-vu} that the short SOA may have resulted in the inability of French speakers to use the syllable in lexical access, Carreiras and Perea \parencite*{Carreiras2002-mp} investigated syllable congruency as a means of measuring the facilitation or inhibition in the lexical access of Spanish words utilizing a longer SOA. While they found the a crossover effect using six letter disyllabic Spanish words (CV-CV-CV or CVC-CVC in syllabic structure), they also found a speed–accuracy trade-off effect with the two different SOAs—116 milliseconds versus 166 milliseconds. The participants responded faster to the lexical decision task when they had the additional 50 milliseconds of processing time in 166 msec SOA, but also made significantly more errors than participants who completed the same task with the 116 msec SOA.
The previous results had no way of disentangling the orthographic syllable from the phonological syllable. Álvarez, Carreiras and Perea \parencite*{Alvarez2004-nd} set out to investigate whether or not the phonological syllable was a unit used during visual word recognition. In their first experiment, they use disyllabic words with initial CV or CVC syllables allowing them to look at the differences between segmental and syllabic overlap by having the prime and targets sharing the first three segments, but varying syllabic structure between CV and CVC—i.e. ju.nas-JU.NIO and jun.tu-JU.NIO appeared in separate lists for counterbalancing across participants. They found that CV targets were responded to faster when the prime had a CV initial structure as well when compared to primes that had an initial CVC syllable structure. However, the same pattern was not found for CVC targets. Participants also made more errors on CVC target than CV targets. These findings do support that fact that syllabic priming effects can be found in lexical decision tasks at short SOAs, 64 millisecond in this study. It is important to note that only a 19 millisecond addition is added to the SOA used by Ferrand et al. \parencite*{Ferrand1996-vu} who did not find syllabic priming in French lexical decision task—also a Romance language like Spanish that has clear syllable boundaries. This suggests that very small adjustments in experimental design could affect the results of the experiment and should be considered carefully The second experiment focuses in more on the question of distinguishing between the phonological and orthographic syllable \parencite{Alvarez2004-nd}. The researchers here utilize the clear and unambiguous phoneme–grapheme correspondence and the fact that several graphemes map onto the same phoneme. For example, “b” and “v” graphemes both map onto the phoneme /b/ in Spanish. This study used four conditions to capture the effects of phonological versus orthographic syllables—vi.rel-VI.RUS is the same orthographic and phonological syllable; bi.rel-VI.RUS is a different orthographic syllable, but the same phonological syllable; lastly, vir.ga-VI.RUS and bir.ga-VI.RUS were set as control primes that shared the first three phonemes, but differed in syllable structure. They found faster responses were elicited when overlap included both orthographic and phonological representations than when it only had phonological overlap. Given the speed–accuracy trade-off between the reaction times and error rates of the orthographic-phonological and phonological only conditions, it suggested that phonological activation is occurring during the visual word recognition experiments. The final experiment of Álverez, Carreiras and Perea \parencite*{Alvarez2004-nd} showed that the initial syllable is the better key to lexical retrieval as phonological rime priming was not significantly different than primes sharing first three phoneme, but not the initial syllable structure. Additional evidence for the syllable congruency effect and the initial syllable being an essential key to lexical access comes from the event-related potential (ERP) experiments that presented word in two colors where one color represented the initial syllable or misrepresented it \citep{Carreiras2005-us}. They found syllable congruency effects in Spanish for low-frequency real words and pseudowords, but not for high-frequency words. This congruency effect found also interacted the lexicality judgements, but the time frames in which these two processes started differed in time. The congruency effect was available much earlier than the actual lexicality judgement suggesting that segmentation and lexical access are different processes that may work in conjunction with one another. Several years later, an interesting study introduced age and a medical condition, Alzheimer’s disease, into their experiment design to further investigate the syllable congruency effect \citep{Carreiras2008-ar}. They replicated the fourth experiment of Carreiras and Perea \parencite*{Carreiras2002-mp} and were able to find the syllable congruency effect in the control group—elderly people without Alzheimer’s disease—as well as the Alzheimer patients despite large differences in latencies across groups. Carreiras et al. \parencite*{Carreiras2008-ar} also tested syllable frequency effect in a second experiment. The syllable frequency effect was found in the young controls whereas the older age group appears to have a deterioration in the ability to inhibit lexical competitors.

% \textbf{DELETE ME Write my introduction to the syllabification article here (Will write the introduction LAST) The section of your article most likely to be read, not skimmed or skipped. The first paragraph or two is the overview of the article: describe the problem, question or theory motivating the research. }

%-----------------------------------
%	SUBSECTION 1-1
%-----------------------------------
\subsection{Background}

%In the second part, describe relevant theories, review past research and give more details on the current research question. Do not forget about signposting, which headings and subheadings can naturally create for the reader.

%-----------------------------------
%	SUBSECTION 1-2
%-----------------------------------

\subsection{Present Study}
The current study seeks to replicate the previous findings and expanding on the existing literature by including the three varying degrees of Spanish–English bilingual populations. Given the nature of three bilingual populations, the Spanish-dominant group should replicate the previous findings that Álvarez et al. \parencite*{Alvarez2004-nd} and Carreiras and Perea \parencite*{Carreiras2002-mp} found with monolingual Spanish speakers. Given the Canary Island dialect of Spanish used in the Álvarez et al. \parencite*{Alvarez2004-nd} where letters “z, s, and c” when followed by “e” and “i” are all pronounced as an /s/, all of the experimental items are transferable to the Mexican dialects of Spanish being studied in the current article. Due to syllabification differences between Spanish and English, the English dominant group will not likely show the an effect of syllable structure in contacting the lexicon which has been shown in previous studies. The third group and possibly the most interesting bilingual population being studied is the balanced bilingual group. Given that this group has learned both Spanish and English simultaneously, it is hard to predict whether or not this group will use a syllabic key to lexical access. Nonetheless, either result for this group provides interesting results. If they follow the pattern of the Spanish-dominant group, it will be the first finding of its kind in pairings where English was one of the two languages of the bilinguals to reveal a syllabic effect. If the balanced bilingual group patterns alongside the English-dominant group, then further evidence could be drawn for the access of to one and only one strategy for lexical access. In this case, the syllabic strategy could be argued to be a default strategy for language users until it proves ineffective in which case listeners abandon it for segmental access to the lexicon. In experiment 1, testing this effect at the subconscious level will be done through masked priming \citep{Forster1984-sf}. 
Transitioning from the findings of Chapter 2, which investigates the syllable’s role in speech segmentation, this investigation goes into the role that the syllable plays in lexical access. The study will address the following questions:
1.	Is the structure of the syllable represented in the organization of the mental lexicon?
2.	If there is evidence of syllable structures in the mental lexicon, what are the differences, if any that exist between the three types of bilingual populations?
3.	Using a masked prime—subconscious processing—and visually presented target; Is the phonological syllable activated as a means to kick off a lexical search?

%\textbf{DELETE ME The third section titled, “The Present Experiment” or “The Present Research”, follows and contains experimental descriptions and how they address the questions being asked. For most articles, keep your introduction under 10 pages}

%----------------------------------------------------------------------------------------
%	SECTION 2
%----------------------------------------------------------------------------------------

\section{Methods}

%Write this section FIRST. It is the section that describes how the research was conducted. A good one shows how well thought out the experiment design is and allows other researchers to easily replicate it. This section also follows a formula:

%-----------------------------------
%	SUBSECTION 2-1
%-----------------------------------
\subsection{Participants}

The participants will be recruited from the same three populations as those represented in chapter 2, but no participant that completed experiments for chapter 2 or the quasi pilot study will participate in the experiments of chapter 3. The participants will make up three distinct populations of Spanish–English bilinguals, which will remain the same throughout all the experiments carried out in the following three chapters. The first group, henceforth the Spanish-dominant group, will consist of participants who grew up in Mexico speaking only Spanish and later learned their L2 English after the age of twelve. The participants in this group will be recruited from Guanajuato, Mexico and will have received their formal grade-level schooling in Spanish. The second group, the English-dominant group, will consist of participants who grew up in Arizona speaking only English and later learning their L2 Spanish after the age of twelve. The participants for this group will be recruited from Tucson, Arizona and will have received their formal grade-level schooling in English. The last group, the Balanced-bilingual group, will consist of participants that were exposed to both Spanish and English from infancy. These participants will have acquired both Spanish and English simultaneously, but unlike the Spanish-dominant group, they will have received their formal grade-level schooling in English. The Balanced-bilingual group participants will also be recruited from Tucson, Arizona. The participants will be placed into their appropriate group based off of the combined results of BLP and LexTALE tasks. In addition, based off of responses in the BLP, participants who report proficiency in a language other than English and Spanish will be removed from the data analysis. Due to the likely fact that several participants will necessarily be excluded from the analysis, 30 participants will be recruited for each population in each chapter. If more than twenty participants remain eligible after removing participants who do not fit the population criteria, a random sampling of 20 participants will be selected from eligible pool of participants.

%\textbf{alternate participant description to work with}
The participants will make up three distinct populations of Spanish–English bilinguals, which will remain the same throughout all the experiments carried out in the following three chapters. The first group, henceforth the Spanish-dominant group, will consist of participants who grew up in Mexico speaking only Spanish and later learned their L2 English after the age of twelve. The participants in this group will be recruited from Guanajuato, Mexico and will have received their formal grade-level schooling in Spanish. The second group, the English-dominant group, will consist of participants who grew up in Arizona speaking only English and later learning their L2 Spanish after the age of twelve. The participants for this group will be recruited from Tucson, Arizona and will have received their formal grade-level schooling in English. The last group, the Balanced-bilingual group, will consist of participants that were exposed to both Spanish and English from infancy. These participants will have acquired both Spanish and English simultaneously, but unlike the Spanish-dominant group, they will have received their formal grade-level schooling in English. The Balanced-bilingual group participants will also be recruited from Tucson, Arizona. The participants will be placed into their appropriate group based off of the combined results of BLP and LexTALE tasks. In addition, based off of responses in the BLP, participants who report proficiency in a language other than English and Spanish will be removed from the data analysis. Due to the likely fact that several participants will necessarily be excluded from the analysis, 30 participants will be recruited for each population in each chapter. If more than twenty participants remain eligible after removing participants who do not fit the population criteria, a random sampling of 20 participants will be selected from eligible pool of participants. The participants will be recruited from the same three populations as those represented in chapters 2 and 3, but no participant that completed experiments for chapters 2 or 3 or the quasi pilot study will participate in the experiments of chapter 4. 

%\textbf{DELETE ME Give accurate descriptions of participant groups, how they were classified, number of participants, etc.}

%-----------------------------------
%	SUBSECTION 2-2
%-----------------------------------

\subsection{Design}

%\textbf{This paragraph needs major revision, no longer using Álvarez}
Following the methods used to disentangle phonological and orthographic syllables by Álvarez et al. \parencite*{Alvarez2004-nd}, 80 Spanish words that contain unstressed initial syllables will be chosen as targets for the lexical decision task. 40 of the words began with a CV syllable structure while 40 words begin with a CVC syllable structure. The CVC masked primes will be composed of syllables where half match the first three phonemes in orthography and phonology (bal-BAL.CÓN) and the other half match in the phonology of the first three phonemes, but differed in orthography (val-BAL.CÓN). The CV masked will mimic the pairings of the CVC masked primes with the exception that only the first two phonemes are considered—ba-BAL.CÓN and va-BAL.CÓN. In addition to the 80 real Spanish words, 80 nonce words will be created that all abide by orthographic constraints of Spanish.

%\textbf{DELETE ME Give an accurate idea of how the overall project was designed, what previous studies is it based off, theoretical principles, etc. DO NOT write about what participants do in this section.}

%-----------------------------------
%	SUBSECTION 2-3
%-----------------------------------
\subsection{Instrumentation}

LexTALE and LexTALE-Esp are tasks used to correlate vocabulary knowledge and language proficiency in English and Spanish respectively \citep{Izura2014-yw, Lemhofer2012-hz}. The LexTALE-Esp has also been shown to discriminate well between highly proficient Spanish speaking participants with different language dominances \citep{Ferre2017-lp}. The LexTALE and LexTALE-Esp tests will be used in order to group participants into the appropriate bilingual population—Spanish–dominant bilinguals of English (L2), English–dominant bilinguals of Spanish (L2) or Balanced bilinguals (exposed to both English and Spanish since early childhood). Again, the LexTALE is available online and has been designed to run in PRAAT, Matlab and Presentations, but the participants will take it using PsychoPy at the experiment testing location immediately following the experimental tasks. 
A second instrument used in this study is the Bilingual Language Profile (BLP) is survey based assessment tool for determining language dominance \citep{Birdsong2012-wd}. It assesses language history, use, proficiency and attitudes of participants in less than 10 minutes. This assessment tool is commonly used in language studies with a focus on bilingualism and is available for free under the creative commons license. This tool allows researchers to collect information about participants demographics—name, age, sex, place of residence and educational background. It also allows for the participant to talk about the language history, use, proficiency and attitudes. For the current dissertation project, this survey will be administered through Google Forms as designed by the creators of the BLP while they are in the experiment testing location. Since no modification will be made to the BLP survey, the scores will be automatically calculated as designed by the creators. In all cases, the BLP survey will be completed after the experimental trials to avoid any confounding factors of language activation, Spanish or English, the participants will be able to choose whether they would like to take the survey with Spanish or English instructions. The participants will be recruited from the same three populations as those represented in chapters 2, but no participant that completed experiments for chapters 2 or the quasi pilot study will participate in the experiments of chapter 3.

%\textbf{DELETE ME Any special instrumentation could be included here (I am not sure that mine deserves a devoted section to instrumentation.}


%-----------------------------------
%	SUBSECTION 2-4
%-----------------------------------

\subsection{Procedure}

Participants were seated in front a laptop computer and given a button box in which to respond with following the entering of demographic information at the beginning of the experiment. Once the participant was ready to begin, the first task they completed was the LexTALE-Esp---the Spanish version of the vocabulary test. This served to place the bilingual participants in Spanish mode before proceeding the lexical decision task.

For the lexical decision task, all instructions were given in Spanish regardless of the response they gave to "preferred language" in the demographic section. The instructions told participants that they would see a series of pound signs on the screen followed by a word. When the word appeared on the screen, the participants were instructed to respond as quickly and accurately as possible to whether or not the word was a real Spanish word. The overall experimental trial presented a forward mask "\#\#\#\#\#\#" of six pound signs for 500 milliseconds. The forward mask was replaced by a prime revealing 2 or 3 lowercase letters followed by either 4 or 3 pound signs, respectively. The prime remained on the screen for 116 milliseconds before being replaced by the actual word in all capital letters on which the participants had to make a lexical decision. The prime always shared the same sequence of letters as the beginning of the word. The word remained on the screen for 2000 milliseconds or the participant entered a response before moving on to the next trial. In total, there were 320 trials where half the words were real Spanish words and the other half were noncewords that followed all Spanish orthographic rules. For trials with real words, half of the words began had a word initial stressed syllable while the other half did not. The 20 of the words containing a word initial stressed syllable began with a CV syllable structure---where C is a consonant and V is a vowel---while the remaining 20 words began with a CVC syllable structure. Half of both the syllable initial stressed words and syllable initial unstressed words, the sequence of letters revealed by the prime matched the initial syllable structure of the presented word. The other half of the words, the sequence of letters revealed in the prime did not match the initial syllable structure of the word. In some cases, the prime revealed one letter too many---the third letter would be represented as the first letter of the second syllable of the word---while in other cases, the prime did not reveal the entire first syllable---the third letter of the initial CVC syllable was not revealed until the entire word appeared on the screen. The nonword trials followed the same pattern as the real word trials.

Following the lexical decision task, participants then took part in the LexTALE task---the English version of a vocabulary test.

Following the LexTALE task, participants also completed the bilingual language profile (BLP).

%\textbf{this paragraph also needs major revisions}
As in experiment 1 of chapter 2, There will necessarily be four different versions of the current experiment for counter-balancing purposes, which is accomplished by balancing across participant. For example, the word “balcón” (balcony) will be the target one time for each participant who each receive it under a different condition. Condition 1 would prime BALCÓN with bal, which matches the initial syllable structure, the orthography of first three phonemes and the first three phonological segments. Condition 2 will prime BALCÓN with val, which matches the initial syllable structure, the first three phonological segments, but does not share the first three graphemes. Condition 3 uses the prime ba for the target BALCÓN, which does not match the initial syllable structure of the target, but shares the first two graphemes and phonological segments. The final condition, Condition 4, will use the prime va for the target BALCÓN, which does not match the initial syllable structure of the target, nor the first two graphemes, but does match the first two phonological segments. “Balcón” illustrates target words with a CVC initial syllable, but the experiment will also have targets with CV initial syllable structures. For example, balón (ball) would have four separate conditions: ba-BALÓN, va-BALÓN, bal-BALÓN and val-BALÓN.
Each participant will be randomly assigned to one of the four versions of the experiment, which means that no participant will see any target word more than once during the experiment. For example, if participant 1 is assigned to a version of the experiment where condition 1 (bal-BALCÓN) is presented to them then participant 1 would not see the target—BALCÓN—in conditions 2, 3 or 4 in their experiment. Each version of the experiment presents 40 CV and 40 CVC trials where half of each type of syllable structure contains a match between the syllable structure of the prime-TARGET while the other half are mismatched. In each version of the experiment half of the CV and CVC trials share orthographic and phonological segments while the other half only share phonological segments. In order to control for hand dominance, half of the participants will indicate a real Spanish word with the left button on the response box while the other half of participants will indicate a nonce word with the left button.

%\textbf{this paragraph should be combined with the first one of this section}
Participants will be instructed that a word will appear in the center of the screen and that they will need to indicate whether the word is a real word of Spanish as quickly and accurately as possible. Participants will be instructed to press one button on the response box if the word is a real Spanish word and will press the other button on the response box if the word is not a real Spanish word. Participants in experiment 1 will complete a visual word recognition task, which will require a lexical decision under the following experimental conditions. At the beginning of trial a series of six pounds signs (\#\#\#\#\#\#) will appear as a forward mask in the center of the screen for 500 milliseconds. The forward mask will immediately be followed by a masked lexical fragment—the prime—for 116 milliseconds which will be immediately followed by a real or nonce word—the target. 116 milliseconds for the forward mask was chosen following the findings of Carreiras and Perea \parencite*{Carreiras2002-mp} who showed that this timeframe was sufficient to show the syllabic effect and extending it to 166 milliseconds resulted in an accuracy–speed trade off. The trial will end once the participant has made their response which will clear the screen for one second before the forward mask of the following trial appears. The participants will see 16 practice trials that match the procedure described above for the experimental trials. Following the practice trials and the 160 experimental trials, the participants will complete the LexTALE and LexTALE-Esp vocabulary tasks using PsychoPy. The last task that participants will complete in their one laboratory visit is the bilingual language profile (BLP) in a Google Form developed by the creators.



%\textbf{DELETE ME The first refer to mainly scope and size of research project while the procedure section is usually much more in-depth. Here, you connect your procedures to those in already published articles when possible and given detailed descriptions of classifications and scales used in procedure. Essentially ensure the reader does not suspect anything is being hidden and the researcher is honest. Do not repeat any unnecessary information in subsequent experiments of the paper.}


%----------------------------------------------------------------------------------------
%	SECTION 3
%----------------------------------------------------------------------------------------

\section{Results}

Participants who have less than a 70 percent success rate on real word trials will be removed from the data (Number of participants will be reported). For the latency data, incorrect responses for remaining participants will be removed. Then a second pass will remove any correct individual participant responses under 300 milliseconds and over 2000 milliseconds. Finally, correct responses with latency data more than two standard deviations away from that participants average will also be removed (The percentage of trials removed will be reported here).

%\textbf{DELETE ME describe the analyses the researcher has done, but do not overload. Instead of creating a laundry list of statistics, create the story you want to tell using only the statistics that are related to addressing your problem. For each task, review the hypothesis, give the statistics and say what the result of the test means. Do not discuss the findings until you reach the discussion session. This is where tables and figures can help keep the paper looking clean and crisp instead of cluttered unorganized statistical test lists that are hard to follow. Figures show patterns while tables give details.}


%----------------------------------------------------------------------------------------
%	SECTION 4
%----------------------------------------------------------------------------------------

\section{Discussion}

%split into several small sections following each individual experiment’s results sections where applicable. The General Discussion steps back and begins with an overview of the problem and then the findings. A general rule of thumb is to keep this section shorter than the introduction. Only give limitation directly related to the current study, not the general limitation of the research or the field as a whole and be sure to give a good reason for why these limitations are not as bad as they sound on the surface.


%----------------------------------------------------------------------------------------
%	SECTION 5
%----------------------------------------------------------------------------------------

\section{References}

%Insert references for this chapter here.




