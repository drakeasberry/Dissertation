% Chapter Template

\chapter{Visual Word Recognition} % Main chapter title

\label{Chapter4} % Change X to a consecutive number; for referencing this chapter elsewhere, use \ref{Chapter4}
%----------------------------------------------------------------------------------------
%	SECTION 0
%----------------------------------------------------------------------------------------

\section{Abstract}

Give abstract here

Keywords: (list all words necessary)

%----------------------------------------------------------------------------------------
%	SECTION 1
%----------------------------------------------------------------------------------------

\section{Introduction}

Write my introduction to the syllabification article here (Will write the introduction LAST)
The section of your article most likely to be read, not skimmed or skipped. The first paragraph or two is the overview of the article: describe the problem, question or theory motivating the research. 

%-----------------------------------
%	SUBSECTION 1-1
%-----------------------------------
\subsection{Background}

In the second part, describe relevant theories, review past research and give more details on the current research question. Do not forget about signposting, which headings and subheadings can naturally create for the reader.

%-----------------------------------
%	SUBSECTION 1-2
%-----------------------------------

\subsection{Present Study}
The third section titled, “The Present Experiment” or “The Present Research”, follows and contains experimental descriptions and how they address the questions being asked.
For most articles, keep your introduction under 10 pages

%----------------------------------------------------------------------------------------
%	SECTION 2
%----------------------------------------------------------------------------------------

\section{Methods}

Write this section FIRST
is the section that describes how the research was conducted. A good one shows how well thought out the experiment design is and allows other researchers to easily replicate it. This section also follows a formula:

%-----------------------------------
%	SUBSECTION 2-1
%-----------------------------------
\subsection{Participants}

Give accurate descriptions of participant groups, how they were classified, number of participants, etc.

%-----------------------------------
%	SUBSECTION 2-2
%-----------------------------------

\subsection{Design}

Give an accurate idea of how the overall project was designed, what previous studies is it based off, theoretical principles, etc.

DO NOT write about what participants do in this section.

%-----------------------------------
%	SUBSECTION 2-3
%-----------------------------------
\subsection{Instrumentation}

Any special instrumentation could be included here (I am not sure that mine deserves a devoted section to instrumentation.


%-----------------------------------
%	SUBSECTION 2-4
%-----------------------------------

\subsection{Procedure}

The first refer to mainly scope and size of research project while the procedure section is usually much more in-depth. Here, you connect your procedures to those in already published articles when possible and given detailed descriptions of classifications and scales used in procedure. Essentially ensure the reader does not suspect anything is being hidden and the researcher is honest. Do not repeat any unnecessary information in subsequent experiments of the paper.


%----------------------------------------------------------------------------------------
%	SECTION 3
%----------------------------------------------------------------------------------------

\section{Results}

describe the analyses the researcher has done, but do not overload. Instead of creating a laundry list of statistics, create the story you want to tell using only the statistics that are related to addressing your problem. 

For each task, review the hypothesis, give the statistics and say what the result of the test means. 

Do not discuss the findings until you reach the discussion session. 

This is where tables and figures can help keep the paper looking clean and crisp instead of cluttered unorganized statistical test lists that are hard to follow. 

Figures show patterns while tables give details.


%----------------------------------------------------------------------------------------
%	SECTION 4
%----------------------------------------------------------------------------------------

\section{Discussion}

split into several small sections following each individual experiment’s results sections where applicable. 
The General Discussion steps back and begins with an overview of the problem and then the findings. A general rule of thumb is to keep this section shorter than the introduction. Only give limitation directly related to the current study, not the general limitation of the research or the field as a whole and be sure to give a good reason for why these limitations are not as bad as they sound on the surface.


%----------------------------------------------------------------------------------------
%	SECTION 5
%----------------------------------------------------------------------------------------

\section{References}

Insert references for this chapter here.