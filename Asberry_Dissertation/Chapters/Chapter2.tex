% Chapter Template

\chapter{Syllabification} % Main chapter title

\label{Chapter2} % Change X to a consecutive number; for referencing this chapter elsewhere, use \ref{Chapter2}

%----------------------------------------------------------------------------------------
%	SECTION 0
%----------------------------------------------------------------------------------------

\section{Abstract}

Give Syllabification abstract here

Keywords: (list all words necessary)

%----------------------------------------------------------------------------------------
%	SECTION 1
%----------------------------------------------------------------------------------------

\section{Introduction}

Talk about syllabification
differences in syllabification between spanish and english

The syllable intuition experiment is an important step in the process because how the three groups of participants syllabify the words will have a direct effect on how fast or useful the syllable is in their segmentation strategy. It is expected that syllabic intuitions will vary based on being a native speaker of Spanish, an early learner of Spanish or a late learner of Spanish. These differences are likely to stem from age of acquisition, language dominance or type of schooling. This experiment will begin to build a data source of Spanish–English bilinguals syllabification of Spanish words that will help to determine whether or not age of acquisition or language dominance are sources are different syllabification patterns. 

\textbf{DELETE ME Write my introduction to the syllabification article here (Will write the introduction LAST)
The section of your article most likely to be read, not skimmed or skipped. The first paragraph or two is the overview of the article: describe the problem, question or theory motivating the research.} 

%-----------------------------------
%	SUBSECTION 1-1
%-----------------------------------
\subsection{Background}

In the second part, describe relevant theories, review past research and give more details on the current research question. Do not forget about signposting, which headings and subheadings can naturally create for the reader.

%-----------------------------------
%	SUBSECTION 1-2
%-----------------------------------

\subsection{Present Study}


\textbf{DELETE ME The third section titled, “The Present Experiment” or “The Present Research”, follows and contains experimental descriptions and how they address the questions being asked.
For most articles, keep your introduction under 10 pages}

%----------------------------------------------------------------------------------------
%	SECTION 2
%----------------------------------------------------------------------------------------

\section{Methods}

Write this section FIRST
is the section that describes how the research was conducted. A good one shows how well thought out the experiment design is and allows other researchers to easily replicate it. This section also follows a formula:

%-----------------------------------
%	SUBSECTION 2-1
%-----------------------------------
\subsection{Participants}

The participants were split into three distinct populations of Spanish–English bilinguals. The Spanish–dominant group consisted of native speakers of Spanish that lived in Sonora, Mexico and learned English after \[INSERT AGE HERE\]. The English–dominant group were native English speakers that lived in Tucson, Arizona and learned Spanish after \[INSERT AGE HERE\]. The early bilingual group consisted of participants who lived in Tucson, Arizona, but were exposed to both English and Spanish before the \[INSERT AGE HERE\].
\emph{The participants will be placed into their appropriate group based off of the combined results of Bilingual Language Profile (BLP), the LexTALE and the LexTALE-Esp vocabulary tests.} 

Based off of responses in the BLP, participants who report proficiency in a language other than English and Spanish will be removed from the data analysis. According to the findings of Lemhöfer \& Broersma (2012), the LexTALE vocabulary tests can distinguish between lower intermediate (up to 59 percent), upper intermediate (60–80 percent) and advanced (above 80 percent) levels of proficiency based on average percent correct responses. Participants that score lower than \[INSERT CUTOFF PERCENTAGE HERE\] were excluded due to having too low of a proficiency score in one of the two languages. The Spanish–dominant group had a  \[INSERT PERCENTAGE HERE\] success rate in Spanish and it was higher than their English success rate. The English-dominant group scored \[INSERT PERCENTAGE HERE\] or higher in English and it was higher than their Spanish scores. The early bilingual group had a correctness score of \[INSERT PERCENTAGE HERE\] or more in both English and Spanish. \[INSERT \# OF PARTICIPANTS\] participants were excluded from the analysis due to falling below the minimum standards used to describe each group. \[INSERT \# OF PARTICIPANTS\]  participants were recruited from each bilingual population and are reported in the analysis section. 
\emph{During data collection, more participants than reported here were collected since it was expected that several participants would be removed for one or more of the reasons listed above. When more than \[INSERT \# OF PARTICIPANTS\] participants remained eligible after removing participants who did not fit the population criteria, a random sampling of \[INSERT \# OF PARTICIPANTS\] participants were selected from eligible pool of participants.}

\textbf{DELETE ME Give accurate descriptions of participant groups, how they were classified, number of participants, etc.}

%-----------------------------------
%	SUBSECTION 2-2
%-----------------------------------

\subsection{Design}
The same 24 critical word pairs used in experiment 1 will be used in experiment 2. \textbf{this is has seen a major overall since writing and now treated separately}

Following the syllable monitoring in experiment 1, the same participants will immediately complete the syllabic intuition experiment. It is important to note that no participant will see a repeat of any critical word previously seen in experiment 1 to avoid any effects of repetition priming. In order to accomplish this, the participants will see the other word of the critical word pairs that they saw in experiment 1. For the example word pair “balada–baldosa”, if the participant saw “balada” in experiment 1, they will see “baldosa” in the experiment 2 or vice versa. In order to counterbalance for hand dominance and visual presentation in this task, half of the participants will use the left response button to indicate a CVC response and the right response button to indicate a CV response while half of the participants will do the opposite. Half of the participants will visually see the CVC syllable orthographic representation on the left side of the critical item and the CV syllable orthographic representation on the right side of the critical item while the other half of the participants will see the opposite visual displays for syllable orthographic representations.

\textbf{DELETE ME Give an accurate idea of how the overall project was designed, what previous studies is it based off, theoretical principles, etc.
DO NOT write about what participants do in this section.}

%-----------------------------------
%	SUBSECTION 2-3
%-----------------------------------
\subsection{Instrumentation}
The LexTALE and the LexTALE-Esp are tasks used to correlate vocabulary knowledge and language proficiency in English and Spanish respectively (Izura, Cuetos, \& Brysbaert, 2014; Lemhöfer \& Broersma, 2012). The LexTALE-Esp has also been shown to discriminate well between highly proficient Spanish speaking participants with different language dominances (Ferré \& Brysbaert, 2017). The LexTALE and LexTALE-Esp tests will be used in order to group participants into the appropriate bilingual population—Spanish–dominant bilinguals of English (L2), English–dominant bilinguals of Spanish (L2) or Balanced bilinguals (exposed to both English and Spanish since early childhood). Again, the LexTALE is available online and has been designed to run in PRAAT, Matlab and Presentations, but the participants will take it using PsychoPy at the experiment testing location immediately following the experimental tasks. 
A second instrument used in this study is the Bilingual Language Profile (BLP), which is a survey based assessment tool for determining language dominance (Birdsong, Gertken, \& Amengual, 2012). It assesses language history, use, proficiency and attitudes of participants in less than 10 minutes. This assessment tool is commonly used in language studies with a focus on bilingualism and is available for free under the creative commons license. This tool allows researchers to collect information about participants demographics—name, age, sex, place of residence and educational background. It also allows for the participant to talk about their language history, use, proficiency and attitudes. For the current dissertation project, this survey will be administered while the participants are in the experiment testing location through Google Forms as designed by the creators of the BLP. Since no modification will be made to the BLP survey, the scores will be automatically calculated as designed by the creators. In all cases, the BLP survey will be completed after the experimental trials to avoid any confounding factors of language activation, Spanish or English, the participants will be able to choose whether they would like to take the survey with Spanish or English instructions.


\textbf{DELETE ME Any special instrumentation could be included here (I am not sure that mine deserves a devoted section to instrumentation.}


%-----------------------------------
%	SUBSECTION 2-4
%-----------------------------------

\subsection{Procedure}

The participants were instructed that they would see Spanish words presented individually on the computer screen and that each word would be followed by two options for the first syllable of the presented word. There was no time constraint for the participant to select the option that they considered to be the first syllable of the presented word and the experiment would not progress until the participant had indicated a response. In total, there were 10 practice trials that followed the same criteria and procedure as the 48 experimental trials presented to the participants. For each trial, the word in lowercase letters appeared in the center of the screen for 1500 milliseconds before being replaced by two syllable options---one a CV syllable and the other a CVC syllable---which remained on the screen, in all capital letters, until a response was entered. Following each response, the screen remained blank for 500 milliseconds before the next word appeared. For half of the participants, the CV syllable option was on the left side of the screen while the CVC syllable option was on the right. For the other half of the participants, the CVC syllable option was on the left side of the screen while the CV syllable option was on the right.


\textbf{combine this with first paragraph}
Participants will be instructed that a word will appear in the center of the screen and to indicate their response as quickly and accurately as possible. Participants will press one button on the response box if the first syllable of the word is a CVC structure (Consonant-Vowel-Consonant) and will press the other button on the response box if the first syllable of the word is a CV structure (Consonant-Vowel). The orthographic representation of each syllable structure type will also be printed on the screen to the left and the right of the critical word to ensure participants are not confused by the CV/CVC terminology. Participants will be given 5 practice trials followed by 24 experimental trials that each had their own set of directions differing only in letting the participant know whether they are about to enter a practice or experimental phase.

\textbf{DELETE ME WHEN FINISHED
The first refer to mainly scope and size of research project while the procedure section is usually much more in-depth. Here, you connect your procedures to those in already published articles when possible and given detailed descriptions of classifications and scales used in procedure. Essentially ensure the reader does not suspect anything is being hidden and the researcher is honest. Do not repeat any unnecessary information in subsequent experiments of the paper.}


%----------------------------------------------------------------------------------------
%	SECTION 3
%----------------------------------------------------------------------------------------

\section{Results}

The practice trials will be removed from the data collected and so that only the critical trials remain. Participants syllabification patterns will be analyzed by group and then compared across groups. One valuable piece of data that will be analyzed here is the success rate at determining the correct initial syllable of the Spanish words presented on screen. However, the types of error patterns is also valuable because a preference for CVC or CV type syllabification may emerge. 

\textbf{DELETE ME describe the analyses the researcher has done, but do not overload. Instead of creating a laundry list of statistics, create the story you want to tell using only the statistics that are related to addressing your problem. 
For each task, review the hypothesis, give the statistics and say what the result of the test means. 
Do not discuss the findings until you reach the discussion session. 
This is where tables and figures can help keep the paper looking clean and crisp instead of cluttered unorganized statistical test lists that are hard to follow. 
Figures show patterns while tables give details.}


%----------------------------------------------------------------------------------------
%	SECTION 4
%----------------------------------------------------------------------------------------

\section{Discussion}

split into several small sections following each individual experiment’s results sections where applicable. 
The General Discussion steps back and begins with an overview of the problem and then the findings. A general rule of thumb is to keep this section shorter than the introduction. Only give limitation directly related to the current study, not the general limitation of the research or the field as a whole and be sure to give a good reason for why these limitations are not as bad as they sound on the surface.


%----------------------------------------------------------------------------------------
%	SECTION 5
%----------------------------------------------------------------------------------------

\section{References}

Insert references for this chapter here.









