% Chapter Template

\chapter{Introduction} % Main chapter title

\label{Chapter1} % Change X to a consecutive number; for referencing this chapter elsewhere, use \ref{Chapter2}

%----------------------------------------------------------------------------------------
%	SECTION 0
%----------------------------------------------------------------------------------------

\section{Abstract}

Give abstract of entire dissertation here

Keywords: (list all words necessary)

%----------------------------------------------------------------------------------------
%	SECTION 1
%----------------------------------------------------------------------------------------

\section{Introduction}

This is a proposal for the three article dissertation that will be written in order to complete the degree requirements of the Second Language Acquisition and Teaching program (SLAT) at the University of Arizona. The main focus of this dissertation is on early and late Spanish–English bilingual populations and their ability to use the syllable in order to process spoken Spanish. The three articles that make up this dissertation will all focus on the same three speaker populations and will be written as three independent articles.

%-----------------------------------
%	SUBSECTION 1-1
%-----------------------------------
\subsection{Background}

A syllable is a pronounceable linguistic unit of a given language that generally contains a highly sonorant sound as it nucleus—typically a vowel in most languages. The syllable has been the focus of many speech processing studies in the past, but many avenues of research have yet to be investigated. Originally, the syllable made its highlight as spoken language processing unit when it was found that participants could detect syllables faster than they could detect individual phonemes \citep*{Savin1970-oy}. With the findings of several other studies, it became clear that speech processing was a complicated task that was not likely to be answered through finding a minimal perceptual unit. Specifically, one linguistic unit—phoneme, syllable, word, etc.— could not be the sole mechanism in which listeners of language break the speech stream into smaller or processable chunks \citep{Foss1973-ll,Healy1976-js,McNeill1973-bo}. Mehler and Hayes \parencite*{Mehler1981-wp} captured the need for a change in the direction of research regarding the syllable, “Traditionally, psycholinguistics research has invested the bulk of its efforts into uncovering the units used in speech processing. Although it is currently fashionable to claim that such work is pointless since it has no very clear outcome, many of the more meaningful advances in the field have come from projects whose framework included the problem of processing units.” They go on to delineate two different levels in which the research around the syllable could move forward: (1) The syllable as a phonological unit of the language which can efficiently explain the grammar of language and (2) The syllable as a unit which aides speech perception and comprehension. This dissertation will be comprised of three articles utilizing psycholinguistic methodologies that fall into the second level of investigation, which also builds off the researchers who began refocusing their investigations in this vein of syllable research in the early 1980s. Even with this new angle, the results of previous research began to reveal that even focusing on the syllable’s use in speech processing may be too large of a question. The research suggests that there are two distinct subprocesses of speech processing in which the syllable may play a role—speech segmentation and lexical access. The first article of this dissertation will focus on speech segmentation while the second and third articles will focus on lexical access.

%-----------------------------------
%	SUBSECTION 1-2
%-----------------------------------

\subsection{Present Study}
The three proposed articles for this dissertation will add information about syllable structure and the representation of the syllable in Spanish to the knowledge base of the field. Within the speech segmentation strain of research, several gaps still remain. The first being the bilingual degrees of the speaker populations being studied. Up to this point in time, the majority of the investigations have been conducted with monolingual speakers. Those that have investigated the role of the syllable in speech processing of bilinguals have generally compared the bilinguals against the monolingual speakers of the two languages. This dissertation will not utilize monolingual Spanish speaker controls, but instead will utilize three distinct bilingual populations of Spanish and English—late learners of L2 English, late learners of L2 Spanish and balanced bilinguals (early learners of both Spanish and English)—which will be compared against the other bilingual populations. This type of setup will allow for the control group to be the Spanish native L2 English speaker group and two test groups be early and late learners of Spanish. Many previous lexical access studies have generally been conducted using the visual word recognition paradigm to study speech processing in conjunction with the syllable while only Italian has really been investigated in this realm utilizing spoken word recognition techniques. Therefore, this dissertation will add to the knowledge base of the syllable’s role through replication studies using a different population of speakers in the visual word recognition while an expansion to other language groups will be the outcome for the spoken word recognition study. 
The overarching questions that underlie this dissertation project are: (1) Whether or not the common linguistic unit—the syllable—is available to Spanish–English bilinguals when processing spoken Spanish?, (2) Whether or not the syllable is a strategy used by Spanish–English bilinguals in speech segmentation and/or lexical access? and (3) Does the age of acquisition, early versus late bilinguals, affect the ability and efficiency in which Spanish–English bilinguals can make use of the syllable? The format for the remaining chapters of this dissertation project will be as follows in order to address the main overarching questions: chapter two will include two experiments that explore the representation of the syllable in the minds of Spanish–English bilinguals using a syllabic intuition two option forced-choice task and a segmentation task using an auditory monitoring technique. Chapter three will include a visual masked priming experiment with a lexical decision task that explores the syllable’s role in lexical access by Spanish–English bilinguals. It will again incorporate a syllabic intuition task to add more additional knowledge of syllabification differences across native speakers, early learners and late learners of Spanish. Chapter four will include a cross-modal fragment semantic priming experiment with a lexical decision task to investigate for further evidence of the use of the syllable to gain access to the mental lexicon of Spanish–English bilinguals. As in the previous two chapters, chapter 4 will also include a syllabic intuition task. Chapter five will then conclude the dissertation project by drawing overall conclusions and how all three individual studies were necessary to draw the conclusions that were borne out through the various methodologies used to explore the role of the syllable in Spanish speech processing in the three articles of the dissertation.


%----------------------------------------------------------------------------------------
%	SECTION 2
%----------------------------------------------------------------------------------------

\section{Methods}

Not needed

%-----------------------------------
%	SUBSECTION 2-1
%-----------------------------------
\subsection{Participants}


Probably still important here to some degree

Give accurate descriptions of participant groups, how they were classified, number of participants, etc. 


%----------------------------------------------------------------------------------------
%	SECTION 3
%----------------------------------------------------------------------------------------

\section{Results}

Is this needed?


%----------------------------------------------------------------------------------------
%	SECTION 4
%----------------------------------------------------------------------------------------

\section{Discussion}
Is this needed?


%----------------------------------------------------------------------------------------
%	SECTION 5
%----------------------------------------------------------------------------------------

\section{References}

Insert references for this chapter here.









