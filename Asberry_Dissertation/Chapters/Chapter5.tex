% Chapter Template

\chapter{Conclusion} % Main chapter title

\label{Chapter5} % Change X to a consecutive number; for referencing this chapter elsewhere, use \ref{Chapter5}

%----------------------------------------------------------------------------------------
%	SECTION 1
%----------------------------------------------------------------------------------------

\section{Main Section 1}

This proposal has provided a basic background and introduction to the project that will be extended in the formal dissertation. This project will encompass the more in-depth the background knowledge of the research related to the syllable’s role in speech processing. The dissertation will not only attempt to replicate previous findings with a different speaker population, but also fill in some of the research veins that have yet to be investigated. For example, French–English early bilinguals have been tested in a word segmentation paradigm, but a comparative group, Spanish–English early bilinguals, have yet to investigated. Given that Spanish and French share many similarities—i.e. Syllable timed and clear, unambiguous syllable boundaries—replicating the results of the French–English bilinguals with this new population would give additional support for a syllable-based segmentation strategy. In addition, the research proposed for this dissertation would give new insight on late bilinguals—speakers of English who learned Spanish as adults—which will elucidate several uses of the syllable. First, it will provide research data on different bilingual speaker population, and secondly, it will allow for a comparison of degrees of bilingualism. In other words, it will shed light on whether or not native speaker of a language such as English that is known not to implement a syllable-based segmentation strategy ever learn to utilize a syllable-based segmentation strategy when becoming bilingual in a language where native speakers do make productive use of the strategy.
The second main thread of research contained in the forthcoming dissertation will be on the syllable’s role in lexical access. Up to this point, it has mainly been tested in the visual priming paradigm, but Italian has also been studied under a cross-modal fragment priming paradigm. The dissertation will add to this literature by again investigating whether or not the path to lexical access reflects a syllabic representation as well as if the degree of bilingualism is a pertinent factor of its usefulness. In addition, it will expand our knowledge base by replicating the the Italian results in Spanish. 
Both lines of research will lean towards a better understanding how the critical age hypothesis and the role of schooling may relate to speech processing. For example, if a difference is found between the early bilingual group and the late bilingual groups, it would provide evidence that once a person has passed through puberty, learning new segmentation or lexical access strategies are no longer acquirable for speakers learning a language as an adult. If on the other hand, no difference is found between the early bilingual group and late bilingual groups, it would provide evidence that these strategies are still available and able to be learned by adult language learners. A third possibility exists in that a difference will be found between late learners of English bilingual population and the other two bilingual groups—balanced bilinguals and late learners of Spanish bilinguals. If this were the case, then a case could be drawn for the role of schooling in which syllables are taught as a means of breaking words into smaller chunks for Spanish native speakers by Spanish native speakers as the reason for the development of the syllable-based processing strategy. This case can only be built when the early bilingual group consists of speakers who grew up using both Spanish and English, but received formal education only in Spanish.
No results have been report as part of this proposal since no data has been collected under the provisions of the Internal Review Board (IRB) at the University of Arizona. However, the final dissertation project will collect data from participants through the use of several psycholinguistic methodologies. In chapter 2 of the dissertation, a syllable monitoring task will be used in the auditory modality to determine whether or not the syllable has some representation in the minds of Spanish speakers that can be utilized to segment spoken speech. A second task will also be used in the second chapter, which will fall under a syllabification task. This task will ask participants to indicate the first syllable of the word in a two-choice forced decision task. In chapter 3, a visual priming experiment will be used in combination with a lexical decision task in order to determine whether or not the syllable plays a role in the facilitation of lexical retrieval by bilingual speakers of Spanish. Finally in chapter 4, a cross-modal auditory fragment priming in combination with a visually display word lexical decision task will be used to further test the role of the syllable in lexical access of bilingual Spanish–English speakers. In all experiments, reaction times captured in the responses to monitoring and lexical decisions tasks will be the variable of interest in determining whether or not the syllable has a facilitatory effect for word segmentation or lexical access.


%-----------------------------------
%	SUBSECTION 1
%-----------------------------------
\subsection{Subsection 1}

may not need

%-----------------------------------
%	SUBSECTION 2
%-----------------------------------

\subsection{Subsection 2}
probably need something here

%----------------------------------------------------------------------------------------
%	SECTION 2
%----------------------------------------------------------------------------------------

\section{Main Section 2}

may not need

%----------------------------------------------------------------------------------------
%	SECTION 3
%----------------------------------------------------------------------------------------

\section{Main Section 3}

may not need

%-----------------------------------
%	SUBSECTION 1
%-----------------------------------
\subsection{Subsection 1}

May not need

%-----------------------------------
%	SUBSECTION 2
%-----------------------------------

\subsection{Subsection 2}
May not need
%----------------------------------------------------------------------------------------
%	SECTION 4
%----------------------------------------------------------------------------------------

\section{Main Section 4}

May not need